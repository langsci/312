\documentclass[output=paper,hidelinks]{langscibook}
\ChapterDOI{10.5281/zenodo.10186010}
\title{LFG and Continental West-Germanic languages}
\author{Gerlof Bouma\affiliation{University of Gothenburg}}
\abstract{This chapter presents an overview of LFG work on Continental West-Germanic languages. It starts out by giving a broad characterization of the languages that are part of this group, with a special focus on their clause layout, that is, the placement of verbs and arguments in a clause. After this, the different LFG approaches to modelling this layout are discussed, followed by a selection of clausal and verbal domain topics such as topicalization and left-dislocation, asymmetric coordination, cleft constructions, and argument ordering and realization. The chapter concludes by reviewing LFG analyses of topics from the nominal domain, namely determiner-adjective declension, preposition-determiner contraction, case indeterminacy and possessive doubling.}

\IfFileExists{../localcommands.tex}{
   \addbibresource{../localbibliography.bib}
   \addbibresource{thisvolume.bib}
   % add all extra packages you need to load to this file

\usepackage{tabularx}
\usepackage{multicol}
\usepackage{url}
\urlstyle{same}
%\usepackage{amsmath,amssymb}

% Tight underlining according to https://alexwlchan.net/2017/10/latex-underlines/
\usepackage{contour}
\usepackage[normalem]{ulem}
\renewcommand{\ULdepth}{1.8pt}
\contourlength{0.8pt}
\newcommand{\tightuline}[1]{%
  \uline{\phantom{#1}}%
  \llap{\contour{white}{#1}}}
  
\usepackage{listings}
\lstset{basicstyle=\ttfamily,tabsize=2,breaklines=true}

% \usepackage{langsci-basic}
\usepackage{langsci-optional}
\usepackage[danger]{langsci-lgr}
\usepackage{langsci-gb4e}
%\usepackage{langsci-linguex}
%\usepackage{langsci-forest-setup}
\usepackage[tikz]{langsci-avm} % added tikz flag, 29 July 21
% \usepackage{langsci-textipa}

\usepackage[linguistics,edges]{forest}
\usepackage{tikz-qtree}
\usetikzlibrary{positioning, tikzmark, arrows.meta, calc, matrix, shapes.symbols}
\usetikzlibrary{arrows, arrows.meta, shapes, chains, decorations.text}

%%%%%%%%%%%%%%%%%%%%% Packages for all chapters

% arrows and lines between structures
\usepackage{pst-node}

% lfg attributes and values, lines (relies on pst-node), lexical entries, phrase structure rules
\usepackage{packages/lfg-abbrevs}

% subfigures
\usepackage{subcaption}

% macros for small illustrations in the glossary
\usepackage{./packages/picins}

%%%%%%%%%%%%%%%%%%%%% Packages from contributors

% % Simpler Syntax packages
\usepackage{bm}
\tikzstyle{block} = [rectangle, draw, text width=5em, text centered, minimum height=3em]
\tikzstyle{line} = [draw, thick, -latex']

% Dependency packages
\usepackage{tikz-dependency}
%\usepackage{sdrt}

\usepackage{soul}

\usepackage[notipa]{ot-tableau}

% Historical
\usepackage{stackengine}
\usepackage{bigdelim}

% Morphology
\usepackage{./packages/prooftree}
\usepackage{arydshln}
\usepackage{stmaryrd}

% TAG
\usepackage{pbox}

\usepackage{langsci-branding}

   % %%%%%%%%% lang sci press commands

\newcommand*{\orcid}{}

\makeatletter
\let\thetitle\@title
\let\theauthor\@author
\makeatother

\newcommand{\togglepaper}[1][0]{
   \bibliography{../localbibliography}
   \papernote{\scriptsize\normalfont
     \theauthor.
     \titleTemp.
     To appear in:
     Dalrymple, Mary (ed.).
     Handbook of Lexical Functional Grammar.
     Berlin: Language Science Press. [preliminary page numbering]
   }
   \pagenumbering{roman}
   \setcounter{chapter}{#1}
   \addtocounter{chapter}{-1}
}

\DeclareOldFontCommand{\rm}{\normalfont\rmfamily}{\mathrm}
\DeclareOldFontCommand{\sf}{\normalfont\sffamily}{\mathsf}
\DeclareOldFontCommand{\tt}{\normalfont\ttfamily}{\mathtt}
\DeclareOldFontCommand{\bf}{\normalfont\bfseries}{\mathbf}
\DeclareOldFontCommand{\it}{\normalfont\itshape}{\mathit}
\makeatletter
\DeclareOldFontCommand{\sc}{\normalfont\scshape}{\@nomath\sc}
\makeatother

% Bug fix, 3 April 2021
\SetupAffiliations{output in groups = false,
                   separator between two = {\bigskip\\},
                   separator between multiple = {\bigskip\\},
                   separator between final two = {\bigskip\\}
                   }

% commands for all chapters
\setmathfont{LibertinusMath-Additions.otf}[range="22B8]

% punctuation between a sequence of years in a citation
% OLD: \renewcommand{\compcitedelim}{\multicitedelim}
\renewcommand{\compcitedelim}{\addcomma\space}

% \citegen with no parentheses around year
\providecommand{\citegenalt}[2][]{\citeauthor{#2}'s \citeyear*[#1]{#2}}

% avms with plain font, using langsci-avm package
\avmdefinestyle{plain}{attributes=\normalfont,values=\normalfont,types=\normalfont,extraskip=0.2em}
% avms with attributes and values in small caps, using langsci-avm package
\avmdefinestyle{fstr}{attributes=\scshape,values=\scshape,extraskip=0.2em}
% avms with attributes in small caps, values in plain font (from peter sells)
\avmdefinestyle{fstr-ps}{attributes=\scshape,values=\normalfont,extraskip=0.2em}

% reference to previous or following examples, from Stefan
%(\mex{1}) is like \next, referring to the next example
%(\mex{0}) is like \last, referring to the previous example, etc
\makeatletter
\newcommand{\mex}[1]{\the\numexpr\c@equation+#1\relax}
\makeatother

% do not add xspace before these
\xspaceaddexceptions{1234=|*\}\restrict\,}

% Several chapters use evnup -- this is verbatim from lingmacros.sty
\makeatletter
\def\evnup{\@ifnextchar[{\@evnup}{\@evnup[0pt]}}
\def\@evnup[#1]#2{\setbox1=\hbox{#2}%
\dimen1=\ht1 \advance\dimen1 by -.5\baselineskip%
\advance\dimen1 by -#1%
\leavevmode\lower\dimen1\box1}
\makeatother

% Centered entries in tables.  Requires array package.
\newcolumntype{P}[1]{>{\centering\arraybackslash}p{#1}}

% Reference to multiple figures, requested by Victoria Rosen
\newcommand{\figsref}[2]{Figures~\ref{#1}~and~\ref{#2}}
\newcommand{\figsrefthree}[3]{Figures~\ref{#1},~\ref{#2}~and~\ref{#3}}
\newcommand{\figsreffour}[4]{Figures~\ref{#1},~\ref{#2},~\ref{#3}~and~\ref{#4}}
\newcommand{\figsreffive}[5]{Figures~\ref{#1},~\ref{#2},~\ref{#3},~\ref{#4}~and~\ref{#5}}

% Semitic chapter:
\providecommand{\textchi}{χ}

% Prosody chapter
\makeatletter
\providecommand{\leftleadsto}{%
  \mathrel{\mathpalette\reflect@squig\relax}%
}
\newcommand{\reflect@squig}[2]{%
  \reflectbox{$\m@th#1$$\leadsto$}%
}
\makeatother
\newcommand\myrotaL[1]{\mathrel{\rotatebox[origin=c]{#1}{$\leadsto$}}}
\newcommand\Prosleftarrow{\myrotaL{-135}}
\newcommand\myrotaR[1]{\mathrel{\rotatebox[origin=c]{#1}{$\leftleadsto$}}}
\newcommand\Prosrightarrow{\myrotaR{135}}

% Core Concepts chapter
\newcommand{\anterm}[2]{#1\\#2}
\newcommand{\annode}[2]{#1\\#2}

% HPSG chapter
\newcommand{\HPSGphon}[1]{〈#1〉}
% for defining RSRL relations:
\newcommand{\HPSGsfl}{\enskip\ensuremath{\stackrel{\forall{}}{\Longleftarrow{}}}\enskip}
% AVM commands, valid only inside \avm{}
\avmdefinecommand {phon}[phon] { attributes=\itshape } % define a new \phon command
% Forest Set-up
\forestset
  {notin label above/.style={edge label={node[midway,sloped,above,inner sep=0pt]{\strut$\ni$}}},
    notin label below/.style={edge label={node[midway,sloped,below,inner sep=0pt]{\strut$\ni$}}},
  }

% Dependency chapter
\newcommand{\ua}{\ensuremath{\uparrow}}
\newcommand{\da}{\ensuremath{\downarrow}}
\forestset{
  dg edges/.style={for tree={parent anchor=south, child anchor=north,align=center,base=bottom},
                 where n children=0{tier=word,edge=dotted,calign with current edge}{}
                },
dg transfer/.style={edge path={\noexpand\path[\forestoption{edge}, rounded corners=3pt]
    % the line downwards
    (!u.parent anchor)-- +($(0,-l)-(0,4pt)$)-- +($(12pt,-l)-(0,4pt)$)
    % the horizontal line
    ($(!p.north west)+(0,l)-(0,20pt)$)--($(.north east)+(0,l)-(0,20pt)$)\forestoption{edge label};},!p.edge'={}},
% for Tesniere-style junctions
dg junction/.style={no edge, tikz+={\draw (!p.east)--(!.west) (.east)--(!n.west);}    }
}


% Glossary
\makeatletter % does not work with \newcommand
\def\namedlabel#1#2{\begingroup
   \def\@currentlabel{#2}%
   \phantomsection\label{#1}\endgroup
}
\makeatother


\renewcommand{\textopeno}{ɔ}
\providecommand{\textepsilon}{ɛ}

\renewcommand{\textbari}{ɨ}
\renewcommand{\textbaru}{ʉ}
\newcommand{\acutetextbari}{í̵}
\renewcommand{\textlyoghlig}{ɮ}
\renewcommand{\textdyoghlig}{ʤ}
\renewcommand{\textschwa}{ə}
\renewcommand{\textprimstress}{ˈ}
\newcommand{\texteng}{ŋ}
\renewcommand{\textbeltl}{ɬ}
\newcommand{\textramshorns}{ɤ}

\newbool{bookcompile}
\booltrue{bookcompile}
\newcommand{\bookorchapter}[2]{\ifbool{bookcompile}{#1}{#2}}




\renewcommand{\textsci}{ɪ}
\renewcommand{\textturnscripta}{ɒ}

\renewcommand{\textscripta}{ɑ}
\renewcommand{\textteshlig}{ʧ}
\providecommand{\textupsilon}{υ}
\renewcommand{\textyogh}{ʒ}
\newcommand{\textpolhook}{̨}

\renewcommand{\sectref}[1]{Section~\ref{#1}}

%\KOMAoptions{chapterprefix=true}

\renewcommand{\textturnv}{ʌ}
\renewcommand{\textrevepsilon}{ɜ}
\renewcommand{\textsecstress}{ˌ}
\renewcommand{\textscriptv}{ʋ}
\renewcommand{\textglotstop}{ʔ}
\renewcommand{\textrevglotstop}{ʕ}
%\newcommand{\textcrh}{ħ}
\renewcommand{\textesh}{ʃ}

% label for submitted and published chapters
\newcommand{\submitted}{{\color{red}Final version submitted to Language Science Press.}}
\newcommand{\published}{{\color{red}Final version published by
    Language Science Press, available at \url{https://langsci-press.org/catalog/book/312}.}}

% Treebank definitions
\definecolor{tomato}{rgb}{0.9,0,0}
\definecolor{kelly}{rgb}{0,0.65,0}

% Minimalism chapter
\newcommand\tr[1]{$<$\textcolor{gray}{#1}$>$}
\newcommand\gapline{\lower.1ex\hbox to 1.2em{\bf \ \hrulefill\ }}
\newcommand\cnom{{\llap{[}}Case:Nom{\rlap{]}}}
\newcommand\cacc{{\llap{[}}Case:Acc{\rlap{]}}}
\newcommand\tpres{{\llap{[}}Tns:Pres{\rlap{]}}}
\newcommand\fstackwe{{\llap{[}}Tns:Pres{\rlap{]}}\\{\llap{[}}Pers:1{\rlap{]}}\\{\llap{[}}Num:Pl{\rlap{]}}}
\newcommand\fstackone{{\llap{[}}Tns:Past{\rlap{]}}\\{\llap{[}}Pers:\ {\rlap{]}}\\{\llap{[}}Num:\ {\rlap{]}}}
\newcommand\fstacktwo{{\llap{[}}Pers:3{\rlap{]}}\\{\llap{[}}Num:Pl{\rlap{]}}\\{\llap{[}}Case:\ {\rlap{]}}}
\newcommand\fstackthr{{\llap{[}}Tns:Past{\rlap{]}}\\{\llap{[}}Pers:3{\rlap{]}}\\{\llap{[}}Num:Pl{\rlap{]}}} 
\newcommand\fstackfou{{\llap{[}}Pers:3{\rlap{]}}\\{\llap{[}}Num:Pl{\rlap{]}}\\{\llap{[}}Case:Nom{\rlap{]}}}
\newcommand\fstackonefill{{\llap{[}}Tns:Past{\rlap{]}}\\{\llap{[}}Pers:3{\rlap{]}}\\%
  {\llap{[}}Num:Pl{\rlap{]}}}
\newcommand\fstackoneint%
    {{\llap{[}}{\bf Tns:Past}{\rlap{]}}\\{\llap{[}}Pers:\ {\rlap{]}}\\{\llap{[}}Num:\ {\rlap{]}}}
\newcommand\fstacktwoint%
    {{\llap{[}}{\bf Pers:3}{\rlap{]}}\\{\llap{[}}{\bf Num:Pl}{\rlap{]}}\\{\llap{[}}Case:\ {\rlap{]}}}
\newcommand\fstackthrchk%
    {{\llap{[}}{\bf Tns:Past}{\rlap{]}}\\{\llap{[}}{Pers:3}{\rlap{]}}\\%
      {\llap{[}}Num:Pl{\rlap{]}}} 
\newcommand\fstackfouchk%
    {{\llap{[}}{\bf Pers:3}{\rlap{]}}\\{\llap{[}}{\bf Num:Pl}{\rlap{]}}\\%
      {\llap{[}}Case:Nom{\rlap{]}}}
\newcommand\uinfl{{\llap{[}}Infl:\ \ {\rlap{]}}}
\newcommand\inflpass{{\llap{[}}Infl:Pass{\rlap{]}}}
\newcommand\fepp{{\llap{[}}EPP{\rlap{]}}}
\newcommand\sepp{{\llap{[}}\st{EPP}{\rlap{]}}}
\newcommand\rdash{\rlap{\hbox to 24em{\hfill (dashed lines represent
      information flow)}}}


% Computational chapter
\usepackage{./packages/kaplan}
\renewcommand{\red}{\color{lsLightWine}}

% Sinitic
\newcommand{\FRAME}{\textsc{frame}\xspace}
\newcommand{\arglistit}[1]{{\textlangle}\textit{#1}{\textrangle}}

%WestGermanic
\newcommand{\streep}[1]{\mbox{\rule{1pt}{0pt}\rule[.5ex]{#1}{.5pt}\rule{-1pt}{0pt}\rule{-#1}{0pt}}}

\newcommand{\hspaceThis}[1]{\hphantom{#1}}


\newcommand{\FIG}{\textsc{figure}}
\newcommand{\GR}{\textsc{ground}}

%%%%% Morphology
% Single quote
\newcommand{\asquote}[1]{`{#1}'} % Single quotes
\newcommand{\atrns}[1]{\asquote{#1}} % Translation
\newcommand{\attrns}[1]{(\asquote{#1})} % Translation
\newcommand{\ascare}[1]{\asquote{#1}} % Scare quotes
\newcommand{\aqterm}[1]{\asquote{#1}} % Quoted terms
% Double quote
\newcommand{\adquote}[1]{``{#1}''} % Double quotes
\newcommand{\aquoot}[1]{\adquote{#1}} % Quotes
% Italics
\newcommand{\aword}[1]{\textit{#1}}  % mention of word
\newcommand{\aterm}[1]{\textit{#1}}
% Small caps
\newcommand{\amg}[1]{{\textsc{\MakeLowercase{#1}}}}
\newcommand{\ali}[1]{\MakeLowercase{\textsc{#1}}}
\newcommand{\feat}[1]{{\textsc{#1}}}
\newcommand{\val}[1]{\textsc{#1}}
\newcommand{\pred}[1]{\textsc{#1}}
\newcommand{\predvall}[1]{\textsc{#1}}
% Misc commands
\newcommand{\exrr}[2][]{(\ref{ex:#2}{#1})}
\newcommand{\csn}[3][t]{\begin{tabular}[#1]{@{\strut}c@{\strut}}#2\\#3\end{tabular}}
\newcommand{\sem}[2][]{\ensuremath{\left\llbracket \mbox{#2} \right\rrbracket^{#1}}}
\newcommand{\apf}[2][\ensuremath{\sigma}]{\ensuremath{\langle}#2,#1\ensuremath{\rangle}}
\newcommand{\formula}[2][t]{\ensuremath{\begin{array}[#1]{@{\strut}l@{\strut}}#2%
                                         \end{array}}}
\newcommand{\Down}{$\downarrow$}
\newcommand{\Up}{$\uparrow$}
\newcommand{\updown}{$\uparrow=\downarrow$}
\newcommand{\upsigb}{\mbox{\ensuremath{\uparrow\hspace{-0.35em}_\sigma}}}
\newcommand{\lrfg}{L\textsubscript{R}FG} 
\newcommand{\dmroot}{\ensuremath{\sqrt{\hspace{1em}}}}
\newcommand{\amother}{\mbox{\ensuremath{\hat{\raisebox{-.25ex}{\ensuremath{\ast}}}}}}
\newcommand{\expone}{\ensuremath{\xrightarrow{\nu}}}
\newcommand{\sig}{\mbox{$_\sigma\,$}}
\newcommand{\aset}[1]{\{#1\}}
\newcommand{\linimp}{\mbox{\ensuremath{\,\multimap\,}}}
\newcommand{\fsfunc}{\ensuremath{\Phi}\hspace*{-.15em}}
\newcommand{\cons}[1]{\ensuremath{\mbox{\textbf{\textup{#1}}}}}
\newcommand{\amic}[1][]{\cons{MostInformative$_c$}{#1}}
\newcommand{\amif}[1][]{\cons{MostInformative$_f$}{#1}}
\newcommand{\amis}[1][]{\cons{MostInformative$_s$}{#1}}
\newcommand{\amsp}[1][]{\cons{MostSpecific}{#1}}

%Glue
\newcommand{\glues}{Glue Semantics} % macro for consistency
\newcommand{\glue}{Glue} % macro for consistency
\newcommand{\lfgglue}{LFG$+$Glue} 
\newcommand{\scare}[1]{`{#1}'} % Scare quotes
\newcommand{\word}[1]{\textit{#1}}  % mention of word
\newcommand{\dquote}[1]{``{#1}''} % Double quotes
\newcommand{\high}[1]{\textit{#1}} % highlight (italicize)
\newcommand{\laml}{{L}} 
% Left interpretation double bracket
\newcommand{\Lsem}{\ensuremath{\left\llbracket}} 
% Right interpretation double bracket
\newcommand{\Rsem}{\ensuremath{\right\rrbracket}} 
\newcommand{\nohigh}[1]{{#1}} % nohighlight (regular font)
% Linear implication elimination
\newcommand{\linimpE}{\mbox{\small\ensuremath{\multimap_{\mathcal{E}}}}}
% Linear implication introduction, plain
\newcommand{\linimpI}{\mbox{\small\ensuremath{\multimap_{\mathcal{I}}}}}
% Linear implication introduction, with flag
\newcommand{\linimpIi}[1]{\mbox{\small\ensuremath{\multimap_{{\mathcal{I}},#1}}}}
% Linear universal elimination
\newcommand{\forallE}{\mbox{\small\ensuremath{\forall_{{\mathcal{E}}}}}}
% Tensor elimination
\newcommand{\tensorEij}[2]{\mbox{\small\ensuremath{\otimes_{{\mathcal{E}},#1,#2}}}}
% CG forward slash
\newcommand{\fs}{\ensuremath{/}} 
% s-structure mapping, no space after                                     
\newcommand{\sigb}{\mbox{$_\sigma$}}
% uparrow with s-structure mapping, with small space after  
\newcommand{\upsig}{\mbox{\ensuremath{\uparrow\hspace{-0.35em}_\sigma\,}}}
\newcommand{\fsa}[1]{\textit{#1}}
\newcommand{\sqz}[1]{#1}
% Angled brackets (types, etc.)
\newcommand{\bracket}[1]{\ensuremath{\left\langle\mbox{\textit{#1}}\right\rangle}}
% glue logic string term
\newcommand{\gterm}[1]{\ensuremath{\mbox{\textup{\textit{#1}}}}}
% abstract grammatical formative
\newcommand{\gform}[1]{\ensuremath{\mbox{\textsc{\textup{#1}}}}}
% let
\newcommand{\llet}[3]{\ensuremath{\mbox{\textsf{let}}~{#1}~\mbox{\textsf{be}}~{#2}~\mbox{\textsf{in}}~{#3}}}
% Word-adorned proof steps
\providecommand{\vformula}[2]{%
  \begin{array}[b]{l}
    \mbox{\textbf{\textit{#1}}}\\%[-0.5ex]
    \formula{#2}
  \end{array}
}

%TAG
\newcommand{\fm}[1]{\textsc{#1}}
\newcommand{\struc}[1]{{#1-struc\-ture}}
\newcommand{\func}[1]{\mbox{#1-function}}
\newcommand{\fstruc}{\struc{f}}
\newcommand{\cstruc}{\struc{c}}
\newcommand{\sstruc}{\struc{s}}
\newcommand{\astruc}{\struc{a}}
\newcommand{\nodelabels}[2]{\rlap{\ensuremath{^{#1}_{#2}}}}
\newcommand{\footnode}{\rlap{\ensuremath{^{*}}}}
\newcommand{\nafootnode}{\rlap{\ensuremath{^{*}_{\nalabel}}}}
\newcommand{\nanode}{\rlap{\ensuremath{_{\nalabel}}}}
\newcommand{\AdjConstrText}[1]{\textnormal{\small #1}}
\newcommand{\nalabel}{\AdjConstrText{NA}}

%Case
\newcommand{\MID}{\textsc{mid}{}\xspace}

%font commands added April 2023 for Control and Case chapters
\def\textthorn{þ}
\def\texteth{ð}
\def\textinvscr{ʁ}
\def\textcrh{ħ}
\def\textgamma{ɣ}

% Coordination
\newcommand{\CONJ}{\textsc{conj}{}\xspace}
\newcommand*{\phtm}[1]{\setbox0=\hbox{#1}\hspace{\wd0}}
\newcommand{\ggl}{\hfill(Google)}
\newcommand{\nkjp}{\hfill(NKJP)}

% LDDs
\newcommand{\ubd}{\attr{ubd}\xspace}
% \newcommand{\disattr}[1]{\blue \attr{#1}}  % on topic/focus path
% \newcommand{\proattr}[1]{\green\attr{#1}}  % On Q/Relpro path
\newcommand{\disattr}[1]{\color{lsMidBlue}\attr{#1}}  % on topic/focus path
\newcommand{\proattr}[1]{\color{lsMidGreen}\attr{#1}}  % On Q/Relpro path
\newcommand{\eestring}{\mbox{$e$}\xspace}
\providecommand{\disj}[1]{\{\attr{#1}\}}
\providecommand{\estring}{\mb{\epsilon}}
\providecommand{\termcomp}[1]{\attr{\backslash {#1}}}
\newcommand{\templatecall}[2]{{\small @}(\attr{#1}\ \attr{#2})}
\newcommand{\xlgf}[1]{(\leftarrow\ \attr{#1})} 
\newcommand{\xrgf}[1]{(\rightarrow\ \attr{#1})}
\newcommand{\rval}[2]{\annobox {\xrgf{#1}\teq\attr{#2}}}
\newcommand{\memb}[1]{\annobox {\downarrow\, \in \xugf{#1}}}
\newcommand{\lgf}[1]{\annobox {\xlgf{#1}}}
\newcommand{\rgf}[1]{\annobox {\xrgf{#1}}}
\newcommand{\rvalc}[2]{\annobox {\xrgf{#1}\teqc\attr{#2}}}
\newcommand{\xgfu}[1]{(\attr{#1}\uparrow)}
\newcommand{\gfu}[1]{\annobox {\xgfu{#1}}}
\newcommand{\nmemb}[3]{\annobox {{#1}\, \in \ngf{#2}{#3}}}
\newcommand{\dgf}[1]{\annobox {\xdgf{#1}}}
\newcommand{\predsfraise}[3]{\annobox {\xugf{pred}\teq\semformraise{#1}{#2}{#3}}}
\newcommand{\semformraise}[3]{\annobox {\textrm{`}\hspace{-.05em}\attr{#1}\langle\attr{#2}\rangle{\attr{#3}}\textrm{'}}}
\newcommand{\teqc}{\hspace{-.1667em}=_c\hspace{-.1667em}} 
\newcommand{\lval}[2]{\annobox {\xlgf{#1}\teq\attr{#2}}}
\newcommand{\xgfd}[1]{(\attr{#1}\downarrow)}
\newcommand{\gfd}[1]{\annobox {\xgfd{#1}}}
\newcommand{\gap}{\rule{.75em}{.5pt}\ }
\newcommand{\gapp}{\rule{.75em}{.5pt}$_p$\ }

% Mapping
% Avoid having to write 'argument structure' a million times
\newcommand{\argstruc}{argument structure}
\newcommand{\Argstruc}{Argument structure}
\newcommand{\emptybracks}{\ensuremath{[\;\;]}}
\newcommand{\emptycurlybracks}{\ensuremath{\{\;\;\}}}
% Drawing lines in structures
\newcommand{\strucconnect}[6]{%
\draw[-stealth] (#1) to[out=#5, in=#6] node[pos=#3, above]{#4} (#2);%
}
\newcommand{\strucconnectdashed}[6]{%
\draw[-stealth, dashed] (#1) to[out=#5, in=#6] node[pos=#3, above]{#4} (#2);%
}
% Attributes for s-structures in the style of lfg-abbrevs.sty
\newcommand{\ARGnum}[1]{\textsc{arg}\textsubscript{#1}}
% Drawing mapping lines
\newcommand{\maplink}[2]{%
\begin{tikzpicture}[baseline=(A.base)]
\node(A){#1\strut};
\node[below = 3ex of A](B){\pbox{\textwidth}{#2}};
\draw ([yshift=-1ex]A.base)--(B);
% \draw (A)--(B);
\end{tikzpicture}}
% long line for extra features
\newcommand{\longmaplink}[2]{%
\begin{tikzpicture}[baseline=(A.base)]
\node(A){#1\strut};
\node[below = 3ex of A](B){\pbox{\textwidth}{#2}};
\draw ([yshift=2.5ex]A.base)--(B);
% \draw (A)--(B);
\end{tikzpicture}%
}
% For drawing upward
\newcommand{\maplinkup}[2]{%
\begin{tikzpicture}[baseline=(A.base)]
\node(A){#1};
\node[above = 3ex of A, anchor=base](B){#2};
\draw (A)--(B);
\end{tikzpicture}}
% Above with arrow going down (for argument adding processes)
\newcommand{\argumentadd}[2]{%
\begin{tikzpicture}[baseline=(A.base)]
\node(A){#1};
\node[above = 3ex of A, anchor=base](B){#2};
\draw[latex-] ([yshift=2ex]A.base)--([yshift=-1ex]B.center);
\end{tikzpicture}}
% Going up to the left
\newcommand{\maplinkupleft}[2]{%
\begin{tikzpicture}[baseline=(A.base)]
\node(A){#1};
\node[above left = 3ex of A, anchor=base](B){#2};
\draw (A)--(B);
\end{tikzpicture}}
% Going up to the right
\newcommand{\maplinkupright}[2]{%
\begin{tikzpicture}[baseline=(A.base)]
\node(A){#1};
\node[above right = 3ex of A, anchor=base](B){#2};
\draw (A)--(B);
\end{tikzpicture}}
% Argument fusion
\newenvironment{tikzsentence}{\begin{tikzpicture}[baseline=0pt, 
  anchor=base, outer sep=0pt, ampersand replacement=\&
   ]}{\end{tikzpicture}}
\newcommand{\Subnode}[2]{\subnode[inner sep=1pt]{#1}{#2\strut}}
\newcommand{\connectbelow}[3]{\draw[inner sep=0pt] ([yshift=0.5ex]#1.south) -- ++ (south:#3ex)
  -| ([yshift=0.5ex]#2.south);}
\newcommand{\connectabove}[3]{\draw[inner sep=0pt] ([yshift=0ex]#1.north) -- ++ (north:#3ex)
  -| ([yshift=0ex]#2.north);}
  
\newcommand{\ASNode}[2]{\tikz[remember picture,baseline=(#1.base)] \node [anchor=base] (#1) {#2};}

% Austronesian
\newcommand{\LV}{\textsc{lv}\xspace}
\newcommand{\IV}{\textsc{iv}\xspace}
\newcommand{\DV}{\textsc{dv}\xspace}
\newcommand{\PV}{\textsc{pv}\xspace}
\newcommand{\AV}{\textsc{av}\xspace}
\newcommand{\UV}{\textsc{uv}\xspace}

\apptocmd{\appendix}
         {\bookmarksetup{startatroot}}
         {}
         {%
           \AtEndDocument{\typeout{langscibook Warning:}
                          \typeout{It was not possible to set option 'staratroot'}
                          \typeout{for appendix in the backmatter.}}
         }

   %% hyphenation points for line breaks
%% Normally, automatic hyphenation in LaTeX is very good
%% If a word is mis-hyphenated, add it to this file
%%
%% add information to TeX file before \begin{document} with:
%% %% hyphenation points for line breaks
%% Normally, automatic hyphenation in LaTeX is very good
%% If a word is mis-hyphenated, add it to this file
%%
%% add information to TeX file before \begin{document} with:
%% %% hyphenation points for line breaks
%% Normally, automatic hyphenation in LaTeX is very good
%% If a word is mis-hyphenated, add it to this file
%%
%% add information to TeX file before \begin{document} with:
%% \include{localhyphenation}
\hyphenation{
Aus-tin
Bel-ya-ev
Bres-nan
Chom-sky
Eng-lish
Geo-Gram
INESS
Inkelas
Kaplan
Kok-ko-ni-dis
Lacz-kó
Lam-ping
Lu-ra-ghi
Lund-quist
Mcho-mbo
Meu-rer
Nord-lin-ger
PASSIVE
Pa-no-va
Pol-lard
Pro-sod-ic
Prze-piór-kow-ski
Ram-chand
Sa-mo-ye-dic
Tsu-no-da
WCCFL
Wam-ba-ya
Warl-pi-ri
Wes-coat
Wo-lof
Zae-nen
accord-ing
an-a-phor-ic
ana-phor
christ-church
co-description
co-present
con-figur-ation-al
in-effa-bil-ity
mor-phe-mic
mor-pheme
non-com-po-si-tion-al
pros-o-dy
referanse-grammatikk
rep-re-sent
Schätz-le
term-hood
Kip-ar-sky
Kok-ko-ni
Chi-che-\^wa
au-ton-o-mous
Al-si-na
Ma-tsu-mo-to
}

\hyphenation{
Aus-tin
Bel-ya-ev
Bres-nan
Chom-sky
Eng-lish
Geo-Gram
INESS
Inkelas
Kaplan
Kok-ko-ni-dis
Lacz-kó
Lam-ping
Lu-ra-ghi
Lund-quist
Mcho-mbo
Meu-rer
Nord-lin-ger
PASSIVE
Pa-no-va
Pol-lard
Pro-sod-ic
Prze-piór-kow-ski
Ram-chand
Sa-mo-ye-dic
Tsu-no-da
WCCFL
Wam-ba-ya
Warl-pi-ri
Wes-coat
Wo-lof
Zae-nen
accord-ing
an-a-phor-ic
ana-phor
christ-church
co-description
co-present
con-figur-ation-al
in-effa-bil-ity
mor-phe-mic
mor-pheme
non-com-po-si-tion-al
pros-o-dy
referanse-grammatikk
rep-re-sent
Schätz-le
term-hood
Kip-ar-sky
Kok-ko-ni
Chi-che-\^wa
au-ton-o-mous
Al-si-na
Ma-tsu-mo-to
}

\hyphenation{
Aus-tin
Bel-ya-ev
Bres-nan
Chom-sky
Eng-lish
Geo-Gram
INESS
Inkelas
Kaplan
Kok-ko-ni-dis
Lacz-kó
Lam-ping
Lu-ra-ghi
Lund-quist
Mcho-mbo
Meu-rer
Nord-lin-ger
PASSIVE
Pa-no-va
Pol-lard
Pro-sod-ic
Prze-piór-kow-ski
Ram-chand
Sa-mo-ye-dic
Tsu-no-da
WCCFL
Wam-ba-ya
Warl-pi-ri
Wes-coat
Wo-lof
Zae-nen
accord-ing
an-a-phor-ic
ana-phor
christ-church
co-description
co-present
con-figur-ation-al
in-effa-bil-ity
mor-phe-mic
mor-pheme
non-com-po-si-tion-al
pros-o-dy
referanse-grammatikk
rep-re-sent
Schätz-le
term-hood
Kip-ar-sky
Kok-ko-ni
Chi-che-\^wa
au-ton-o-mous
Al-si-na
Ma-tsu-mo-to
}

   \togglepaper[31]%%chapternumber
}{}

\begin{document}
\maketitle
\label{chap:WestGermanic}

\section{Introduction\label{sec:Germanic:intro}}

This chapter is concerned with the Lexical-Functional Grammar
treatment of the present-day West-Germanic languages \textit{sans}
English or Scots. This group is sometimes referred to as Continental
West Germanic \citep{zwart:2008}, as it is mostly comprised of
Germanic languages\footnote{Unless it is relevant to the discussion, I
will use the term \textit{language} in a broad sense that ignores
matters like the language/language
variety/regiolect/dialect/etcetera's political status, whether a
language label covers a homogeneous or heterogeneous group of
subvarieties, or whether it is mutually intelligible with languages
that do not fall under the same label.} spoken in countries on the
European mainland: from Belgium and the Netherlands, through
Luxembourg and Germany, to Switzerland, the northernmost parts of
Italy, and Austria; and in addition in smaller regions bordering these
countries. In spite of the label `continental', the group of
Continental West Germanic languages (henceforth: CWG) also contains
languages like Yiddish (spoken in Israel, North America and elsewhere), Afrikaans (South Africa and Namibia), Dutch in the
Antilles and Suriname (in terms of language politics part of the same
standardization body as Belgium and the Netherlands), and German
heritage variants in the Americas \citep{putnam:2011} and Siberia
\citep{andersen:2016}, to name but a few group members outside of
continental Europe.

In terms of L1 speakers, the largest of these languages are the standard
varieties of German and Dutch, with circa 75 and 25 million speakers,
respectively.\footnote{Counts based on
\citet{EberhardSimonsFennig2019}.}  Their status as standardized
national languages in multiple countries also means they are supported
by strong academic infrastructures. It is therefore not surprising
that most of the work on CWG in LFG is done on these two
languages. Standard German and Dutch\footnote{Unless the context requires otherwise,  I will use \textit{German} and \textit{Dutch} without further modification to refer to the standardized, national language varieties of these two CWG languages.} will figure prominently in this
chapter. This is merely a reflection of their salience in the LFG
literature, and should not be interpreted linguistically, for instance
as a sign of them being more typical CWG languages than other members of the
language group.

A comprehensive inventory of West-Germanic languages with demographic
and linguistic information can be found in
Ethnologue\footnote{\url{https://www.ethnologue.com/subgroups/west-0},
consulted July 2022} \citep{EberhardSimonsFennig2019}. Bibliographic
data on West-Germanic can be found in
Glottolog\footnote{\url{https://glottolog.org/resource/languoid/id/west2793},
consulted July 2022} \citep{glottolog}. Note that neither of these
resources distinguishes a CWG branch in their taxonomies. For an
overview of the syntactic traits of Continental West Germanic, I refer
the reader to \citet{zwart:2008}. An accessible description of how
German is syntactically different from English and from the
North-Germanic languages can be found in the introductory chapters of
\citet{haider:2010}, with arguments that in many cases carry over to
the other CWG languages.


\subsection{A general picture of Continental West Germanic, with a focus on the clause}

In this subsection we will discuss some of the syntactic traits of
CWG. Our focus will be on the clause/verbal domain, since this has
been the main interest of the LFG literature on CWG. We will discuss
the nominal domain more briefly. The purpose of this subsection is
twofold. First, it gives a very general impression of CWG syntax and
indicates how it differs from its North- and
West-Germanic neighbours. Secondly, it introduces some of the
language-particular background needed to understand the individual LFG
analyses discussed in the rest of the chapter.


\subsubsection{Clause layout\label{sec:Germanic:clauselayout}}
A prominent syntactic feature of CWG languages is the combination of
\textit{asymmetric verb second} together with \textit{verb final}
\citep{zwart:2008,haider:2010}. The former label characterizes a
clause structure in which the finite verb in main clauses, but
\textit{not} in subordinate clauses, is preceded by exactly one
constituent, which can have a wide range of grammatical functions. The
latter label covers the generalization that any verbal material that
is not in second position -- finite verbs in subordinate clauses and
non-finite verbs in general -- clusters together towards the end of
the clause, potentially following arguments and adjuncts. As such, CWG
contrasts with Modern English, which lacks both pervasive verb second
in main clauses and verb finality, and follows a more rigid
subject-verb-complement order. CWG also differs from the present-day
North-Germanic languages, which can be characterized as combining verb
second with subject-verb-complement order.\footnote{Two remarks are in
order with respect to this characterization of CWG clause
layout. First, as it can be used to demarcate CWG from English as well
as from North-Germanic, it gives some linguistic substance to the
pooling of CWG languages into one group, as we do in this
chapter. Secondly, and somewhat weakening the first point, once we
start to look closer at individual CWG languages, we find deviations
from the general pattern. Modern Yiddish in particular fits the
description poorly, both in terms of the verb-second pattern and the
verb-final pattern. It is beyond the scope of this chapter to go into
all the exceptions, but some of them will be discussed towards the end
of this section and in the context of LFG analyses of these
exceptions.}

For the discussion of the layout of a CWG clause, it is helpful to
make use of the so-called \textit{topological field model} of the
clause, which can be found in traditional descriptions of German and
Dutch and in reference grammars like
\citet{zifonun-etal:1997:vol7} and \citet{haeseryn-ed:1997}. In this model, the
layout of a clause is described in terms of linearly ordered fields,
and different word order variants associated with different clause
types are obtained by assigning constituents to different fields. The
field schema we use in this chapter is given in~(\ref{ex:topmodel}).
%
\begin{exe}
  \ex\label{ex:topmodel} lead $\|$ Vorfeld | left bracket | Mittelfeld | right bracket | Nachfeld  $\|$ tail
\end{exe}
%
\largerpage
In a main clause, the \textit{left bracket} (lb) and \textit{right
  bracket} (rb) are reserved for verbal material: a single,
second-position finite verb is in the left bracket, and any other
verbs are in the verb cluster in the right bracket. Between the
brackets there is the \textit{Mittelfeld} `middle field' (Mf), which
may contain any number of constituents. The \textit{Vorfeld}
`prefield' (Vf) is the designated place for the single constituent
preceding the finite verb, whereas the \textit{Nachfeld} `postfield'
(Nf) may contain several items, and is typically reserved for heavier
constituents like clausal arguments, extraposed relative clauses and
adverbial prepositional phrases. The \textit{lead} and
\textit{tail} fields\footnote{The terminology around these last two fields is
not as established as for the fields that are part of the clause
proper. The lead is for instance also known as \textit{Vorvorfeld}
`pre-prefield' or \textit{linkes Au{\ss}enfeld} `left outfield', and the
tail as \textit{Nachnachfeld} `post-postfield' or \textit{rechtes
  Au{\ss}enfeld} `right outfield'.} host material that is more loosely
connected to the clause, such as vocatives and hanging topics. Note
that not every field needs to contain material. Examples of different
declarative main clause types are given in~(\ref{ex:WGerm:2}).

%
\begin{exe}
  \ex\label{ex:WGerm:2} Dutch 
  \begin{xlist}
  \ex Subject-initial declarative:\\
      {\glll
        Vf\streep{3.25em} {}     lb\streep{1em}   Mf\streep{4.25em}    {}  rb\streep{2.5em}\\
        De draken doen Doris dadelijk duizelen.\\
           the dragons make.\PRS.\PL{} Doris immediately feel.dizzy.\INF{}\\}
           \glt `The dragons immediately make Doris feel dizzy.'
        
   \ex Object-initial declarative:\\    
      {\glll%
        Vf\streep{1em}      lb\streep{1em}   Mf\streep{6.75em} {} {}  rb\streep{2.5em}\\
        Doris doen de draken dadelijk duizelen.\\
        Doris make.\PRS.\PL{} the dragons immediately feel.dizzy.\INF{}\\}
   
   \ex Sentence adverb-initial declarative:\\   
      {\glll%
        Vf\streep{2.15em}      lb\streep{1em}   Mf\streep{6em} {} {}  rb\streep{2.5em}\\
        Dadelijk doen de draken Doris~ duizelen.\\
        immediately make.\PRS.\PL{} the dragons Doris feel.dizzy.\INF{}\\}
  \end{xlist}
\end{exe}
%
The verb-second constraint is clear here: the finite verb is always in
the left bracket and precedes its subject if a non-subject is in the
Vorfeld~(\ref{ex:WGerm:2}b,c). The subject and object can appear in identical
positions -- contrast the OVS order in~(\ref{ex:WGerm:2}b) with SVO
in~(\ref{ex:WGerm:2}a). Linear order is therefore not fully determined by
grammatical function, or vice versa. The Vorfeld is also the target of
long-distance dependencies, like fronting of wh-constituents out of
embedded clauses (not shown here). The Mittelfeld may contain a
collection of (nominal) arguments and (simple) adverbials, which are
typically local to the clause. The extent to which the order of material
within the Mittelfeld is fixed differs between languages
(see~\sectref{sec:Germanic:scrambling}). Grammatical-function assignment under
word order variation, long-distance dependencies, and the order of
elements in the Mittelfeld are all basic CWG phenomena that the LFG
models discussed below must address.

Other clause types have empty Vorfeld regions, such as the polar
interrogative~(\ref{ex:WGerm:3}a), which is a verb-first construction, and the subordinate
clause~(\ref{ex:WGerm:3}b), in which the left bracket is filled by the complementizer.

\begin{exe}
  \ex\label{ex:WGerm:3} Dutch 
  \begin{xlist}
    \ex  Polar interrogative:\\
         {\glll%
           lb\streep{1.25em}      Mf\streep{9.5em} {} {} {}  rb\streep{2.5em}\\    
           Doen de draken {Doris} dadelijk duizelen?\\
           make.\PRS.\PL{} the dragons Doris immediately feel.dizzy.\INF{}\\}
    \ex Subordinate:\\
        {\glll%
          \phantom{\ldots}lb\streep{.375em}  Mf\streep{9.5em} {} {} {}  rb\streep{7.5em}\\
          \ldots{}dat de draken Doris dadelijk doen duizelen.\\
         \phantom{\ldots}\COMP{} the dragons Doris immediately make.\PRS.\PL{} feel.dizzy.\INF{}\\}
  \end{xlist}
\end{exe}
%
Example~(\ref{ex:WGerm:3}b) has the finite verb in the right bracket, in the verb
cluster. This shows the asymmetry of the verb-second phenomenon:
unlike in a main clause, the finite verb in a subordinate clause can
be preceded by any number of constituents in the Mittelfeld. The
topological model accommodates the complementary distribution of the
finite verb of a main clause and the complementizer of a subordinate
clause by locating both in the left bracket.

The right bracket in~(\ref{ex:WGerm:3}b) contains two verbs: first the finite verb,
then the non-finite verb. This is considered to be the default
order in Standard Dutch, but there is considerable variation in this
ordering, both between and within CWG languages. An extensive overview
of ordering possibilities in CWG verb clusters is given
in~\citet{wurmbrand:2004}.

The topological schema based on the combination of verb second and
verb final is widely applicable to the CWG languages, but, as with any
generalization, there are cases where it does not apply. To start, we
must keep separate the notion of main clause vs.\ subordinate clause
\textit{word order} from the notion of unembedded and embedded clause
\textit{uses}. This is because German, amongst others, allows embedded
clauses to have verb second under bridge verbs in the absence of a
complementizer (see~\sectref{sec:Germanic:top}); that is, it allows main clause
word order for certain embedded clauses. Furthermore, the separation
of non-verbal material in the Mittelfeld and verbal material in
the right bracket is not always as clean as the topological model
suggests, as languages may allow for material from the two fields to
be mixed, blurring the border between them (see
\sectref{sec:Germanic:mfrb}). Finally, Afrikaans and Yiddish have clause
structures that deviate further. Spoken Afrikaans optionally allows
the combination of a complementizer and verb second in subordinate
clauses \citep{biberauer:2009}. Modern Yiddish has verb second in main
as well as in subordinate clauses, and in addition its status as a
verb-final language is debated \citep{diesing:1997}. Historical stages
of Yiddish did however follow CWG's characteristic pattern more
closely \citep{santorini:1992}.

\subsubsection{Clause union}

The examples in~(\ref{ex:WGerm:2}) and~(\ref{ex:WGerm:3}) each contain two verbs. We
have discussed the topological model as a schema of the clause,
without questioning whether we are dealing with mono-clausal
structures here. Since \citet{bech:1955}, it is common to distinguish
between \textit{coherent} and \textit{incoherent} verb
combinations. The former describes a combination of two verbs into a
single clause, \textit{clause union}, whereas the latter results in a
biclausal structure. The contrast is illustrated below. In~(\ref{ex:WGerm:4}a),
the coherently combining \textit{durfde} `dared' shares the verb
cluster with its embedded verb \textit{te kopen} `to buy', and the
embedded object sits in the Mittelfeld of the clause headed by
\textit{durfde}.  Example~(\ref{ex:WGerm:4}b) contains the incoherently
combining \textit{beloofd} `promised', and here we see that both the
embedded verb and its object appear after the matrix verb, in the
Nachfeld of the matrix clause. As shown in~(\ref{ex:WGerm:4}c), this word order
is not available for the coherently combining \textit{durfde} `dared'.
%
\begin{exe}
  \ex\label{ex:WGerm:4} Dutch
  \begin{xlist}
    \ex[]{\glll
      \phantom{\ldots}lb\streep{1.625em}  Mf\streep{3.5em} {} {} rb\streep{5.5em}\\
      \ldots{}omdat hij geen auto durfde {te kopen}.\\
      \phantom{\ldots}because he no car dared buy.\textsc{teinf}\footnotemark\\
      \glt `\ldots because he didn't dare to buy a car.'}
    \footnotetext{The abbreviations \textsc{teinf} and \textsc{zuinf}
      in the glosses are used for the verb forms in Dutch and German
      that combine the infinitive marker (\textit{te} in Dutch,
      \textit{zu} in German) with an infinitive. Unlike corresponding
      forms in for instance English, these combinations are generally
      not separable.}
    \ex[]{\glll
      \phantom{\ldots}lb\streep{1.625em} Mf rb\streep{2.5em} Nf\streep{6.75em}\\
      \ldots{}omdat hij beloofde geen auto {te kopen}.\\
      \phantom{\ldots}because he promised no car buy.\textsc{teinf}\\
      \glt `\ldots because he promised not to buy a car.'}    
    \ex[*]{\gll \ldots{}omdat hij durfde geen auto {te kopen}.\\
      \phantom{\ldots}because he dared no car buy\\}
  \end{xlist}
\end{exe}
%
The second sign that we are dealing with one clause in~(\ref{ex:WGerm:4}a) and two
in~(\ref{ex:WGerm:4}b) is the scope of the negation, as evident from the
translations. In both examples, negation is marked on the embedded object through the
negative determiner but it nevertheless scopes over the finite verb
in~(\ref{ex:WGerm:4}a). The same negation marking in the biclausal~(\ref{ex:WGerm:4}b)
yields a narrow scope negation.

A third phenomenon associated with clause union is the potential to trigger
\textit{infinitivus pro participio} (IPP; German:
\textit{Ersatzinfinitiv} `replacement infinitive'). IPP refers to the
realization of a verb in the infinitive when a participle is expected
on the basis of the selecting auxiliary. For this to occur, the clause itself
must also contain a further, lower verb in the
infinitive.\footnote{Further conditions may apply, for instance on the
order of the auxiliary and the middle verb.} The occurrence of IPP is
therefore evidence of the middle and lower verb combining
coherently. Below, IPP is triggered in~(\ref{ex:dutchipp}a), affecting the
coherently combining \textit{durven} `dare', but not in~(\ref{ex:dutchipp}b) for
the incoherently combining \textit{beloofd} `promised'.
%
\begin{exe}
  \ex \label{ex:dutchipp}Dutch\footnote{A note on the use of brackets and parentheses in examples in this chapter: I will use curly brackets to indicate choice. The choice is either between several forms in one position, such as in the current example~(\ref{ex:dutchipp}), or between several positions for one form, such as in the example in~(\ref{ex:particleclimbing}). Square brackets delimit constituents when this is relevant, such as in~(\ref{ex:germannested}). Parentheses indicate optionality as usual.}
  \begin{xlist}
    \ex[]{\gll Hij heeft geen auto $\{$durven / *gedurfd$\}$ {te kopen}.\\
      he has no   car  \phantom{$\{$}dare.\INF{} {} \phantom{*}dare.\PTCP{}  buy.\textsc{teinf}\\
      \glt `He didn't dare to buy a car.'}
    \ex[]{\gll Hij had $\{$beloofd / *beloven$\}$ geen auto {te kopen}.\\
      he had \phantom{$\{$}promise.\PTCP{} {} \phantom{*}promise.\INF{} no car buy.\textsc{teinf}\\
    \glt `He promised not to buy a car.'}
  \end{xlist}
\end{exe}
%
Example~(\ref{ex:dutchipp}a) additionally shows that a clause can contain more
than two verbs. In principle, there is no limit to the number of verbs 
involved in clause union, since the same couple of coherently
combining verbs can appear at multiple levels of
embedding.\footnote{In practice, it seems that three-verb combinations are common, but more complex clauses are rare. For instance, \citet{cousse-bouma:2022} report numbers for a mixed corpus of written and spoken Dutch: about 3\% of coherence domains contain three verbs, but only 0.1\% contain four verbs.}

A wide range of verbs allow for coherent combination. For instance, for Dutch, 
the reference grammar \citet{haeseryn-ed:1997} lists over 100
verbs that always combine coherently, and an additional 20 that do
so optionally. In this list we find auxiliaries; evidential, modal and aspectual
verbs; but also verbs with a clearer lexical contribution such as
causal and perceptual verbs, and for instance verbs corresponding to
\textit{help}, \textit{learn}, \textit{try} or \textit{forget}. In
theoretical syntactic work, combining behaviour is commonly taken to
be an underived, lexical property of the embedding 
verb, but see \citet{cook:2001:revdiss} for an
explanation of coherence in German in
terms of information structure.

\subsubsection{Crossing dependencies\label{sec:Germanic:xingdeps}}

When we have coherently combining verbs that also introduce their own
object, we can end up with a clause in which a sequence of objects in
the Mittelfeld is followed by a corresponding sequence of verbs in the
cluster. In languages like German or West Frisian, the
unmarked order of the objects is by increasing order of embedding
O$_1$O$_2$\ldots{}O$_n$, whereas the order of verbs is by decreasing
order of embedding V$_n$\ldots{}V$_2$V$_1$. This gives rise to a
pattern of nested dependencies between the objects and their
verbs~(\ref{ex:germannested}).
%
\begin{exe}
  \ex \label{ex:germannested}Standard German\\
      {\glll
        {} {} \phantom{$[$}O$_1$ {} \phantom{$[$}O$_2$ {} V$_2$ V$_1$\\
        \ldots{}dass wir $[$dem Hans$]$ $[$das Haus$]$ streichen halfen\\
        \phantom{\ldots}\textsc{comp} we \phantom{$[$}the.\textsc{dat} Hans \phantom{$[$}the.\textsc{acc} house paint helped\\
          }
    \glt `\ldots that we helped (V$_1$) Hans (O$_1$) paint (V$_2$) the house (O$_2$).'

\end{exe}
%
In Dutch and Swiss German, however, objects and verbs can \textit{both} be
ordered by increasing level of embedding O$_1$O$_2$\ldots{}O$_n$ and
V$_1$V$_2$\ldots{}V$_n$. This creates cross-serial
dependencies between objects and verbs, as in~(\ref{ex:germancross}).
%
\begin{exe}
\ex \label{ex:germancross}Swiss German \citep[§2, example~1]{Shieber1985}\\
    {\glll
       {} {} \phantom{$[$}O$_1$ {} \phantom{$[$}O$_2$ {} V$_1$ V$_2$\\
      \ldots{}das mer $[$em Hans$]$ $[$es huus$]$ hälfed aastriiche.\\
            \phantom{\ldots}\textsc{comp} we \phantom{$[$}the.\textsc{dat} Hans \phantom{$[$}the.\textsc{acc} house helped paint\\}
\end{exe}
%
The phenomenon of cross-serial dependencies has received ample interest in the literature, because
it requires more than context-free power to model
\citep{BKPZ:Dutch,pullum-gazdar1982,Shieber1985}.

\subsubsection{In and around the nominal domain}

We end the overview of CWG syntax by briefly discussing the main
characteristics of the nominal domain and, even more briefly,
adpositions. This is to give a general sense of what these domains
look like in CWG languages. Most of what is discussed below resembles
what we find in the North-Germanic languages and English.

The nominal domain in CWG generally follows a
determiner--adjective--noun pattern, with further adnominal material
(relative clauses, PPs, etc.) realized postnominally. This is exemplified
in~(\ref{ex:WGerm:8}).
%
\begin{exe}
  \ex  Dutch\label{ex:WGerm:8}\\{\gll de mooiste plek van Europa\\
    the beautiful.\textsc{superlative} place of Europe\\
    \glt `the most beautiful place in Europe'}
\end{exe}
%
Present-day CWG languages have at most four cases (\NOM, \GEN, \DAT
and \ACC; for example Standard German), but many make fewer case distinctions (see \citealp{kasper:2014} for a compact description of the situation in German varieties and references), and  several have only a subject-object form distinction
remaining in the pronominal paradigm (for example, Afrikaans, Dutch, and West Frisian).\footnote{See \url{taalportaal.org} for linguistic descriptions of Afrikaans, Dutch, and West Frisian. Consulted September 2022.} There are two numbers (\SG, \PL). Any of three genders
(\M, \F and \N; Alemannic\footnote{The term
Alemannic (German) covers amongst others Alsace German, Swabian and Swiss German.}, Low Saxon,\footnote{Low
Saxon is used interchangeably with Low German by some authors. Our use
of the term here comprises regional languages of the north of Germany
and the east/north-east of the Netherlands. Our choice for the term
Low Saxon is partly driven by the fact that LFG work on this language
uses this term: see~\sectref{sec:Germanic:posslowsax}.}, Standard German, West Flemish), two
genders (\textsc{common} gender and \N; Dutch, West Frisian) or no distinction
(Afrikaans) may occur. Gender agreement distinctions only show up in the
singular. The different paradigm sizes with respect to gender are
illustrated in~(\ref{ex:WGerm:9}). Note the form contrasts in the definite
determiners.
%
\begin{exe}
  \ex \label{ex:WGerm:9}
  \begin{xlist}
    \ex Alemannic\\
        {\gll d Frau \hspace{2.25em} dr Maa \hspace{2.25em} s Chind\\
              the.\F.\NOM{} woman {} the.\M.\NOM{} man {} the.\N.\NOM{} child\\}
    \ex West Frisian\\
    {\gll de frou \hspace{1.225em} de man \hspace{1.4em} it bern\\
         the.\textsc{common} woman {} the.\textsc{common} man {} the.\N{} child\\} 
    \ex Afrikaans\\
        {\gll die vrou  \hspace{4.96875em} die man  \hspace{5.1875em} die kind\\
          the woman {} the man {} the child\\}
  \end{xlist}
\end{exe}

Adjectives can be associated with multiple inflectional paradigms --
see \sectref{sec:Germanic:dquant_aquant} for a discussion of these
\textit{declension classes} in Standard German.  Even in languages
with more elaborate paradigms, there is typically a great level of
syncretism between forms across inflectional dimensions for
determiners, pronouns, adjectives and nouns. The consequences of
syncretism for grammatical modelling are discussed
in~\sectref{sec:Germanic:indeterminacy}.

Adpositions are overwhelmingly prepositional~(\ref{ex:WGerm:10}a), but the sporadic
postposition~(\ref{ex:WGerm:10}b) and circumposition~(\ref{ex:WGerm:10}c) occur as well.
%
\begin{exe}
  \ex\label{ex:WGerm:10} Gronings (Low Saxon)
  \begin{xlist}
    \ex {%
      \gll \textbf{op} de grìns\\
      on the border\\
      \glt `on the border'}
    \ex {%
      \gll t haile joar \textbf{deur}\\
      the whole year through\\
      \glt `the whole year through'}
    \ex {\gll \textbf{om} de provìnzie \textbf{tou}\\
              around the province around\\
         \glt `around the province'}
  \end{xlist}
\end{exe}


\subsection{Overview of the rest of the chapter}

Thus far, I have talked about the geographic distribution and the
syntactic characteristics of the Contintental West-Germanic languages,
to define the scope of the chapter, and to give a background for what
is to come below.

The remainder of this chapter is devoted to LFG analyses of different
aspects of CWG syntax. In~\sectref{sec:Germanic:clausverb}, I discuss LFG
accounts of the clause and the verbal domain, and
in~\sectref{sec:Germanic:nom}, I discuss LFG studies of the nominal
domain. These sections are structured in a parallel fashion: they
start with analyses of the overall structure of their respective
domains, and then continue with a discussion of more specific LFG
accounts organized by topic. In the LFG literature on CWG, the clausal
and verbal domains have received by far the most attention, which
means that the corresponding section dominates this chapter in terms
of size.

The chapter ends with concluding remarks in~\sectref{sec:Germanic:conclude},
in which I briefly touch upon some LFG and LFG-related work that was not
included in detail here, and give pointers for further reading.

\section{LFG analyses in the clausal and verbal domains\label{sec:Germanic:clausverb}}

This section deals with phenomena at the level of the clause. I will
start in~\sectref{sec:Germanic:overall} with a discussion of the variety of
ways in which the overall shape of the clause has been modelled, mostly in
terms of c-structure. I then look at specific topics that have been
prominent in the LFG literature on CWG languages. The topics are
divided into thematic sections. Phenomena at the left and right
periphery are discussed in~\sectref{sec:Germanic:leftperi}
and~\sectref{sec:Germanic:rightperi}, respectively. Studies dealing with the
ordering of dependents are discussed
in~\sectref{sec:Germanic:ordering}. Finally, mapping-based analyses of areas of
CWG clause syntax are presented in~\sectref{sec:Germanic:mapping}.


\subsection{The overall shape of the clause\label{sec:Germanic:overall}}

The discussion of the different LFG conceptions of the overall
shape of the clause is organized according to the topological field
model. I first consider the top level of the clause
(directly containing the Vorfeld and left bracket)
in~\sectref{sec:Germanic:vflb}, and then the lower level of the clause (the Mittelfeld and right bracket) in~\sectref{sec:Germanic:mfrb}. The Nachfeld
is discussed in~\sectref{sec:Germanic:nf}.

\subsubsection{Vorfeld and left bracket\label{sec:Germanic:vflb}}

\citet{berman-frank:1996}, \citet{Choi1999},
\citet{Berman2003}, and \citet{frank:2006} model the
German verb-second clause as a CP. The finite verb sits in C irrespective of
whether the initial position is occupied by the subject of a
declarative clause~(\ref{ex:WGerm:11}a) or by some other element, like the object in~(\ref{ex:WGerm:11}b). The
complementary distribution between the finite verb and the
complementizer in the left bracket follows as well: the complementizer can
only appear in C, and when it is realized, the finite verb must occur
in another, lower position~(\ref{ex:WGerm:11}c).
%
\newcommand{\gllx}{}
%
\begin{exe}
  \ex\label{ex:cpshape}\label{ex:WGerm:11} German
  \begin{xlist}
    \ex\label{ex:cpshape:subjinit}%
    \begin{forest}
      [CP
        [\rulenode{(\UP\SUBJ)=\DOWN\\DP}
          [{\gll
              Annie\\
              Annie\\}]
        ]
        [\rulenode{\UP=\DOWN\\\BAR{C}}
          [\rulenode{\UP=\DOWN\\C}
            [{\gll
                hat\\
                has\\}]]
          [\rulenode{\UP=\DOWN\\\ldots}
            [{\gll
                die Kinder gelobt.\\
                the children praised\\},roof]
          ]
        ]
      ]
    \end{forest}
    %
    \ex\label{ex:cpshape:objinit}%
    \begin{forest}
      [CP
        [\rulenode{(\UP\DF)=\DOWN\\DP}
          [{\gll
              Die Kinder\\
              the children\\}, roof]
        ]
        [\rulenode{\UP=\DOWN\\\BAR{C}}
          [\rulenode{\UP=\DOWN\\C}
            [{\gll
                hat\\
                has\\}]
          ]
          [\rulenode{\UP=\DOWN\\\ldots}
            [{\gll
                Annie gelobt.\\
                Annie praised\\} ,roof]
          ]
        ]
      ]   
    \end{forest}
    %
    \ex\label{ex:cpshape:subord}%
    \begin{forest}
      [CP
        [\rulenode{\UP=\DOWN\\C}
          [{\gll
              dass\\
              \COMP{}\\}]
        ]
        [\rulenode{\UP=\DOWN\\\ldots}
          [{\gll
              Annie die Kinder gelobt hat.\\
              Annie the children praised has\\},roof]
        ]
      ]   
  \end{forest}
  \end{xlist}
\end{exe}
%
The nature of Comp-CP, the node
dominating the combined Mittelfeld and right bracket, differs between
these authors, however, and will be discussed
below. \Citet{vanderbeek:2005} and \citet{jones:2020:lfg}, on Dutch,
consider only main clauses, which they posit to be IPs.

\citeauthor{zaenen-kaplan1995}
\citetext{\citeyear{zaenen-kaplan1995}, on Dutch;
  \citeyear{ZaenenKaplan2002:Subsumption} on German} prefer a slightly
flatter structure, exemplified in~(\ref{ex:sbarshape:subjinit}). The
label `S|VP', a convention from the cited papers, is used to show that
the authors do not wish to choose between these categories. 
%
\begin{exe}
  \ex\label{ex:sbarshape:subjinit} Dutch\\
  \begin{forest}
    [\BAR{S}
      [\rulenode{(\UP\SUBJ)=\DOWN\\DP}
        [{\gll
            Annie\\
            Annie\\}]
      ]
      [\rulenode{\UP=\DOWN\\V}
        [{\gll
            heeft\\
            has\\}]
      ]
      [\rulenode{\UP=\DOWN\\S|VP}
        [{\gll
            de kinderen geprezen.\\
            the children praised\\} ,roof]
      ]
    ]
  \end{forest}
\end{exe}
%
\citeauthor{zaenen-kaplan1995}'s subordinate clauses are isomorphic to those
in~(\ref{ex:cpshape:subord}), but are labelled \BAR{S} instead of CP.

An even flatter structure appears in the computational grammar fragment
discussed in \citet{clement-etal:2002:lfg}, who model the topological
field schema directly in LFG. All topological fields
are c-structure nodes and direct descendants of the MD node (`main
domain') that represents the whole
sentence. Example~(\ref{ex:topfield}) gives a somewhat simplified
tree, using the abbreviations for the topological fields which I introduced
in~\sectref{sec:Germanic:clauselayout}.
%
\begin{exe}
  \ex\label{ex:topfield}
  \begin{forest}
           [MD
             [\rulenode{\UP=\DOWN\\Vf}
               [\rulenode{(\UP\SUBJ)=\DOWN\\DP}
                 [{\gll
                     Annie\\
                     Annie\\}]
               ]
             ]
             [\rulenode{\UP=\DOWN\\lb}
               [\rulenode{\UP=\DOWN}\\V
                 [{\gll
                     hat\\
                     has\\}]
               ]
             ]
             [\rulenode{\UP=\DOWN\\Mf}
               [\rulenode{(\UP\XCOMP*\,\OBJ)=\DOWN\\DP}
                 [{\gll
                     die Kinder\\
                     the children\\}, roof]
               ]
             ]
             [\rulenode{\UP=\DOWN\\rb}
               [\rulenode{(\UP\XCOMP)=\DOWN\\VC}
                 [{\gll
                     gelobt\\
                     praised\\}, roof]
               ]
             ]
           ]
  \end{forest}
\end{exe}
%
A very similar flat structure can be found in
\citet{rohrer:1996}.\footnote{\citet{rohrer:1996}, however, also writes
``Diese flache Struktur lä{\ss}t sich problemlos in eine binäre
rechtsverzweigende Struktur umwandeln. [\ldots] Wir behalten das
flache Mittelfeld hier primär aus expositorischen Gründen bei'' (p96,
fn 3). [This flat structure can be converted to a binary
  right-branching structure without problems. We maintain the
  flat Mittelfeld here primarily for reasons of exposition.]}


\subsubsection{Mittelfeld and right bracket\label{sec:Germanic:mfrb}}

The Mittelfeld and right bracket form the lower c-structure level in
the clause. This is the unlabelled Comp-CP in~(\ref{ex:cpshape}) and
the S|VP node in~(\ref{ex:sbarshape:subjinit}). All authors agree that
this part of the tree does \textit{not} involve an IP.\footnote{In
fact, in LFG, the assumption of an IP anywhere in CWG c-structure is
rare. We mentioned \Citet{vanderbeek:2005} and \citet{jones:2020:lfg},
on Dutch, who use it as the category at the top level, for the whole
V2 declarative clause. The choice is not further motivated in these
works, and moreover it is peripheral to the respective
discussions. \citet{BresnanEtAl2016} posit that Comp-CP contains an IP
in one of the book's exercises on German. However, since this is a textbook, it is
unclear whether the authors are theoretically committed to this
choice, or whether it was made for other reasons, for instance
pedagogical ones.} This choice against an intermediate IP can also be
found in analyses of German in the Chomskyan tradition, for instance
in the line of work summarized in \citet[see §2.2 therein for an
  overview of the arguments]{haider:2010}.

A salient question in the analysis of this part of the clause is the
order of the verbs and the arguments, and the concomitant contrast
between nested versus cross-serial dependencies. We will focus first
on the two polar opposites: verbs ordered after increasing level of
embedding (cross-serial dependencies) and verbs ordered after
decreasing level of embedding (nested dependencies). The following pair, a variation on~(\ref{ex:germannested}--\ref{ex:germancross}) above, illustrates the difference with three verbs in the verb cluster: 
%
\begin{exe}
  \ex
  \begin{xlist}
  \ex Standard German\\
      {\gll
        \ldots{}dass wir $[$die Kinder$]$ $[$dem Hans$]$ $[$das Haus$]$ streichen helfen lassen\\
        \phantom{\ldots}\textsc{comp} we \phantom{$[$}the.\ACC{} children \phantom{$[$}the.\DAT{} Hans \phantom{$[$}the.\ACC{} house paint help let\\
          }
    \glt `\ldots that we let the children help Hans paint the house.'
\ex Swiss German \citep[§2, example~5]{Shieber1985}\\
    {\gll
      \ldots{}das mer $[$d' Chind$]$ $[$em Hans$]$ $[$es huus$]$ lönd hälfe aastriiche.\\
        \phantom{\ldots}\textsc{comp} we \phantom{$[$}the.\ACC{} children \phantom{$[$}the.\DAT{} Hans \phantom{$[$}the.\ACC{} house let help paint.\\}
  \end{xlist}
\end{exe}
%
As mentioned in the introduction, there is considerable variation in
the order of the verbal elements beyond these two opposites, and there
is even variation in the extent to which the nominal material in the
Mittelfeld and verbal material in the right bracket is kept separated,
both between and within CWG languages. This variation will be
briefly discussed at the end of this subsection.

\subsubsubsection{Cross-serial dependencies} An early LFG analysis of Dutch
cross-serial dependencies is found in \citet{BKPZ:Dutch}, which was a
prominent demonstration of how LFG's formalism has the power needed
for linguistically valid analyses of such dependencies.\footnote{The
paper played a central role in the discussion of the context-freeness
of natural language syntax. See e.g. \citet{pullum-gazdar1982} and
\citet{Shieber1985} for more discussion of the issues involved and the
kind of evidence considered.} Starting from a proposal by
\citet{evers:1975}, schematically in~(\ref{ex:xs:evers}), with a flat
Mittelfeld and a right-branching verb-cluster, \citeauthor{BKPZ:Dutch}
argue that a structured Mittelfeld is to be preferred, as
in~(\ref{ex:xs:bresnan}).

\begin{exe}
  \ex
  \begin{xlist}
  \ex\label{ex:xs:evers}
    \begin{forest}
      [S %, s sep=2pt
        [DP\\(S) [Mf\streep{10em}, no edge, l=11em, tier=fields]]
        [DP\\(O$_1$)]
        [DP\\(O$_2$)]
        [DP\\(O$_3$)]
        [{\vphantom{DP}}, edge={dotted, line width=.75pt}]
        [\BAR{V}
          [V$_1$ [rb\streep{4em}, no edge, tier=fields]]
          [\BAR{V} [V$_2$]
            [\BAR{V} [V$_3$]
              [\vphantom{V},edge={dotted, line width=.75pt}]
            ]
          ]
        ]
      ]
    \end{forest}
  \ex\label{ex:xs:bresnan}
    \begin{forest}
      [S %, s sep=3.5pt
        [DP\\(S) [Mf\streep{11em}, no edge, l=14em, tier=fields]]
        [VP [DP\\(O$_1$)]
          [VP [DP\\(O$_2$)]
            [VP [DP\\(O$_3$)]
              [\vphantom{DP},edge={dotted, line width=.75pt}]
            ]
          ]
          [\BAR{V}
            [V$_1$ [rb\streep{4em}, no edge, tier=fields]]
            [\BAR{V}
              [V$_2$]
              [\BAR{V} [V$_3$]
                [\vphantom{V},edge={dotted, line width=.75pt}]
              ]
            ]
          ]
        ]
      ]
    \end{forest}
  \end{xlist}
  \end{exe}
% 
The tree in~(\ref{ex:xs:bresnan}) contains two parallel embedding structures: one
for the objects in the Mittelfeld and one for the verbs
in the right bracket. This is captured in the c-structure
definitions in~(\ref{ex:bresnanetal1982:cstr}).

\pagebreak 
\begin{exe}
\ex\label{ex:bresnanetal1982:cstr}\begin{xlist}
  \ex \phraserule{S}{\rulenode{DP\\(\UP\SUBJ)=\DOWN} \rulenode{VP\\\UP=\DOWN}}
  \ex \phraserule{VP}{\optrulenode{DP\\(\UP\OBJ)=\DOWN} \optrulenode{VP\\(\UP\XCOMP)=\DOWN} \optrulenode{\BAR{V}\\\UP=\DOWN}}
  \ex \phraserule{\BAR{V}}{\rulenode{V\\\UP=\DOWN} \optrulenode{\BAR{V}\\(\UP\XCOMP)=\DOWN}}
  \end{xlist}
\end{exe}
%
For the objects, each level of VP embedding adds a level of \XCOMP
embedding at f-structure. For the verbs, each level of \BAR{V}
embedding does the same. A compatible stacking of \XCOMP{}s is thus
built up in both parts of the tree. The optionality of the object DP
in~(\ref{ex:bresnanetal1982:cstr}b) allows for verbs that do not introduce their own
object. It is essential that an \XCOMP{} level is introduced for these
in both parts of the tree, too, to maintain the parallel structure.

The accusative with infinitive verbs involved in the cross-serial
construction are analyzed as raising-to-object verbs. The inflected
verb form \textit{zag} `saw', for instance, receives a lexical entry
along the lines of~(\ref{ex:bresnanetal1982:lex}).
%
\begin{exe}
  \ex\label{ex:bresnanetal1982:lex} \catlexentry{zag}{V}{%
    (\UP\PRED)=\mbox{`\rm see\arglist{\SUBJ,\XCOMP}\OBJ'}\\
    (\UP\XCOMP\,\SUBJ)=(\UP\OBJ)\\
    (\UP\XCOMP\,\textsc{form})=\textsc{inf}\\
    (\UP\SUBJ\,\NUM)=\SG\\
    (\UP\TENSE)=\PST}\\
\end{exe}
%
Complemented with rules for DPs and additional lexical entries, this
grammar fragment gives us analyses like the one
in~(\ref{ex:zagtweeberen}). 
%

\begin{exe}
  \ex\label{ex:zagtweeberen}
  \begin{xlist}
    \ex Dutch\\
        {\glll
                 {}  S \phantom{$[$}O$_1$ {} O$_2$ V$_1$ V$_2$\\
          \ldots{}dat ik $[$twee beren$]$ broodjes zag smeren.\\
          \phantom{\ldots}\COMP{} I \phantom{$[$}two bears sandwiches saw spread\\
            \glt `\ldots that I saw (V$_1$) two bears (O$_1$) prepare (V$_2$) sandwiches (O$_2$).'
        }
    \ex\hspace{9.25em}\attop{%
    %  If this ^^^^^^ is changed...      
      \avm[style=fstr,pic,picname=smeren1f]{
        \rnode{w}{
          [ pred  & `{\upshape see}\arglist{\SUBJ,\XCOMP}\OBJ' \\
            tense &  pst \\
            subj  & \rnode{wsu}{[pred & `{\upshape pro}'\\
                             num & sg]}\vspace{\smallskipamount} \\
          obj   & \rnode{wob}{[pred & `{\upshape bear}' \\
                            spec & [pred & `{\upshape two}'] \\
                            num & pl]}\vspace{\smallskipamount}            \\
          xcomp & \rnode{wxc}{[pred & `{\upshape spread}\arglist{\SUBJ,\OBJ}' \\
                   form & inf \\
                   subj & \rnode{wxcwsu}{\strut}\\
                   obj & \rnode{wxcwob}{[pred & `{\upshape sandwich}'\\
                          num & pl]}]}~~\\
          ]}
      }
% 
\CURVE[1.8]{-2pt}{0}{wob}{0pt}{0}{wxcwsu}
    }\\
    \hspace*{-3em}\begin{forest}
    %...this ^^^^ should change accordingly 
      [\BAR{S}
         [C
           [{\gll
               dat\\
               \COMP{}\\}]
         ]
         [\rnode{s}{S}
           [{\rnode{dp}{DP}}
             [{\gll
                 ik\\
                 I\\}, roof]]
           [\rnode{vp1}{VP}
             [\rnode{dp1}{DP} [{\gll
                   twee beren\\
                   two bears\\}, roof]]
             [\rnode{vp2}{VP}
               [\rnode{dp2}{DP} [{\gll
                     broodjes\\
                     sandwiches\\}, roof]]]
             [\rnode{vbar1}{\BAR{V}}
               [\rnode{v1}{V} [{\gll
                     zag\\
                     saw\\}]]
               [\rnode{vbar2}{\BAR{V}}
                 [\rnode{v2}{V}
                   [{\gll
                       smeren\\
                       spread\\}]]
               ]
             ]
           ]
         ]
       ]
     \end{forest}
     % top level verbal spine
     \nccurve[linewidth=.5pt,
       ncurvA=.5, nodesepA=1pt, angleA=0,
       ncurvB=2.5, nodesepB=0pt, angleB=180
     ]{->}{s}{w}
     \nccurve[linewidth=.5pt,
       ncurvA=.31250, nodesepA=1pt, angleA=0,
       ncurvB=2, nodesepB=0pt, angleB=180
     ]{->}{vp1}{w}
     \nccurve[linewidth=.5pt,
       ncurvA=.5, nodesepA=1pt, angleA=0,
       ncurvB=1.40625, nodesepB=0pt, angleB=180
     ]{->}{vbar1}{w}
     \nccurve[linewidth=.5pt,
       ncurvA=1, nodesepA=1pt, angleA=0,
       ncurvB=1.53125, nodesepB=0pt, angleB=180
     ]{->}{v1}{w}
     % top level arguments
     \nccurve[linewidth=.5pt,
       ncurvA=.25, nodesepA=1pt, angleA=180,
       ncurvB=1.375, nodesepB=0pt, angleB=180, offsetB=-1.5ex
     ]{->}{dp}{wsu}     
     \nccurve[linewidth=.5pt,
       ncurvA=.25, nodesepA=1pt, angleA=180,
       ncurvB=1, nodesepB=0pt, angleB=180, offsetB=-2ex
     ]{->}{dp1}{wob}
     % subclause verbal spine
     \nccurve[linewidth=.5pt,
       ncurvA=.625, nodesepA=1pt, angleA=0,
       ncurvB=.625, nodesepB=0pt, angleB=0, offsetB=2ex
     ]{->}{vp2}{wxc}
     \nccurve[linewidth=.5pt,
       ncurvA=.25, nodesepA=1pt, angleA=0,
       ncurvB=.75, nodesepB=0pt, angleB=0, offsetB=2ex
     ]{->}{vbar2}{wxc}
     \nccurve[linewidth=.5pt,
       ncurvA=.5, nodesepA=1pt, angleA=0,
       ncurvB=.75, nodesepB=0pt, angleB=0, offsetB=2ex
     ]{->}{v2}{wxc}
     % subclause argument
     \nccurve[linewidth=.5pt,
       ncurvA=.25, nodesepA=1pt, angleA=180,
       ncurvB=.87500, nodesepB=0pt, angleB=180, offsetB=-1.5ex
     ]{->}{dp2}{wxcwob}

  \end{xlist}
\end{exe}


However successful in capturing cross-serial dependencies, this
analysis runs into descriptive problems if taken more generally as a
model of the Dutch sentence. \citet{zaenen-kaplan1995} give the
example in~(\ref{ex:Germ:19}), which involves the coordination of two \BAR{V}s that
each require a different level of \textsc{xcomp} embedding for the
object supplied in the Mittelfeld.

\pagebreak
\begin{exe}
  \ex\label{ex:Germ:19} Dutch \citep[§2.3, example~9]{zaenen-kaplan1995}\\
  \gll \ldots{}dat Jan een liedje {$[ [_{\BAR{\scriptsize V}}$ schreef$\,]$} en {$[_{\BAR{\scriptsize V}}$ trachtte} {$[_{\BAR{\scriptsize V}}$ te verkopen$]]]$.}\\
       \phantom{\ldots}\COMP{} Jan a song \phantom{$[ [_{\BAR{\scriptsize V}}$ }wrote and \phantom{$[_{\BAR{\scriptsize V}}$ }tried \phantom{$[_{\BAR{\scriptsize V}}$}sell\textsc{.teinf}\\
      
%      \gll dat Jan een liedje $[_\mathrm{\BAR{V}}$ schreef en trachtte te verkopen$]$.\\
%          Jan a song write and tried sell\\
      \glt `\ldots that Jan wrote and tried to sell a song.'
\end{exe}
%
Since different levels of \textsc{xcomp} embedding of the object
correspond to different c-structures in the model of
\citet{BKPZ:Dutch}, example~(\ref{ex:Germ:19}) cannot receive an analysis if we use the standard treatment of constituent coordination in LFG. It would
require the shared material to receive two different c-structures at
the same
time. \citeauthor{zaenen-kaplan1995}'s~(\citeyear{zaenen-kaplan1995})
alternative relies on functional uncertainty to connect the objects to
predicates at the required level of \textsc{xcomp} embedding, and on
functional precedence rules to make sure that the linear order of
objects reflects their level of embedding. They replace the
VP and \BAR{V} rules of~(\ref{ex:bresnanetal1982:cstr}) with those
in~(\ref{ex:zk95:vprules}).
%% \footnote{If we call the \OBJ{} of an \XCOMP{} to an
%% embedded object, then the functional uncertainty expression
%% \mbox{\XCOMP{}* \OBJ} describes the set of paths to unembedded objects
%% or embedded objects at any level of embedding, and \mbox{\XCOMP{}$^+$
%%   \OBJ} describes embedded objects at any level of embedding (so not
%% unembedded objects). The operator $<_f$ in~(\ref{ex:zk95:vprules}b)
%% marks the f-precedence relation, which holds between two f-structures
%% if the c-structure material projecting to the first f-structure
%% precedes the c-structure material projecting to the second. (See the
%% chapter on ``Core concepts of LFG'' in Part I of this volume for a
%% detailed discussion of both functional uncertainty and f-precendence.)
%% The constraint \neg\big((\UP\XCOMP{}$^+$ \OBJ) $<_f$ (\UP\OBJ)\big)
%% can thus be paraphrased as: it should not be the case that I have an
%% embedded object, at any level of embedding, whose c-structure precedes
%% that of my object.}
%
\begin{exe}
  \ex\label{ex:zk95:vprules}
  \begin{xlist}
    \ex\label{ex:zk95:vprulesa} \phraserule{VP}{\rulenode{DP*\\(\UP\XCOMP{}* \OBJ)=\DOWN} \rulenode{\BAR{V}\\\UP=\DOWN}}
    \ex\label{ex:zk95:vprulesb} \phraserule{\BAR{V}}{\rulenode{V\\\UP=\DOWN} \optrulenode{\BAR{V}\\(\UP\XCOMP)=\DOWN\\$\neg$\big((\UP$\XCOMP^+$ \OBJ) $<_f$ (\UP\OBJ)\big)}}
  \end{xlist}
\end{exe}
%
This analysis abandons the nested c-structure of the VP in favour of a
flat one, which moves us back in the direction
of~(\ref{ex:xs:evers}). The functional uncertainty equation on the
object DP in~(\ref{ex:zk95:vprulesa}) allows connecting the object to a predicate at
any depth of \XCOMP embedding, and the general principles of f-structure
coherence and completeness make sure each object is matched to exactly
one predicate. The functional precedence constraint on the \BAR{V} node 
in~(\ref{ex:zk95:vprulesb}) prevents more embedded objects from preceding less
embedded ones. Together, these f-structure
constraints force the same relation between Mittelfeld objects and right bracket verbs as the
c-structures subtrees in \citeauthor{BKPZ:Dutch}'s analysis. Moreover,
the interaction between functional uncertainty and the standard LFG
approach to constituent coordination lets us handle sentences
like~(\ref{ex:Germ:19}) correctly.

\citet{zaenen-kaplan1995} also apply the
combined use of functional uncertainty and functional precedence to
Zürich German, where cross-serial dependencies are observed as well.

\subsubsubsection{Nested dependencies\label{sec:Germanic:nesteddep}} The analyses of the structures that
would give rise to consistently nested dependencies all come from LFG
work on Standard German. However, explicit discussion of constructions
with multiple objects at different levels of embedding is rare in this
part of the literature -- perhaps because the modelling of these
dependencies is not seen as particularly problematic. We therefore do
not always fully know how the relevant nested dependencies are to
be derived in these LFG models.

Some authors assume nested VPs, which rather naturally correspond to
nested dependencies between objects and verbs, even when this
consequence is not a central concern. One example is the grammar
fragment of \citet{netter:1988}, who gives annotated c-structures like
the one in~(\ref{ex:vpcenter}).
%
\begin{exe}
  \ex\label{ex:vpcenter}
  \begin{xlist}
    \ex German \citep[§1, example C4]{netter:1988}\\
    \gll \ldots{}dass Leo sie {zu kommen} gebeten hat.\\
    \phantom{\ldots}\COMP{} Leo her come.\textsc{zuinf} asked has\\
    \glt `\ldots that Leo asked her to come.'

    \ex
    \begin{forest}
      [S
        [\rulenode{(\UP\SUBJ)=\DOWN\\DP}
          [{\gll
              Leo\\
              Leo\\}
            [Mf\streep{7em}, no edge, tier=fields]
        ]]
        [\rulenode{(\UP\XCOMP)=\DOWN\\VP}
          [\rulenode{(\UP\OBJ)=\DOWN\\NP}
            [{\gll
                sie\\
                her\\}]]
          [\rulenode{(\UP\XCOMP)=\DOWN\\VP} [\rulenode{\UP=\DOWN\\V}
              [{\gll
                  {{zu kommen}}\\
                  come.\textsc{zuinf}\\}, l sep=0pt
                [rb\streep{12em}\hspace*{4em}, no edge, tier=fields]
          ]]]
            [\rulenode{\UP=\DOWN\\V}
              [{\gll
                  gebeten\\
                  asked\\}]]
        ]
          [\rulenode{\UP=\DOWN\\V}
            [{\gll
                hat\\
                has\\}]]
      ]
    \end{forest}
  \end{xlist}
\end{exe}
%
We also find nested VPs in \citet{Choi1999}, where the combined
Mittelfeld and verb cluster of the subordinate clause in~(\ref{ex:WGerm:22}a) would
get the structure given in~(\ref{ex:WGerm:22}b).\footnote{\citet{Choi1999} does
not provide a tree for this exact sentence, but does show a more
complex example with a comparable structure. In addition,
\citet{Choi1999} never explicitly motivates the specific c-structure
associated with embedded verbs in the VP. Nevertheless, we can infer the
structure given here from the examples and discussion there.}\pagebreak

\begin{exe}
  \ex\label{ex:WGerm:22} German
  \begin{xlist}
    \ex \gll \ldots{}dass der Junge dem Mann geholfen hat.\\
    \phantom{\ldots}\COMP{} the.\NOM.\SG{} boy the.\DAT.\SG{} man helped has\\
    \glt `\ldots that the boy has helped the man.'
  \ex
  \begin{forest}
    [S
      [NP
        [{\gll
           der Junge\\
           the.\NOM.\SG{} boy\\}, roof
          [Mf\streep{14.5em}\hspace*{6em}, no edge, tier=fields]
        ]
      ]
      [VP
        [VP
          [\BAR{V}
            [NP
              [{\gll
                  dem Mann\\
                  the.\DAT.\SG{} man\\}, roof]
            ]
            [V [{\gll
                  geholfen\\
                  helped\\}, l sep=0pt
                [rb\streep{6em}\hspace*{3em}, no edge, tier=fields]
              ]
            ]
          ]
        ]
        [V
          [{\hspace*{2em}{\gll
                hat\\
                has\\}\hspace*{2em}}
          ]
        ]
      ]
    ]
  \end{forest}
  \end{xlist}
\end{exe}
%
In addition to a nested VP structure, the tree in~(\ref{ex:WGerm:22}b) shows the
subject appearing in S and the object inside the VP. Any deviations
from the canonical word order implied by this structure are
modelled using optional adjunction of objects to higher
positions. \citet{Choi1999} motivates this partially
configurational structure for German by appealing to contrasts like
the following: a verb and its object can be realized together in the
Vorfeld~(\ref{ex:vptop}a), whereas -- it would appear -- a verb and its
subject cannot~(\ref{ex:vptop}b).
%
\begin{exe}
  \ex German\label{ex:vptop}\citep[§2.1, example 12]{Choi1999}
  \begin{xlist}
    \ex[]{\gll $[$Dem Mann geholfen$]$ hat der Junge.\\
    \phantom{$[$}the.\DAT.\SG{} man helped has the.\NOM.\SG{} boy\\
    \glt `Help the man, the boy did.'\\}
    \ex[*]{\gll $[$Der Junge geholfen$]$ hat dem Mann.\\
    \phantom{$[$}the.\NOM.\SG{} boy helped has the.\DAT.\SG{} man\\}
  \end{xlist}
\end{exe}
%
Under \citeauthor{Choi1999}'s analysis, this contrast follows
straightforwardly by assuming that a VP, unlike an S, can be put in
the Vorfeld.

These analyses with nested VPs, which in principle could
directly yield the pattern of nested dependencies we find in German,
do \textit{not} have a single node containing the whole right bracket
and nothing else. Put differently, they do not include the verb
cluster as such. This contrasts with the analyses we saw for the
cross-serial dependency languages (Dutch, Zürich German) above, where
the verb cluster exactly matched a \BAR{V} node.

Proposals for Standard German that have a c-structure node corresponding to the
verb cluster do exist in the LFG literature. One prominent such proposal is made by 
\citet{Berman2003}, who rejects \citeauthor{Choi1999}'s claim that the
German VP includes the object but excludes the subject, on the basis
of data like~(\ref{ex:WGerm:24}), which, in contrast to~(\ref{ex:vptop}b), is a successful example of Vorfeld
realization of a verb with its subject.
%
\begin{exe}
  \ex German\label{ex:WGerm:24} \citep[§3.2.3.2, example 28a]{Berman2003}\\
   \gll $[$Kinder gespielt$]$ haben hier noch nie.\\
    \phantom{$[$}children played have here yet never\\
    \glt `Children have never played here.'\\
\end{exe}
%
Instead of assuming that there are canonical positions for subjects
and objects in S and VP respectively, and that only scrambled objects
are adjoined, \citeauthor{Berman2003} does away with S completely, and
always adjoins Mittelfeld arguments to VP. Furthermore,
\citeauthor{Berman2003} posits that verbs in the verb cluster are
combined by head adjunction. The structure of~(\ref{ex:vptop}a) under this
model is then~(\ref{ex:compcp:berman}).
%
\begin{exe}
  \ex\label{ex:compcp:berman}
  \begin{forest}
    [VP
      [DP
        [{\gll
            der Junge\\
            the.\NOM.\SG{} boy\\}, roof
          [Mf\streep{13.5em}\hspace*{6em}, no edge, tier=fields]
        ]
      ]
      [VP
        [DP
          [{\gll dem Mann\\
              the.\DAT.\SG{} man\\}, roof]
        ]
        [VP
          [V$^0$
            [V$^0$
              [{\gll
                  geholfen\\
                  helped\\}, l sep=0pt
                [rb\streep{6em}\hspace*{3em}, l=0pt, no edge, tier=fields]
              ]
            ]
            [V$^0$
              [{\gll
                  hat\\
                  has\\}]
            ]
          ]
        ]
      ]
    ]
  \end{forest}
\end{exe}
%
Since the association of arguments with their predicates can no longer
rely on positional grammatical function annotations,
\citet{Berman2003} argues that case is responsible for this
association in German. This is modelled using the standard approach of
conditional expressions that relate specific cases to specific
functions, for instance,
(\DOWN\CASE)=\ACC$\Rightarrow$(\UP\OBJ)=\DOWN. 
However, \citeauthor{Berman2003} does not discuss how this standard
approach should be extended to allow embedded objects of coherently
combined verbs in general, and it is not clear how one would correctly
constrain the projection of multiple Mittelfeld objects onto f-structures that
are embedded under one or more layers of \XCOMP{} without resorting
to nested VPs or functional uncertainty.\footnote{\citet{Berman2003}
partially sidesteps the issue by (tacitly) assuming that auxiliaries
add features and do not create \XCOMP embeddings. This means that, for
instance, \textit{lobte} `praised' and \textit{hat gelobt} `has
praised' both have their objects directly in the containing
f-structure as \OBJ{}s. However, since not all coherently combining
verbs can be analysed as auxiliaries and some clearly have enough
lexical content to warrant their own \PRED values, this does not
completely address the problem.}

We have seen that the (idealized) Dutch and German patterns are mirror images
in terms of the order of the verbs in the verb cluster. The approach
of \citet{ZaenenKaplan2002:Subsumption} capitalizes on this by taking
the mirror image of the \BAR{V} rule for Dutch in~(\ref{ex:zk95:vprules}b) as
the basis for their analysis of the German verb cluster~(\ref{ex:zk02:vprules}).
%
\begin{exe}
    \ex \label{ex:zk02:vprules}\phraserule{\BAR{V}}{\optrulenode{\BAR{V}\\(\UP\XCOMP)=\DOWN\\$\neg$((\UP$\XCOMP^+$\,\OBJ) $<_f$ (\UP\OBJ))} \rulenode{V\\\UP=\DOWN}}
\end{exe}
%
As before, in the proposal for Dutch, the highest \BAR{V} node
corresponds directly to the right bracket and functional uncertainty
solves the relation of Mittelfeld material to embedded verbs without
having to assume nested VPs.

\subsubsubsection{Variation} I already mentioned at the beginning of this subsection that
characterizing languages as having either cross-serial or nested
dependencies is an oversimplification. For instance, both German and
Dutch allow further variation in the ordering of elements in the
verbal cluster. Moreover, in Zürich German -- amongst other CWG languages --
Mittelfeld and right bracket material can mix to some extent.\largerpage[-1]

In German, \textit{Oberfeldumstellung} (\citealt{bech:1955}: Vol I, §62–§66; an alternative term is auxiliary flip) can occur with three-verb combinations where V$_1$ is a
perfect or passive auxiliary, and V$_2$ is itself a coherently
combining verb that selects an infinitive. In this construction, the
verb cluster has the order V$_1$V$_3$V$_2$, and IPP is triggered for
V$_2$. Contrast the ``regularly ordered''~(\mex{1}a) with the
Oberfeldumstellung in~(\mex{1}b).\footnote{\textit{Oberfeldumstellung} also occurs with longer
verb clusters. Furthermore, there is a (possibly regional)
construction called \textit{Zwischenstellung} that has
V$_3$V$_1$V$_2$. See~\citet{cook:2001:revdiss} for empirical
discussion and an analysis.}

\begin{exe}
  \ex German \citep[§1.4, example 1.31]{cook:2001:revdiss}
  \begin{xlist}
    \ex{\gll \ldots{}dass ich dich kommen gesehen habe.\\
      \phantom{\ldots}\COMP{} I you.\ACC.\SG{} come.\INF{} seen have.\PRS{}\\
      \glt `\ldots that I have seen you come.'}
    \ex{\gll \ldots{}dass ich dich habe kommen sehen.\\
      \phantom{\ldots}\COMP{} I you.\ACC.\SG{}  have.\PRS{} come.\INF{} see.\INF{}\\}
  \end{xlist}
\end{exe}
%
This word order variant is problematic for the nested VP models
mentioned above \citep[namely,][]{netter:1988,Choi1999}, since the verb
cluster-initial finite verb ``interrupts'' the embedded VP. Models in which a
c-structure node corresponds to the verb cluster
\citep[namely,][]{ZaenenKaplan2002:Subsumption,Berman2003,clement-etal:2002:lfg}
have an easier time capturing such variation. An analysis of this
variation can be found in~\citet[in terms of
  c-structure]{clement-etal:2002:lfg} and in \citet[in terms of the
  interaction between syntax and information
  structure]{cook:2001:revdiss}. An OT-LFG analysis of verb order in
Swiss German dialects is outlined in \citet{seiler:2007}.

Dutch verb clusters have so-called \textit{participle climbing} and
\textit{particle climbing}, which refer to the realization of
participles and particles to the left of the position expected from
the principle of ordering by increasing embedding. Example~(\mex{1})
shows the different positions a particle can occupy in a three-verb
cluster.
%
\begin{exe}
  \ex\label{ex:particleclimbing} Dutch\\
  \gll \ldots{}dat Jan {het liedje} \{mee\} zal \{mee\} hebben \{mee-\}gezongen.\\
       \phantom{\ldots}\COMP{} Jan {the song} \phantom{(}along will \phantom{(}along have \phantom{(}along-sung\\
       \glt `\ldots that Jan will have sung along to the song.'
  %% \ex\label{ex:creep:participle}
  %% \gll \ldots{} dat Jan {het liedje} (gezongen) zal (gezongen) hebben (gezongen).\\
  %%      {} \COMP{} Jan {the song} {\phantom{(}sung} will \phantom{(}sung have {\phantom{(}sung}\\
  %%      \glt \ldots that Jan will have sung to the song.
\end{exe}
%
\citet{KaplanZaenen2003} adapt their earlier model of the
Dutch Mittelfeld and verb-cluster to allow these and further variants,
and to capture the IPP effect. \citet{poortvliet:2015:lfg} is a
further development of this model.

Zürich German has cross-serial dependencies, like Dutch, but in
addition allows the nominal and verbal material to mix, as in~(\mex{1}).
\begin{exe}
  \ex Zürich German\\
  \gll \ldots{}das er sini chind wil mediziin la schtudiere.\\
       \phantom{\ldots}\COMP{} he his children wants medicine let study\\
       \glt `that he wants to let his children study medicine.'
  \end{exe}
%
\citet{zaenen-kaplan1995} use the combination of functional precedence
and functional uncertainty developed for Dutch to capture these
data. Another case of mixing verbal and non-verbal material can be
found in Standard German, which allows a variant of Oberfeldumstellung
where $V_1$ precedes a collocational nominal complement of $V_3$. An
analysis of this construction can be found
in~\citet{cook:2001:revdiss}.


\subsubsection{Nachfeld\label{sec:Germanic:nf}}

The two options for adding Nachfeld material to the different
c-structures of the clause given above are 1) adjunction to any of the
nodes at the right periphery and 2) inclusion of one or more optional
daughters on the right-hand side of the relevant c-structure
rules. Adjunction is used by \citet{Berman2003}, who assumes Nachfeld
occupants (typically PPs, VPs or CPs) are right-adjoined at the VP
level. \citet{rohrer:1996}, \citet{clement-etal:2002:lfg} and
\Citet{vanderbeek:2005} model the Nachfeld as an optional daughter in
the node covering the whole clause. \citeauthor{zaenen-kaplan1995}
(\citeyear{zaenen-kaplan1995,ZaenenKaplan2002:Subsumption}), and
\citet{KaplanZaenen2003} insert the optional daughter in the node
covering the Mittelfeld/right bracket.

In Dutch and German, the non-finite complement of an incoherently
combining verb appears in the Nachfeld. In Dutch, Mittelfeld placement of such a complement is ruled out~(\mex{1}), but in German it is allowed (see for
  instance \citealt{rohrer:1996} for examples).
%
\begin{exe}
  \ex Dutch\\
      {\glll
        Vf\streep{.25em} lb\streep{.5em} \phantom{\{*}Mf\streep{6.5em} {} {} rb\streep{2em} \phantom{\{}Nf\streep{6.75em}\\
        Hij\hspace*{1pt} had \{*geen auto {te kopen}\} beloofd \{geen auto {te kopen}\}.\\
        he had \phantom{\{*}no car buy.\textsc{teinf} promised \phantom{\{}no car buy.\textsc{teinf}\\
        \glt `He had promised not to buy a car.'}
\end{exe}
%
To facilitate lexical specification of whether a verb combines
coherently or not, and the formulation of placement restrictions on
the non-finite verbal complement, \citet{rohrer:1996},
\citeauthor{zaenen-kaplan1995}
(\citeyear{zaenen-kaplan1995,ZaenenKaplan2002:Subsumption}), and
\citet{KaplanZaenen2003} associate coherence with selecting an
\XCOMP{} and incoherence with selecting a \COMP{}. The relevant
c-structure rule from \citet{KaplanZaenen2003} is an extension of
(\ref{ex:zk95:vprules}a) and is given here in slightly simplified form
as~(\mex{1}).

\begin{exe}
    \ex \phraserule{VP}{%
      \rulenode{DP*\\(\UP\XCOMP{}* (\COMP) \OBJ)=\DOWN}
      \rulenode{\BAR{V}\\\UP=\DOWN}
      \optrulenode{VP\\(\UP\XCOMP{}* \COMP)=\DOWN}
    } 
\end{exe}
%
The optional rightmost daughter contains a non-finite complement in the
Nachfeld, assigned \COMP{}.\footnote{The distinction between \COMP{}
and \XCOMP{} is that of complements that supply their own subject
(closed complements) and complements that do not supply their own
subject (open complements). Since non-finite \COMP{s} do not have an
overt subject, they therefore must have an f-structure subject $\PRED
\mbox{ `pro'}$, whose interpretation is equated to one of the
arguments of the selecting verbs using anaphoric control. See
\citet[Chapter 12, §3]{dalrymple01} for a discussion of anaphoric
control.}

The rule in~(\mex{0}) also allows for the so-called \textit{third
  construction}, a marked construction in Dutch and German in which a
dependent of an incoherently combined non-finite complement in the
Nachfeld is realized in the Mittelfeld of the containing clause. In
terms of word order, this construction therefore mixes properties of
coherent and incoherent combination. An example of the third
construction is presented in~(\mex{1}). Note that the lack of an IPP effect on
\textit{geprobeerd} `tried' shows that we are dealing with incoherent
combination.

\begin{exe}
      \ex Dutch\\{\gll Hij had een auto geprobeerd {te kopen}.\\
      he had a car tried buy.\textsc{teinf}\\
    \glt `He had tried to buy a car.'}
\end{exe}
%
The c-structure rule in~(\mex{-1}) captures the third construction by
the functional uncertainty-based grammatical function assignment of
DPs in the Mittelfeld: the optional \COMP{} in the path allows it to
reach into an incoherently combined
complement.\footnote{\citet{KaplanZaenen2003} are not concerned with
CP complements -- that is, finite complement clauses -- but if these
are assigned \COMP as well, the analysis of the third construction
sketched here will need to be further constrained to prevent lifting
dependents from finite subordinate clauses into the Mittelfeld.} 
LFG analyses of the the German third construction are discussed in
\citet{rohrer:1996} and \citet{KaplanZaenen2003}.%\citet{ZaenenKaplan2002:Subsumption}.

\subsection{Topics related to the left periphery\label{sec:Germanic:leftperi}}

\subsubsection{Topicalization\label{sec:Germanic:top}}

In the context of the verb-second CWG languages, 
topicalization refers to Vorfeld placement of material, in particular material
that is \textit{not} put there by default. Roughly, then,
topicalization is Vorfeld placement of anything but the local
subject. The term topicalization is used irrespective of whether the
Vorfeld occupant is a topic or not.  In both German and Dutch, the
Vorfeld may be occupied by a categorially and functionally wide range
of constituents. It is also a target position for material extracted
from embedded clauses and phrases.

\citet[Chapter 6]{Berman2003} formally distinguishes two different
types of topicalization for German, depending on whether the Vorfeld
constituent is local to the matrix clause or whether a long-distance
dependency is involved. \citeauthor{Berman2003} introduces this
distinction on the basis of observations from weak cross-over, which
will be discussed in~\sectref{sec:Germanic:weakxover}, below. In either case,
the Vorfeld is Spec-CP, and its definition is part of the
straightforward c-structure rule in~(\mex{1}).

\begin{exe}
  \ex \phraserule{CP}{%
    \rulenode{XP\\(\UP\DF)=\DOWN}
    \rulenode{\BAR{C}\\\UP=\DOWN}
    }
\end{exe}
%
When material local to the f-structure projected from CP is put in the
Vorfeld, \citeauthor{Berman2003} assumes information like case and
agreement drives the association with grammatical function, just as it
does in the Mittelfeld -- see the earlier discussion of
\citeauthor{Berman2003}'s model in~\sectref{sec:Germanic:mfrb}, around
example~(\ref{ex:compcp:berman}). For long-distance dependencies,
\citeauthor{Berman2003} posits the presence of a trace at the
extraction site, annotated with an inside-out functional uncertainty
equation to incorporate the f-structure of the Vorfeld constituents --
which by~(\mex{0}) is the \DF of the f-structure of the whole clause -- 
into the extraction site's f-structure.\footnote{See
\citetv{chapters/LDDs} for more
information on modelling long-distance dependencies using inside-out
functional uncertainty.}

German has embedded verb-second clauses with bridge verbs, provided the
complementizer is absent, as in~(\mex{1}a). Extraction out of such
embedded clauses is also allowed, on the condition that none of the
clauses involved in the long-distance dependency has material in
Spec-CP (that is, no intermediate clause has a Vorfeld occupant). This is shown in the
contrast~(\mex{1}b,c).
%
\begin{exe}
  \ex German \citep[b,c from][§6.2.4, examples 23, 24]{Berman2003}
  \begin{xlist}
    \ex[]{\gll Ich glaube, (*dass) der Hans sagte gestern, (*dass) die Maria hat den Peter eingeladen.\\
      I believe \phantom{(*}\COMP{} the.\NOM{} Hans said yesterday,  \phantom{(*}\COMP{} the.\NOM{} Maria has the.\ACC{} Peter invited\\
      \glt `I think Hans said yesterday that Maria invited Peter.'}
    \ex[]{\gll
      Den Peter glaube ich, sagte der Hans gestern, hat die Maria eingeladen.\\
       the.\ACC{} Peter think I said the.\NOM{} Hans yesterday has the.\NOM{} Maria invited \\
    }
    \ex[*]{\gll
      Den Peter glaube ich, gestern sagte der Hans, hat die Maria eingeladen.\\
      the.\ACC{} Peter think I yesterday said the.\NOM{} Hans has the.\NOM{} Maria invited \\
    }
  \end{xlist}
\end{exe}
%
\citeauthor{Berman2003} captures this restriction with an off-path constraint \mbox{$\neg(\rightarrow\DF)$} on the functional
uncertainty equation for extractions. Since only Spec-CP introduces \DF{} in \citeauthor{Berman2003}'s model,\footnote{\label{fn:dfnosubj}For this to hold, we need to understand \DF as not
including \SUBJ, since subjects can be introduced in other positions in
the clause, too. Indeed, as the example shows, there is no ban on subjects occurring
anywhere in the path of a long-distance dependency, as long as they do
not occur in the Vorfeld. An unfortunate side effect of taking \DF as not including \SUBJ would be that the special Vorfeld privileges of subjects, see \sectref{sec:Germanic:vfsubobjasym}, remain unmodelled.} this effectively rules out examples like (\mex{0}c).\footnote{However, consider the following data:
%
\begin{exe}
  \ex
  \begin{xlist}
    \ex {\gll
      Ich denke hier, Sie sollten etwas präziser sein.\\
      I think here you should somewhat precise.\textsc{comparative} be\\
      \glt `I think here(:) you should be a bit more precise.'}
  \ex Hier denke ich, Sie sollten etwas präziser sein.

  \end{xlist}
  \ex
  \begin{xlist}
    \ex Ich denke, Sie sollten hier etwas präziser sein.
    \glt   `I think(:) you should be a bit more precise here.'
  \ex Hier denke ich, sollten Sie etwas präziser sein.
  \end{xlist}
\end{exe}
%
Although the off-path constraint against \DF{}s gets rid of the form
(i$\,$b) as a realization of the meaning of~(ii), it does not block
(ii$\,$b) as a realization of the meaning of (i). In other words, the
off-path constraint itself leaves unexplained why the embedded V1 of
(ii$\,$b) signals that it is involved in an extraction. A possible
line of defence against this criticism is to appeal to a form of
Economy of Expression: the embedded V1 is a slightly more complex
structure than embedded V2, a complexity that is not needed for the
relational information expressed in~(i).}

In non-LFG work, \citet{reis:1996} argues that sentences
like~(\mex{0}b) are only apparent cases of extraction, and that they
involve parenthetical constructions, instead. An LFG analysis of
German parentheticals with bridge verbs is given in
\citet{fortmann:2006:lfg}, although he does not consider the exact
type of sentence discussed here.

German allows topicalization of VPs (as discussed above
in~\sectref{sec:Germanic:nesteddep}, example \ref{ex:vptop}) and, in the case
of coherent combination, topicalization of partial VPs. For instance,
in~(\mex{1}), the main verb and its accusative object are realized in
the Vorfeld, whereas the dative object is in the Mittelfeld.
%
\begin{exe} 
  \ex{German (similar to \citealt[3a]{Nerbonne94a})\\
  \gll Ein Märchen erzählen wird er ihr.\\
    a {fairy tale} tell.\INF{} will he her.\DAT{}\\
  \glt `He will tell her a story.'}
\end{exe}
%
VP topicalization can in principle be modelled using the
standard mechanisms. For instance, under the assumption that
coherently combined verbal complements are \XCOMP{}s and if we use
outside-in functional uncertainty, a c-structure rule like
in~(\mex{1}) implements topicalization of coherently combined VPs. 
%
\begin{exe}
  \ex \phraserule{CP}{%
    \rulenode{VP\\(\UP\TOPIC)=\DOWN\\(\UP$\XCOMP^+$)=\DOWN}
    \rulenode{\BAR{C}\\\UP=\DOWN}
    }
\end{exe}
%
If the rule for VPs allows partial VPs, rule~(\mex{0}) says very
little about which material is required to be present in the fronted
VP and which material can be left to be realized in-situ. Potentially,
then, this also captures partial VP
topicalization. \citet{ZaenenKaplan2002:Subsumption} problematize two
aspects of such a straightforward implementation: First, in the case
of partial VP topicalization, the resulting f-structure for the whole
VP contains the combined topicalized and in-situ material, and
therefore there is no way to see at f-structure which part of the VP
was topicalized. This is problematic for approaches to information
structure that associate information status with
f-structures. Secondly, the approach would erroneously allow examples
like~(\mex{1}).
%
\begin{exe}
  \ex[*]{%
    \gll $[$Ihr ein Märchen$]$ wird er erzählen.\\
         \phantom{$[$}her.\DAT{} a {fairy tale} will he tell.\INF{}\\
    \glt `He will tell her a fairy tale.'
         }
\end{exe}
%
The preverbal material is here analysed as a headless VP, which is
generally only allowed postverbally.\footnote{I am aware that this
claim is too broad. See for instance \citet{muller-etal:2012}, who use
headless Vorfeld VPs in their analysis of apparent multiple fronting
in German. However, a discussion of exceptions to this rule would take
us too far away from the main topic here.}

\citet{ZaenenKaplan2002:Subsumption} solve these problems by replacing
unification with \textit{subsumption} in the functional uncertainty
annotation of Spec-CP, along the lines of~(\mex{1}).
%
\begin{exe}
  \ex \phraserule{CP}{%
    \rulenode{VP\\(\UP\TOPIC)=\DOWN\\\DOWN$\sqsubseteq$\,(\UP$\XCOMP^{+})$}
    \rulenode{\BAR{C}\\\UP=\DOWN}
    }
\end{exe}
%
This directly solves the first problem, since the information in
\TOPIC{} now no longer contains the f-structure for the whole VP, but
only information projected from the material in the Vorfeld. It also
solves the second problem, since, as shown in~(\mex{1}), the f-structure for
the example in~(\mex{-1}) under \TOPIC{} now no longer meets LFG's \textit{coherence} condition
-- it contains arguments but no predicate to select them.
\newpage
\begin{exe}
  \ex F-structure for (\mex{-1}), which violates the coherence condition:\\[.5ex]
  \attop{\avm[style=fstr,pic,picname=ihreinmaerchen]{
        [ topic & \node{a}{%
            [ obj$_{\theta}$ & [pred & `{\upshape pro}']\vspace{\smallskipamount}\\
              obj & [pred & `{\upshape fairy tale}']
            ]
          }\vspace{.25ex}\\
          pred & `{\upshape will}\arglist{\XCOMP}\SUBJ{}'\\
          subj & \rnode{b}{[pred & `{\upshape pro}']}\vspace{\smallskipamount}\\
          xcomp & \node{c}{
            [ pred & `{\upshape tell\arglist{\SUBJ,\OBJ,\OBJTHETA}}'\\
              subj & \rnode{d}{\vphantom{x}}\\
              obj$_{\theta}$ & [pred & `{\upshape pro}']\vspace{\smallskipamount}\\
              obj & [pred & `{\upshape fairy tale}']
            ]
          }\hspace{.5ex}
  ]}}%
  \CURVE[3.25]{-2pt}{0}{b}{1pt}{0}{d}
  \begin{tikzpicture}[remember picture,overlay]
    \draw[-{Stealth[length=3.75pt,width=2.5pt]}, rounded corners=6, line width=.5pt] (ihreinmaerchen-a.east) +(-2pt,0) -- ++(1.5,0) |- (ihreinmaerchen-c.east) node[pos=.25, fill=white] {$\sqsubseteq$} ;
  \end{tikzpicture}
\end{exe}
%
Finally, a reformulation of the \textit{completeness} condition to
take subsumption relations into account\footnote{``An f-structure $g$
is complete if and only if each of its subsidiary f-structures is
either locally complete or subsumes a subsidiary f-structure of $g$
that is locally complete'' \citep[{[24]}]{ZaenenKaplan2002:Subsumption}.}
allows the f-structures resulting from topicalizing a partial VP, as
illustrated in (\mex{1}).
%

  \begin{exe}
  \ex
  \begin{xlist}
  \ex{\gll Ein Märchen erzählen wird er ihr.\\
    a {fairy tale} tell.\INF{} will he her.\DAT{}\\
  \glt `He will tell her a story.'}
  \ex{%
    \attop{\avm[style=fstr,pic,picname=einmaerchenerzaehlen]{
        [ topic & \node{a}{%
            [ pred & `{\upshape tell\arglist{\SUBJ,\OBJ,\OBJTHETA}}'\\
              obj & [pred & `{\upshape fairy tale}']
            ]
          }\vspace{.25ex}\\
          pred & `{\upshape will}\arglist{\XCOMP}\SUBJ{}'\\
          subj & \rnode{b}{[pred & `{\upshape pro}']}\vspace{\smallskipamount}\\
          xcomp & \node{c}{
            [ pred & `{\upshape tell\arglist{\SUBJ,\OBJ,\OBJTHETA}}'\\
              subj & \rnode{d}{\vphantom{x}}\\
              obj$_{\theta}$ & [pred & `{\upshape pro}']\vspace{\smallskipamount}\\
              obj & [pred & `{\upshape fairy tale}']
            ]
          }\hspace{.5ex}  
    ]}}
    \CURVE[3.25]{-2pt}{0}{b}{1pt}{0}{d}
    \begin{tikzpicture}[remember picture,overlay]
    \draw[-{Stealth[length=3.75pt,width=2.5pt]}, rounded corners=6, line width=.5pt] (einmaerchenerzaehlen-a.east) +(-2pt,0) -- ++(1.5,0) |- (einmaerchenerzaehlen-c.east) node[pos=.25, fill=white] {$\sqsubseteq$} ;
  \end{tikzpicture}}

\end{xlist}
\end{exe}

\citet[§3.3]{Berman2003} solves the ungrammaticality of examples
structurally similar to~(\mex{-3}) by appealing to the endocentricity
principles as formulated for LFG, which include the constraint that
every lexical category must have an extended head
\citep[§7.2]{BresnanEtAl2016}. In~(\mex{-3}), neither the main verb
\textit{erzählen} nor the auxiliary \textit{wird} c-command the
material in the Vorfeld VP, which leaves the fronted VP without an
extended head. Note that this solution would not be compatible with
\citeauthor{ZaenenKaplan2002:Subsumption}'s conception of the German
clause, as they do not assume the German VP is necessarily
endocentric.\footnote{As mentioned in~\sectref{sec:Germanic:overall}, above,
\citeauthor{ZaenenKaplan2002:Subsumption} use the label S|VP for this
category.}



\subsubsection{The Vorfeld subject-object asymmetry\label{sec:Germanic:vfsubobjasym}}

In both German and Dutch, the main clause subject is privileged when it
comes to realization in the Vorfeld. In LFG work, this can be modelled
directly by annotating the Vorfeld position at c-structure explicitly
with (\UP\SUBJ)=\DOWN\ \citep{theiler-bouma:2012:lfg}, or by
annotating it with (\UP\DF)=\DOWN, under the assumption that the
grammaticalized discourse functions include the subject
\citep[§3.2.1]{Berman2003}.\footnote{I refer, however, to the comment in footnote~\ref{fn:dfnosubj}, where I point out that the analysis of German embedded verb-second clauses discussed there relies on the contradicting assumption that \SUBJ is \textit{not} part of \DF.} In an OT-LFG setting, \citet{Choi2001} posits
a high-ranking constraint \textsc{Subject-Left} that prefers early
realization of the subject, which includes realization in the Vorfeld.

One of the reflexes of this special relation between the Vorfeld and the main clause
subject is a contrast like the following:
%
\begin{exe}
  \ex German \citep{meinunger:2007:li}\\
  Wo ist das Geld?
  \glt `Where is the money?'
  \begin{xlist}
    \ex[]{%
      \gll Es liegt auf dem Tisch.\\
           it lies on the table\\
      \glt `It is on the table.'
    }
    \ex[*]{%
      \gll Es hat Bernd auf den Tisch gelegt.\\         
           it has Bernd on the table put\\
      \glt `Bernd (has) put it on the table.'
    }
  \end{xlist}
\end{exe}
%
Although the referent of the weak pronoun \textit{es} has the same
information status in both cases, it appears it can only occur in the
Vorfeld as a subject, and not as an object. This would fit in with any
of the approaches sketched above: being a subject alone is enough
reason to be allowed in the Vorfeld, but -- apparently -- the weak
pronoun \textit{es} is incompatible with any of the other information
structural functions of Vorfeld constituents.

The ban on object \textit{es} in the Vorfeld is not categorical,
however. \citeauthor{meinunger:2007:li} (\citeyear{meinunger:2007:li}, and references therein)  gives many
examples, and shows that the conditions under which object \textit{es}
can appear in the Vorfeld coincide with the conditions for the use of
the homonymic Vorfeld expletive used in the presentational
construction. In particular, the subject of the clause should not be
topic~(\ref{ex:WGerm:42}).
%
\begin{exe}
  \ex\label{ex:WGerm:42}
  \begin{xlist}
    \ex[]{%
      \gll Es hat \{jemand / *er\} geklaut\\
           it has \phantom{\{}someone {} \phantom{*}he stolen\\
      \glt `Someone / *he has stolen it'}
    \ex[]{%
      \gll Es hat \{jemand / *er\} das Geld geklaut\\
           \textsc{expl} has  \phantom{\{}someone {} \phantom{*}he the money stolen\\
      \glt `Someone / *he has stolen the money.'}
  \end{xlist}
\end{exe}
%
\citet{theiler-bouma:2012:lfg} capture this behaviour by assuming that
the common source of sentences with a Vorfeld object and of those with
Vorfeld expletive \textit{es} is the latter, the presentational
construction. The presence of \textit{es} in the Vorfeld signals exactly that
the main clause subject is not topic. This construction is modelled using
a c-structure rule that explicitly mentions the form of its first
daughter.
%
\begin{exe}
  \ex \phraserule{CP}{%
    \rulenode{NP\\%
      (\DOWN\textsc{form})\,=$_c$\,\textsc{es}\\%
      ((\UP\SUBJ)$_{\sigma}$\,\DF)\,$\neq$\,\TOPIC\\%
      $\big($(\UP\XCOMP*\OBJ)=\DOWN$\big)$}
    \rulenode{\BAR{C}\\\UP=\DOWN}
  }
\end{exe}
The optional assignment to object anywhere in the coherence domain of the
clause is what allows object \textit{es} to appear in the
Vorfeld, under the same circumstances as the presentational
construction's expletive. Expletive \textit{es} also shows up in other
situations; see~\sectref{sec:Germanic:subjectless} below.
 

\subsubsection{Left dislocation}

Thus far, we have not considered the \textit{lead} field, which is
positioned before the Vorfeld in the topological model
given in~(\ref{ex:topmodel}) and which we characterized as reserved for
material more loosely connected to the clause proper. We can
distinguish several \textit{left dislocation} phenomena that target
the lead.\footnote{We can likewise talk of right-dislocated
material, positioned in the tail, but since we are not aware of
any LFG discussions of right dislocation, we will ignore this phenomenon in this chapter.} Two questions raised by this broadened
view of the clause are 1) how tightly left-dislocated material is
coupled to the clause, and 2) whether there are phenomena that we have treated
as Vorfeld occupation that are better analysed as left dislocation
with an empty Vorfeld?

The first question is central in \citet{zaenen:1996}, who studies
contrastive left dislocation~(\ref{ex:WGerm:44}b) in Dutch and Icelandic, and
asks whether this should be treated as topicalization~(\ref{ex:WGerm:44}a) or
as a hanging topic~(\ref{ex:WGerm:44}c). The former counts as a well-integrated
part of the clause, the latter has a looser relation to the
clause.\largerpage[-1]\pagebreak

\begin{exe}
\ex\label{ex:WGerm:44} Dutch
  \begin{xlist} 
  \ex \gll Jan wil ik hier nooit meer zien.\\
           Jan want I here never again see\\
  \ex \gll Jan die wil ik hier nooit meer zien.\\
           Jan \DEM{} want I here never again see\\          
  \ex \gll Jan, ik wil hem hier nooit meer zien.\\
           Jan  I want him here never again see\\
  \glt `Jan, I never want to see (him) here again'
  \end{xlist}
\end{exe}
%
On the basis of categorial constraints on different kinds of
left-dislocated material and on the basis of binding data,
\citeauthor{zaenen:1996} concludes that contrastive left-dislocation
patterns with topicalization in both languages. She proposes an
analysis in which the contrastively left-dislocated material is
connected to the clause using the same functional uncertainty
equations we normally assume for topicalized material in the
Vorfeld. The pronominal element in the Vorfeld in a contrastive
left-dislocation is taken to be an (f-structure) adjunct to the
left-dislocated material, and does not itself engage in the
long-distance dependencies directly.

The second question underlies the discussion in
\citet[§7.4]{Berman2003}, which revolves around the
contrast illustrated in~(\ref{ex:WGerm:45}).

\begin{exe}
  \ex\label{ex:WGerm:45} German \citep[§7.4, examples 58--61]{Berman2003}
  \begin{xlist}
    \ex \gll Dass die Erde rund ist, (das) hat ihn gewundert.\\
             \COMP{} the earth round is \phantom{(}\DEM.\NOM{} has him.\ACC{} surprised\\
        \glt `That the earth is round(, that) surprised him.'
    \ex \gll {Dass die Erde rund ist,} (das) hat er nicht gewusst.\\
     {} \phantom{(}\DEM.\ACC{} has he not known\\
    \glt `That the earth is round, (that) he didn't know.'
    \ex \gll {Dass die Erde rund ist,} *(dessen) war sie sich nicht bewusst.\\
    {} \phantom{*(}\DEM.\GEN{} was she \REFL{} not aware\\
    \glt `That the earth is round, of that she wasn't aware.'
    \ex \gll {Dass die Erde rund ist,} *(darüber) hat sie sich gewundert.\\
    {} \phantom{*(}about.\DEM{} has she \REFL{} surprised\\
    \glt `That the earth is round, that she was surprised about.'
  \end{xlist}
\end{exe}

In~(\ref{ex:WGerm:45}a,b), the fronted CP appears to alternate between being
left-dislocated (with resumption) and appearing in the Vorfeld
(without), whereas in~(\ref{ex:WGerm:45}c,d), the fronted CP must be
left-dislocated. \citeauthor{Berman2003} gives an LFG
interpretation of an existing approach in which this alternation is
only apparent, and the CP is \textit{always} left-dislocated. The
difference in~(\ref{ex:WGerm:45}a,b) is that in German, nominative and accusative
topics may be dropped from the Vorfeld. Whether the resumptive
demonstrative pronoun is realized at c-structure or not, its
f-structure presence is constant, and it is this which is assigned a
grammatical function. The left-dislocated CP is connected
anaphorically to the resumptive pronoun.


\subsubsection{Split NPs\label{sec:Germanic:splitnps}}

The split NP construction in German involves multiple NPs at
different positions in the clause which together describe one argument. The
first NP occurs in the Vorfeld of the top level clause, and a further NP occurs
somewhere further down in the Mittelfeld of a possibly embedded
clause. An example is~(\ref{ex:WGerm:46}a).
%
\begin{exe}
  \ex\label{ex:WGerm:46} German \citep[§1]{kuhn2001}
  \begin{xlist}
  \ex%
  \gll $[$Ein Schwimmbad$]$ hat er sich noch $[$keins$]$ gebaut.\\
  \phantom{$[$}a {swimming pool} has he \REFL{} yet \phantom{$[$}none built.\\
      \ex%
      \gll Er hat sich noch $[$kein Schwimmbad$]$ gebaut.\\
      he has \REFL{} yet  \phantom{$[$}no {swimming pool} built\\
      \glt `He hasn't built a swimming pool yet.'
  \end{xlist}
  \end{exe}
%
A striking property of the two NPs \textit{ein Schwimmbad} and
\textit{keins} is that they both have the form of complete NPs: the
first NP includes a determiner for the head count noun, the second NP
involves the independent form \textit{keins} `none', rather than the form
\textit{kein} `no', which is used when a nominal head is realized in the
NP itself~(\ref{ex:WGerm:46}b).

\citet{kuhn2001} proposes a solution in terms of an LFG variant
with linear logic-based semantics. Semantically, the clause-internal
NP is a regular elliptical NP; the job of the Vorfeld NP is to supply a
property as antecedent. By assuming that the form of the NP can be
syntactically determined completely in terms of c-structure, treating
the two NPs as c-structurally independent, but projecting to the same
f-structure, the form-related characteristics of the NPs can be made to
follow.

\subsubsection{Asymmetric coordination}

\citet{frank:2006} gives an analysis of asymmetric coordination puzzles in German, like the \textit{subject gap with fronted finite verb} (SGF) coordination in~(\ref{ex:WGerm:47}a).
\largerpage[-1]\pagebreak

\begin{exe}
  \ex\label{ex:WGerm:47} German \citep[§3.2]{frank:2006}
  \begin{xlist}
    \ex[]{%
      \gll In den Wald $[$ging der Jäger] und [fing einen Hasen]\\
           in the.\ACC{} forest \phantom{$[$}went the.\NOM{} hunter and  \phantom{$[$}caught a.\ACC{} hare\\
      \glt `The hunter went into the woods and caught a hare.'
    }
    \ex[*]{%
      \gll In den Wald ging der Jäger und einen Hasen fing.\\
           in the.\ACC{} forest went the.\NOM{} hunter and a.\ACC{} hare caught\\
    }
  \end{xlist}
\end{exe}
%
At first sight, this looks like a run-of-the-mill symmetric \BAR{C}
coordination. However, this is not the case, since the PP in the
Vorfeld is unambiguously a directional PP, which is incompatible with
the verb in the second conjunct. Furthermore, what \textit{is} shared
between the two conjuncts is the subject in the Mittelfeld of the
first conjunct, which is not in a c-structural position that would
lead us to expect this possibility.

\citeauthor{frank:2006} models SGF coordination using an optional
annotation on the rule for symmetric CP coordinations, which shares the
(grammaticalized) discourse function of the first conjunct with the
coordination as a whole, and therefore, with the second
conjunct.\footnote{Here, too, the grammaticalized discourse functions
include the subject. In fact, in this construction, the shared
material will always turn out to be the subject.}
%
\begin{exe}
  \ex \phraserule{CP}{%
    \rulenode{CP\\$\uparrow\in\downarrow$\\$\big((\UP\,\textsc{gdf})=(\DOWN\,\textsc{gdf})\big)$}
    \rulenode{Coord\\\UP=\DOWN}
    \rulenode{CP\\$\uparrow\in\downarrow$}
  }
\end{exe}
%
This extra annotation makes sure the completeness requirements in the
second conjunct can be met. \citeauthor{frank:2006} also shows
that this approach makes correct predictions with respect to the
interpretation of the scope of quantified subjects in an SGF
coordination. However, the formal account leaves unexplained why the
second conjunct cannot have a fronted object, like the
ungrammatical~(\ref{ex:WGerm:47}b). For this, \citeauthor{frank:2006}
appeals to the discourse structure of SGF coordination: the second
conjunct is conceptualized as part of the discourse-functional domain
of the first. If the second conjunct were to have a Vorfeld \TOPIC or
\FOCUS, this would indicate that it sets up its own discourse-functional
domain.

\subsection{Topics related to the right periphery\label{sec:Germanic:rightperi}}

\subsubsection{Clefts}

The it-cleft construction in Dutch involves a neuter weak pronoun
(typically \textit{het} `it'), a copula, focused material, and a
backgrounded finite
clause in the Nachfeld. \Citet[Chapter~2]{vanderbeek:2005} shows that, with these
ingredients, there are in fact two distinct cleft constructions: the
intransitive cleft (an existential copula with extraposed complement
clause, \ref{ex:WGerm:49}a) and the transitive cleft (an identificational copula
with extraposed relative clause, \ref{ex:WGerm:49}b).
%
\begin{exe}
  \ex\label{ex:WGerm:49} Dutch \Citep[Figures~2.4 and~2.1]{vanderbeek:2005}
  \begin{xlist}
    \ex \gll Het is aan hem dat ze denkt.\\
             \textsc{expl}  is on him \COMPL{} she thinks.\\
        \glt `It is of him that she is thinking.'
    \ex \gll Het zijn jouw kinderen die huilen.\\
             it  are  your children \REL{} cry\\
         \glt `It is your children who are crying.'
  \end{xlist}
  \end{exe}
%
\Citeauthor{vanderbeek:2005} shows that the two cleft types differ
further in whether they involve an expletive or referential neuter
pronoun, whether they tolerate pseudo-copulas like \textit{bleken}
`seem', or only forms of the verb \textit{zijn} `be', and whether the
neuter pronoun is obligatorily the cleft subject or not.

\Citeauthor{vanderbeek:2005} models both cleft types with dedicated
c-structure rules to capture the fixed position of the backgrounded
clause, and to introduce the construc\-tion-specific annotations. This
ensures, for instance, that the clause must be in the Nachfeld, and cannot
be realized as one constituent with the pronoun or the focused
material. In the intransitive cleft, both the expletive and the complement
clause project to the \SUBJ{}'s f-structure, and the focused material is
linked to an adjunct or oblique position in f-structure (\textit{aan hem} `on him' in \ref{ex:WGerm:49}a). In the
transitive cleft, the relative clause is an adjunct of the referential
weak neuter pronoun, which is the construction's \TOPIC, and which is
subject or object depending on properties of the focused material. The
top level c-structure rules for the two constructions are given
in~(\ref{ex:WGerm:50}). 
%
\begin{exe}
  \ex\label{ex:WGerm:50}
  \begin{xlist}
    \ex intransitive cleft:\\
    \phraserule{IP}{
      \rulenode{NP\\(\UP\SUBJ)=\DOWN}
      \rulenode{CP\\%
        (\UP\SUBJ)=\DOWN\\%
        (\DOWN\textsc{type})\,=$_c$\,\textsc{that}\\%
        (\UP\FOCUS)=(\UP\XCOMP*\,\{\,\ADJ\,${\in}\mid$\,\OBLTHETA\})}
    }
    \ex transitive cleft:\\
    \phraserule{IP}{
      \rulenode{NP\\(\UP\SUBJ)=\DOWN}
      \rulenode{\BAR{I}\\\UP=\DOWN}
      \rulenode{CP\\%
        \DOWN$\in$\,(\UP\TOPIC\ADJ)\\
        (\UP\TOPIC\textsc{prontype})=\textsc{cop}\\
        (\DOWN\textsc{type})\,=$_c$\,\textsc{rel}\\
        (\UP\FOCUS)=(\UP\{\SUBJ|\OBJ\})}}
    
  \end{xlist}
\end{exe}
%
The \textsc{type} feature of the CP-projected f-structures
distinguishes relative clauses from complement clauses headed by
\textit{dat} `that'. The \textsc{prontype}=\textsc{cop} restriction
singles out a class of special copular pronouns which are \N.3\SG{}
in form, but which show a wider range of agreement, whose existence
can be argued for on independent grounds.

\subsubsection{Correlative \textit{es} and extraposed CPs}

The constructions discussed in \citet{berman-etal:1998:lfg} and
\citet[Chapter~8]{Berman2003} also contain a neuter pronoun and a
finite clause in the Nachfeld. In this case, the pronoun and the finite
clause realize a propositional argument of the clause's main verb, and
they can either appear on their own~(\ref{ex:WGerm:51}a,b) or
together~(\ref{ex:WGerm:51}c), in which case the pronoun is referred to as a
\textit{correlative pronoun}.
%
\begin{exe}
\ex\label{ex:WGerm:51} German \citep[§1, examples 1 and 2]{berman-etal:1998:lfg}
\begin{xlist}
  \ex {\gll
    Hans hat bedauert, da{\ss} er gelogen hat.\\
    Hans has regretted that he lied has\\
    \glt Hans regretted that he lied.} 
  \ex {\gll
    Hans hat es bedauert.\\
    Hans has it regretted\\
    \glt Hans regretted it.}
  \ex {\gll
    Hans hat es bedauert, da{\ss} er gelogen hat.\\
    Hans has it regretted that he lied has\\}
  \end{xlist}
 \end{exe}
%
The central modelling assumptions made in both analyses are that the
pronoun is referential -- whether it occurs on its own or as a
correlative together with the finite clause -- and that the pronoun
and the finite clause when they appear alone~(\ref{ex:WGerm:51}a,b), are
\OBJ{}s. In the correlative pronoun construction~(\ref{ex:WGerm:51}c), however,
it is the pronoun that has this grammatical function.  The finite
clause is then either seen as supplying further semantic restrictions
to the interpretation of this pronoun \citep[see also
  \sectref{sec:Germanic:splitnps} above]{berman-etal:1998:lfg} or as an
apposition to the pronoun \citep{Berman2003}. 

\citet{Berman2003} also goes on to show that there is a range of
correlative \textit{es} data, and that despite superficial
resemblances, different syntactic analyses are called for. For
instance, \citeauthor{Berman2003} argues that in contrast to the data
above, the psych verb \textit{stören} `disturb' in~(\ref{ex:WGerm:52}) has a
different argument structure for the cases with correlative
\textit{es}: it either takes the finite clause as subject when the
correlative is absent, or it takes \textit{es} as subject and the
finite clause as object when the correlative is present.
%
\begin{exe}
  \ex\label{ex:WGerm:52} German \citep[§8.1, example 2d]{Berman2003}\\
      {\gll
        \ldots{}weil (es) mich stört, dass sie den Hans liebt.\\
        \phantom{\ldots}because \phantom{(}\textsc{expl} me bothers \COMP{} she the.\ACC{} Hans loves.\\
        \glt `\ldots because it bothers me that she loves Hans.'}
\end{exe}


\subsection{Topics related to the ordering of dependents\label{sec:Germanic:ordering}}

\subsubsection{Scrambling\label{sec:Germanic:scrambling}}

Material in the Mittelfeld can be reordered to a certain extent. For
instance, Dutch allows different orders of object and
adverb~(\ref{ex:WGerm:53}). 
%
\begin{exe}
  \ex\label{ex:WGerm:53} Dutch\\
    \gll Anna heeft \{de was\} gisteren \{de was\} gedaan.\\
    Anna has \phantom{\{}the laundry yesterday  \phantom{\{}the laundry done\\
    \glt `Anna did the laundry yesterday.'
\end{exe}
%
In German, the order of arguments themselves is free, as
well. Example~(\ref{ex:WGerm:54}) shows one order for the arguments of a
ditransitive, but the other five possible argument orders are
grammatical, too.
%
\begin{exe}
    \ex\label{ex:WGerm:54} German \citep[§1, example~1]{haider-rosengren:2003}\\
    \gll \ldots{}dass $[$das Objekt$]$ $[$dem Subjekt] $[$den ersten Platz] {streitig macht}.\\
             \phantom{\ldots}\COMP{} \phantom{$[$}the.\NOM{} object \phantom{$[$}the.\DAT{} subject \phantom{$[$}the.\ACC{} first place competes.for \\
        \glt `\ldots that the object competes for first position with the subject.'
\end{exe}
%       
In general, both scrambling over adjuncts and scrambling of arguments
is sensitive to information structural effects, and -- related to this
-- things like the referential form of the material involved.
\citeauthor{Choi1999}~(\citeyear{Choi1999,Choi2001}) explains German scrambling and
clause-local fronting facts using an OT-LFG model in which constraints
on canonical ordering of grammatical functions conflict with
constraints on information structurally induced ordering. An
information structural account of clause-local word order variation
and quantifier scope in German is given in
\citet{CookPayne}. The explanation given by
\citet{cook:2006:benjamins} for a deviating unmarked
word order in a small group of ditransitives is discussed below,
in~\sectref{sec:Germanic:mappingvariation}.

As far as the order of arguments in the Mittelfeld is concerned, Dutch
is much more restricted than German. Nevertheless there is some
variation. An OT account of the Dutch dative alternation, which also
covers variation in the ordering of direct object and indirect object, is
presented in \Citet{vanderbeek:2005}. \citet{zaenen:1989:techreport}
discusses scrambling of objects over subjects with Dutch experiencer
verbs and passives of ditransitives, and argues for an effect of
thematic role.

The cross-serial dependency pattern comes about
when objects and verbs are in separate groups and both are ordered in
the same fashion according to level of embedding. The verb cluster
rule given in~(\ref{ex:zk95:vprules}) above \citep{zaenen-kaplan1995},
sorts embedding verbs before embedded ones, and explicitly forces the
same order on the objects with the help of an f-precedence
constraint. In her work on the order of objects in Dutch,
\Citet[§3.8]{vanderbeek:2005} argues that this constraint should be
treated as a violable OT constraint. An optionally higher ranking constraint prefers early realization
in the Mittelfeld of a third person, inanimate pronoun. This constraint
explains examples like~(\ref{ex:WGerm:55}), in which the object pronoun
belonging to the embedded verb precedes the object belonging to the
finite verb.
%
\begin{exe}
  \ex\label{ex:WGerm:55} \gll Ik zag$_{\text{\OBJ:1}}$ 't$_2$ Jo$_1$ doen$_{\text{\OBJ:2}}$.\\
  I saw it Jo do\\
  \glt `I saw Jo do it.'
\end{exe}
%
As with all scrambling, this type of scrambling is less constrained in
German, and may also apply to full NPs, and even involve scrambling of
an embedded object over the main clause subject. I am however not aware of
any LFG-related work on this.\footnote{The term \textit{embedded
  object shift} is \citeauthor{vanderbeek:2005}'s term for this type
of word order variation. In the literature on German, the phenomenon
is sometimes discussed as a kind of \textit{long(-distance)
  scrambling}, that is, scrambling across clause boundaries, although
the view that the embedded object leaves its clause goes against the
conception of coherent combining as clause union. In fact, in
\citet{lee-schoenfeld:2007:benjamins}, this type of scrambling is
taken as one of the hallmarks of coherence and thus of monoclausality.}
 

\subsubsection{Weak cross-over\label{sec:Germanic:weakxover}}

In German, scrambling and topicalization interact with binding between
arguments, which results in data like~(\ref{ex:weakxover}). Note that the
grammaticality judgements are relative to the co-indexations given in
the examples.
%
\begin{exe}
  \ex \label{ex:weakxover}German (\citealp{Berman2003}, §5.2, examples 10a, 11b, 10d, 11d, 27, 31{;} examples a--d below originally from \citealp{choi:1995:lsa})
  \begin{xlist}
    \ex[]{%
      \gll \ldots{}dass jeden$_i$ seine$_i$ Mutter mag.\\
            \phantom{\ldots}\COMP{} everyone.\ACC{} his mother likes\\
      \glt `\ldots that their$_i$ mother likes everyone$_i$.'\footnote{English seems to require the passive to achieve the intended bindings. The intended reading in~(\ref{ex:weakxover}a--d) is therefore more naturally given as \textit{(that) everyone is liked by their own mother}.}
    }      
    \ex[]{\gll
      Jeden$_i$ mag seine$_i$ Mutter.\\
      everyone.\ACC{} likes his mother\\
      \glt `Their$_i$ mother likes everyone$_i$.'}
    \ex[*]{\gll
      \ldots{}dass seine$_i$ Mutter jeden$_i$ mag.\\
      \phantom{\ldots}\COMP{} his mother everyone.\ACC{} likes\\}
    \ex[*]{\gll
      Seine$_i$ Mutter mag jeden$_i$.\\
      his mother likes everyone.\ACC{}\\}
    \ex[]{%
      \gll Jeden$_i$ sagte sie, habe seine$_i$ Mutter getröstet.\\
           everyone.\ACC{} said she has.\SBJV{} his mother consoled\\
      \glt `Everyone$_i$, she said their$_i$ mother had consoled.'
    }
    \ex[*]{%
      \gll Jeden$_i$ sagte seine$_i$ Mutter, habe sie getröstet.\\
           everyone.\ACC{} said his mother has.\SBJV{} she consoled\\
      \glt `Everyone$_i$, their$_i$ mother said she had consoled.'
     }
  \end{xlist}
\end{exe}
%
Between dependents of the same predicate, an object may bind into the
subject, provided it precedes it. It does not matter whether it
precedes it in the Mittelfeld~(\ref{ex:weakxover}a) or by being moved into the
Vorfeld~(\ref{ex:weakxover}b), even from an embedded clause~(\ref{ex:weakxover}e). However,
as~(\ref{ex:weakxover}f) shows, an object cannot bind into an upstairs subject,
even when it precedes it.

\citet{Berman2003}, using the framework of \citet{Bresnan1998} and
observations from \citet{choi:1995:lsa}, shows that the data
in~(\ref{ex:weakxover}a--d) is straightforwardly explained by assuming that to
bind a pronoun, an operator must either outrank it in terms of
grammatical function -- this isn't the case in any of these examples
-- or linearly precede it. The linear precedence constraint is
satisfied in~(\ref{ex:weakxover}a,b), but not in~(\ref{ex:weakxover}c,d). However,
example~(\ref{ex:weakxover}f) is problematic under this simple account, since the
operator precedes the pronoun, but cannot bind it.

\hspace*{-3.8pt}\citeauthor{Berman2003} therefore proposes to analyse long-distance
dependencies using a trace, and to interpret the linear precedence
requirement as if it includes this trace. The sentences in~(\ref{ex:weakxover}e,f)
are then as in~(\ref{ex:wxotrace}).\footnote{\citet{Berman2003} assumes that
local arguments are adjoined to VP, in any order. This also applies to
traces -- the object trace may therefore appear before its clause-mate
subject. In the examples in~(\ref{ex:wxotrace}) we have inserted the
trace as early as c-structurally possible.}
%
\begin{exe}
  \ex\label{ex:wxotrace}
  \begin{xlist}
    \ex[]{%
      \gll Jeden$_i$ sagte sie, habe ~$\epsilon_i$~ seine$_i$ Mutter getröstet.\\
      everyone.\ACC{} said she has.\SBJV{} {} his.\NOM{} mother consoled\\
      }
    \ex[*]{%
      \gll Jeden$_i$ sagte seine$_i$ Mutter, habe ~$\epsilon_i$~ sie getröstet.\\
      everyone.\ACC{} said his.\NOM{} mother has.\SBJV{} {} she consoled\\
    }
  \end{xlist}
\end{exe}
%
In~(\ref{ex:wxotrace}a), the operator's trace precedes the bound pronoun, so
that the linear order requirement is met. In~(\ref{ex:wxotrace}b), however, the trace follows the pronoun, which -- under \citeauthor{Berman2003}'s definition -- means the operator as a whole does not precede it
This results in the unavailability of the indexed reading.

\citet[§9.5]{BresnanEtAl2016} discuss the same data using a near-identical framework. Although the difference in linear order of the
bound pronoun and the operator trace between~(\ref{ex:wxotrace}a) and~(\ref{ex:wxotrace}b)
is noted, the ungrammaticality of~(\ref{ex:wxotrace}b) is ultimately explained
by taking the binding domain of the operator to be the f-structure for
the predicate \textit{getröstet} `consoled', irrespective of the
operator's \DF role in the matrix f-structure.\footnote{In contrast,
\citet[§5.2.6]{Berman2003} explicitly considers the binding domain of
the operator to be ``extended to the matrix clause'' because ``it
functions as a discourse function in the matrix clause'' (p.~86).} There
is therefore no need to refer to the position of the trace to explain
the long-distance dependency data. Under that analysis, it would
appear that weak cross-over in German alone is not a reason to assume
long-distance dependencies involve traces.

\citet{dalrympleetal2001}\footnote{This paper is a response to
the trace-based proposals of
\citet{Bresnan1998} and \citet{Berman2003}. The latter was also
published/circulated on earlier occasions, which explains the apparent
anachronism.} give a trace-less account of the German cross-over
data. Rather than considering the linear order of the binding operator
and the bound pronoun, they consider f-precedence between two
f-structures that are dependents of the same predicate, such that one
contains the operator and the other the pronoun. In~(\ref{ex:weakxover}f), these
f-structure siblings are the \SUBJ (containing the pronoun) and the
\COMP (containing the operator) of \textit{sagte}. Since the latter
does not f-precede the former, the linear precedence requirement on
binding is not met.

\subsection{Topics related to mapping\label{sec:Germanic:mapping}}

\subsubsection{Sentences ``without a subject'' in German\label{sec:Germanic:subjectless}}

A recurring debate in German clausal syntax concerns the existence of
true subjectless sentences. \citet[Chapter~4]{Berman2003} points out
that it would appear that German has such sentences, given that 1)
under her analysis, German does not have a dedicated subject position,
2) there are no oblique subjects in German (a common view, following
for instance \citealp{ZMT85:Case}, but contra the later
\citealp{barthdal05}) and 3) there are sentences without
nominatives, such as~(\ref{ex:WGerm:58}). 
%
\begin{exe}
  \ex\label{ex:WGerm:58} German \citep[§4.2, examples 10a, 16b, 10d{;} indication of optionality of expletive mine]{Berman2003}
  \begin{xlist}
    \ex \gll \ldots{}weil (*es) getanzt wurde.\\
             \phantom{\ldots{}}because \phantom{(*}\textsc{expl} danced was\\
        \glt `\ldots{}because people were dancing.'
    \ex \gll \ldots{}weil (*es) dem Mädchen geholfen wurde.\\
             \phantom{\ldots{}}because \phantom{(*}\textsc{expl} the.\DAT{} girl helped was\\
        \glt `\ldots{}because the girl was being helped.'
    \ex \gll \ldots{}weil (es) mich friert.\\
        \phantom{\ldots{}}because \phantom{(}\textsc{expl} me.\ACC{} freezes\\
        \glt `\ldots{}because I'm cold.'
  \end{xlist}
\end{exe}
%
Note that each of these \textit{can} occur without the expletive pronoun
\textit{es}, and the first two \textit{must} occur without it.

\citeauthor{Berman2003} models clauses without a subject using
argument structures without a \SUBJ, and shows that these cases can be given
an analysis in terms of Lexical Mapping Theory (LMT).\footnote{See
\citetv{chapters/Mapping} for more information
on Lexical Mapping Theory.} For the
predicates involved in the examples above, we have the following LMT
derivations:
%
\begin{exe}
  \ex
  \begin{xlist}
    \ex getanzt \arglist{%
      \begin{tabular}[t]{@{}c@{}}{\rm\textit{agent}}\\$[-o]$\\$\emptyset$\end{tabular}%
    } \hfill (lit.\ `danced', impersonal passive)
    \ex geholfen \arglist{%
      \begin{tabular}[t]{@{}c@{}}{\rm\textit{agent}}\\$[-o]$\\$\emptyset$\end{tabular}, %
      \begin{tabular}[t]{@{}c@{}}{\rm\textit{beneficiary}}\\$[+o]$/\DAT\\\OBJTHETA\end{tabular}%
    } \hfill (`helped', passive, lexical case)
    \ex frieren \arglist{%
      \begin{tabular}[t]{@{}c@{}}{\rm\textit{experiencer}}\\$[+o]$/\ACC\\\OBJTHETA\end{tabular}%
    } \hfill (`be cold', active intransitive, lexical case)
  \end{xlist}
\end{exe}
%
The question remains, then, why the expletive is not allowed
in~(\ref{ex:WGerm:58}a,b), whereas it is in
(\ref{ex:WGerm:58}c). \citeauthor{Berman2003} adopts the analysis that
German verbal agreement morphology is distinct enough to contribute
subject features. Thus, the f-structures for the sentences
in~(\ref{ex:WGerm:58}) all contain subjects. This way, German can be analysed
as meeting the \textit{Subject Condition}, which says that every
f-structure with a predicate must contain a \SUBJ.\footnote{This
Subject Condition formulation pertains to f-structure. In other
contexts, for instance in \citet[§14.4]{BresnanEtAl2016}, the
Subject Condition is taken to be a constraint on argument
structures. It is clear that under Berman's view such a constraint
does not hold for German.} Inserting a subject expletive would then be
ruled out as a violation of Economy of Expression. It follows that
the optional \textit{es} in cases like (\ref{ex:WGerm:58}c) is selected for:
verbs like \textit{frieren} have an alternative specification like the
one in~(\ref{ex:WGerm:60}).
%
\begin{exe}
  \ex\label{ex:WGerm:60} \lexentry{frieren}{%
    (\UP\PRED) = `{\rm be-cold}\arglist{\OBJTHETA}\SUBJ'\\
    (\UP\SUBJ\textsc{form}) =$_c$ \textsc{es\_}}
\end{exe}


\subsubsection{Mapping explanations of variation\label{sec:Germanic:mappingvariation}}
\citet{zaenen93} is concerned with (the nature of) the
unaccusative/unergative distinction in Dutch.  One of the challenges
in the characterization of unaccusativity in Dutch is that it not only
applies to intransitives, but also to a subset of transitive
experiencer verbs. Consider the examples in~(\ref{ex:WGerm:61}), which shows two
intransitives, two transitives with the experiencer as the object, and
a transitive with the experiencer as the subject. The selection of a
form of \textit{zijn} `be' instead of \textit{hebben} `have' as the
perfect auxiliary is given here as the reflex of unaccusativity.
%
\begin{exe}
  \ex\label{ex:WGerm:61} Dutch
  \begin{xlist}
    \ex \gll Zij *is / heeft gewerkt.\\
             she \phantom{*}is {} has worked\\
    \ex \gll Zij is / *heeft gestorven.\\
             she is {} \phantom{*}has died\\
    \ex \gll Zij *zijn / hebben haar geirriteerd.\\
             they \phantom{*}are {} have her irritated\\
    \ex \gll Zij zijn / *hebben haar bevallen.\\
             they are {} \phantom{*}have her pleased\\
    \ex \gll Zij *is / heeft hen gevreesd.\\
             She \phantom{*}is {} has them feared.\\
  \end{xlist}
\end{exe}
%
\citeauthor{zaenen93} shows that it is possible to give
semantic correlates of unaccusativity, and discusses which phenomena
can be related directly to unaccusativity in Dutch (namely, auxiliary
selection, prenominal attributive use of perfect participle) and which only relate
indirectly (impossibility of impersonal passive). Her analysis is
formalized in terms of a variant of LMT that does not rely on thematic
roles to determine the intrinsic classifications of a predicate's
arguments. Instead, \citeauthor{zaenen93} incorporates
\citeauthor{Dowty1991}'s~(\citeyear{Dowty1991}) proto-roles into LMT
using the following simple rule: a participant that has more
proto-agent than proto-patient properties is marked $[-o]$, otherwise
the participant is marked $[-r]$. The LMT alternative is further
spelled out to allow derivation of grammatical function assignments
for the data in~(\ref{ex:WGerm:61}): the subjects in examples
(\ref{ex:WGerm:61}a,c,e), with \textit{hebben} `have', come from intrinsic
$[-o]$ markings, whereas the subjects in~(\ref{ex:WGerm:61}b,d), with
\textit{zijn} `be', come from intrinsic $[-r]$. The choice of
auxiliary can be correctly modelled by referring to the intrinsic
markings of the subject. \citet{Kordoni2003} discusses analysing the German locative
alternation in terms of \citeauthor{zaenen93}'s mapping account.\footnote{It should be noted
that \citet{Dowty1991} talks about the \textit{English} locative alternation
in terms of proto-roles in depth.}

Another variation which is shown to be driven by lexical semantic
differences that affect mapping are the so-called ``high'' versus ``low''
datives in German. Although arguments in the German Mittelfeld are
readily scrambled, there is an unmarked order, which can be detected
by studying information structural and quantifier scoping
properties. Between objects, the unmarked order is generally \DAT
before \ACC~(\ref{ex:WGerm:62}; ``high dative''). However, a smaller number of
verbs show \ACC before \DAT~(\ref{ex:WGerm:63}; ``low dative''), and for a
couple of verbs both orders appear to be unmarked. In the examples,
superscript {\scriptsize{}M} marks the marked variant.
%
\begin{exe}
  \ex\label{ex:WGerm:62} German \citep[§1, examples~1--2]{cook:2006:benjamins}
  \begin{xlist}
    \ex[]{%
      \gll Es hat ein Mann $[$einem Kind$]$ $[$ein Buch$]$ geschenkt.\\
           \textsc{expl} has a.\NOM{} man \phantom{$[$}a.\DAT{} child  \phantom{$[$}a.\ACC{} book given\\
      \glt `A man gave a book to a child (as a present).'
    }
    \ex[$^{\mbox{\scriptsize{}M}}$]{Es hat ein Mann $[$ein Buch$]$ $[$einem Kind$]$ geschenkt.}
  \end{xlist}
  \ex\label{ex:WGerm:63}
  \begin{xlist}
    \ex[$^{\mbox{\scriptsize{}M}}$]{%
      \gll Es hat ein Polizist $[$einer Gefahr$]$ $[$einen Zeugen$]$ ausgesetzt.\\
           \textsc{expl} has a.\NOM{} policeman \phantom{$[$}a.\DAT{} danger \phantom{$[$}a.\ACC{} witness exposed\\
      \glt `A policeman has exposed a witness to a danger.'
    }
    \ex[]{Es hat ein Polizist  $[$einen Zeugen$]$ $[$einer Gefahr$]$ ausgesetzt.}
  \end{xlist}
\end{exe}
%
\citet{cook:2006:benjamins} demonstrates that the different unmarked
orders can be related to differences in lexical semantics, which in
turn give rise to thematic alternations. For the alternating verbs, it
is shown that the different word orders prefer different readings in
line with the general lexical semantic observations. All
meanings/word orders involve an agent and a patient/theme, which under
standard LMT assumptions are mapped to \SUBJ and (accusative) \OBJ,
respectively. In addition, the \DAT-\ACC order is associated with a
bene-/maleficiary role, which is mapped to a (dative) \OBJTHETA. The
\ACC-\DAT order, however, involves a third participant which is a goal
or a location and which gets mapped to a (dative)
\OBLTHETA. \citeauthor{cook:2006:benjamins} argues that the unmarked
order of complements in the German Mittelfeld is
\OBJTHETA-\OBJ-\OBLTHETA. The apparent word order variation is thus a
fixed word order seen in the light of the unmarked order of
grammatical functions. \citeauthor{cook:2006:benjamins} extends her
account to explain the compatibility of the different datives with the
\textit{kriegen}-passive, which can be used with a selection of verbs
to promote the dative argument to subject.

\subsubsection{Transitivity of reflexives}

Lexically conditioned reflexives in German and Dutch show up in a
range of situations. The simplex reflexives \textit{sich} in German
and \textit{zich} in Dutch appear for instance in clauses with transitive verbs
with co-referring arguments~(\ref{ex:WGerm:64}a)/(\ref{ex:WGerm:65}a),\footnote{The class
of grooming verbs is part of a larger class of transitive verbs that,
exceptionally, allows the simplex reflexive. In
general, the complex reflexive, \textit{zichzelf} /
\textit{sich selbst} is available to realize reflexive objects with
transitive verbs. This exception is what justifies treating these
reflexives as being lexically specified.} in
anticausatives~(\ref{ex:WGerm:64}b)/(\ref{ex:WGerm:65}b), and in inherent
reflexives~(\ref{ex:WGerm:64}c)/(\ref{ex:WGerm:65}c).
%
\begin{exe}
  \ex\label{ex:WGerm:64}
  \begin{xlist}
    \ex\label{ex:Germ:64a} \gll Max rasiert sich.\\
             Max shaves \REFL{}\\ 
         \glt `Max shaves himself.'
    \ex \gll Die Tür öffnet sich.\\
             the door opens \REFL{}\\
        \glt `The door opens.'
    \ex \gll Max schämt sich.\\
             Max is.ashamed \REFL{}\\
        \glt `Max is ashamed.'
  \end{xlist}
  \ex\label{ex:WGerm:65}
  \begin{xlist}
    \ex\label{ex:Germ:65a} \gll Max scheert zich.\\
        Max shaves \REFL{}\\ 
    \ex \gll De deur opent zich.\\
    the door opens \REFL{}\\
    \ex \gll Max schaamt zich.\\
    Max is.ashamed \REFL{}\\
  \end{xlist}
\end{exe}
%
\largerpage
In a contrastive study of reflexivization, \citet{Sells1987-xz}
distinguish three kinds of transitivity: 1) c-structure transitivity
-- the reflexive is an independent constituent, 2) f-structure
transitivity -- the syntactic predicate selects an \OBJ, 3) semantic
transititivity -- the referential identity of the arguments is
accidental.  Interestingly, German and Dutch simplex reflexives receive
different analyses: they are both considered to be transitive in terms
of c-structure, and intransitive in terms of semantics, but
\citeauthor{Sells1987-xz} analyse the German reflexives as f-structurally
intransitive, and the Dutch reflexives as f-structurally transitive. This
is based upon the contrast in~(\ref{ex:WGerm:66}): the German reflexive can
appear in an impersonal passive, whereas the non-reflexive counterpart
is ruled out. The reflexive thus patterns with intransitives. The
Dutch counterpart is not well-formed, which would suggest Dutch
reflexives pattern with transitives.
%
\begin{exe}
  \ex\label{ex:WGerm:66}
  \begin{xlist}
    \ex[]{German \citep[§2.4, example~74]{Sells1987-xz}\\\gll Jetzt wird \{sich / *ihn\} aber gewaschen!\\
    now   is  \phantom{\{}\REFL{} {} \phantom{*}him however washed\\
    \glt `But now it is time to wash yourself!'\\
    Not: `\ldots to wash him!'}
    \ex[{{\raisebox{-2.8125ex}[0pt][0pt]{*}}}]{Dutch\\ \gll Nu wordt (er) \{zich / hem\} gewassen!\\
                  now is \phantom{(}\textsc{expl} \phantom{\{}\REFL{} {} him washed\\
    }
  \end{xlist}
\end{exe}
%
\citet{Sells1987-xz} model reflexives like~(\ref{ex:Germ:64a}),~(\ref{ex:Germ:65a}) with a lexical rule which maps a
transitive verb (in all the three senses above) to a reflexive
verb. For German, this involves leaving out the object slots, and 
marking the predicates with `[a] feature F, forcing them to combine
with the reflexive element.'\footnote{The reflexive itself carries a
constraining equation checking this feature F to make sure it is only
combined with predicates that have undergone reflexivization. It is
not spelled out in the article how the presence of the reflexive would
be enforced technically, however. One solution is to let
the verb and the reflexive be co-heads that check for the presence of
each other using constraining equations.} The resulting intransitive
can then serve as input for the lexical rule for the impersonal
passive. For Dutch, the reflexivization rule involves moving the
thematic object to a non-thematic object slot, marked to be filled by
a simplex reflexive.

The analysis of \citeauthor{Sells1987-xz} crucially relies on the use of
lexical rules to take care of mapping. Modern LFG work would
rely on a variant of LMT. Data like those in~(\ref{ex:WGerm:66}) then
also receive a different status, as LMT does not model the
(impersonal) passive as a rule to be applied on the output of another
rule. \citet{AlencarKelling2005} propose an analysis of the whole
range of data in~(\ref{ex:WGerm:64}a--c) in terms of LMT, and explicitly
reject the importance of the contrast in~(\ref{ex:WGerm:66}). Instead, they offer
additional data to support the conclusion that the German reflexive \textit{is}
transitive at f-structure. Their LMT analysis of the data in~(\ref{ex:WGerm:64}), above,
is summarized in~(\ref{ex:WGerm:67}).
%
\begin{exe}
  \ex\label{ex:WGerm:67}
  \begin{xlist}
    \ex rasieren/öffnen
    \arglist{%
      \begin{tabular}[t]{@{}c@{}}{\rm\textit{agent}}\\$[-o]$\\\SUBJ\end{tabular}, %
      \begin{tabular}[t]{@{}c@{}}{\rm\textit{theme}}\\$[-r]$\\\OBJ\end{tabular}%
    } \hfill (`shave'/`open', transitive)
    \ex (sich) rasieren
    \arglist{%
      \begin{tabular}[t]{@{}c@{}}{\rm\textit{agent$_i$}}\\$[-o]$\\\SUBJ{}$_i$\end{tabular}, %
      \begin{tabular}[t]{@{}c@{}}{\rm\textit{theme$_i$}}\\$[-r]$\\\OBJ{}$_i$\rlap{$[$\REFL{}$]$}\end{tabular}%
    } \hfill (`shave', reflexive of transitive)
    \ex (sich) öffnen \arglist{%
      \begin{tabular}[t]{@{}c@{}}{\rm\textit{theme$_i$}}\\$[-r]$\\\SUBJ{}$_i$\end{tabular}%
    }%
      \begin{tabular}[t]{@{}c@{}}{\_$_i$}\\$[-r]$\\\OBJ{}$_i$\rlap{$[$\REFL{}$]$}\end{tabular}%
      \hfill (`open', anticausative)
    \ex (sich) schämen \arglist{%
      \begin{tabular}[t]{@{}c@{}}{\rm\textit{theme$_i$}}\\$[-r]$\\\SUBJ{}$_i$\end{tabular}%
    }%
      \begin{tabular}[t]{@{}c@{}}{\_$_i$}\\$[-r]$\\\OBJ{}$_i$\rlap{$[$\REFL{}$]$}\end{tabular}
    \hfill (`be ashamed', inherent reflexive)
  \end{xlist}
\end{exe}
%
In the last two cases, the reflexive is an expletive.\footnote{The notation with indices to indicate reflexivity is taken from the paper. Note that, as they are also on expletives, these indices should not be interpreted as semantic co-reference.}

\section{LFG analyses in the nominal domain\label{sec:Germanic:nom}}

\subsection{Overall shape of nominal constituents}

The CWG nominal domain has received a lot less attention than the
clausal domain in LFG. The authors that have concerned themselves with
the nominal domain in more detail all assume a DP analysis
(\citealp{berman-frank:1996}, Part~I, Chapter~3; \citealp{dipper03},
Chapter~7; \citeauthor{strunk04}, \citeyear{strunk04,Strunk05}). The
general shape of the nominal constituent is characterized by the
familiar representation in~(\ref{ex:WGermanic:68}).
%
\begin{exe}
  \ex\label{ex:WGermanic:68}\begin{forest}
           [DP
             [\hspace*{1em}\ldots\hspace*{1em}]
             [\rulenode{\UP=\DOWN\\\BAR{D}}
               [\rulenode{\UP=\DOWN\\D}]
               [\rulenode{\UP=\DOWN\\NP}
                 [\hspace*{1em}\ldots\hspace*{1em}]
                 [\rulenode{\UP=\DOWN\\\BAR{N}}
                   [\rulenode{\UP=\DOWN\\N}]
                   [\hspace*{1em}\ldots\hspace*{1em}]
                 ]
               ]
             ]
           ]
  \end{forest}
\end{exe}
%
In Spec-DP, elements like pre-determiners (Dutch: \textit{\textbf{al} die mensen} `all those people'), prenominal genitives (that
is, non-pronominal possessives; German: \textit{\textbf{Karls} Auto} `Karl's car'), and non-genitive prenominal possessors (Low Saxon: \textit{\textbf{de'n Jung} sien Vadder}, lit.\ `the boy his father') can occur. We refer to the discussion of possessives
in~\sectref{sec:Germanic:posslowsax} below for more elaborate examples. D holds determiners
and pronouns, although \citet{dipper03} assigns pronouns
proper (in contrast to determiner-like pronouns) to a category Pron,
which is the single daughter of \BAR{D}.

The NP contains mostly lexical material. In \citet[Chapter~7, the
  theoretical discussion]{dipper03}, the class of adjectival
quantifiers (see \sectref{sec:Germanic:dquant_aquant}, below) appears in Spec-NP,
whereas other attributive adjectives appear as sisters to
N. \citet{berman-frank:1996}, however, assume that attributive
adjectives are left-adjoined to NP (not shown in the schematic tree),
whereas oblique and clausal complements are in Comp-NP, on the right.

This overall picture is slightly different in \citet[Chapter~8, the
  computational implementation]{dipper03} and  \citet{Dipper05}, which describe a flat DP/\BAR{D}, under which
predeterminers, determiners, prenominal genitives and adjectival
quantifiers (that is, the material in Spec-DP, D, and Spec-NP
in~\ref{ex:WGermanic:68} above) all appear as sisters of NP.\footnote{This
analysis has the explicit goal of ``serv[ing] as the base of a robust
and efficient implementation'' \citep[101]{Dipper05}, but its
status as a theoretical claim remains a bit unclear. The question of
whether a deeper/different analysis would have been preferred in a
more theoretically oriented analysis and whether this flat structure
should mostly be seen as an operationalization of a deeper structure,
is unfortunately not discussed.}

\subsection{Topics in noun phrase syntax}

\subsubsection{Possessives in Low Saxon\label{sec:Germanic:posslowsax}}

Low Saxon has, amongst others, the range of possessive constructions illustrated in~(\ref{ex:WGermanic:69}--\ref{ex:WGermanic:71}).
%
\begin{exe}
  \ex\label{ex:WGermanic:69} Low Saxon \citep[§2.2, examples~2.52 and~2.84]{strunk04}
    \begin{xlist}
      \ex \gll sienen Weg\\
               his.\M.\SG.\ACC{} way\\
          \glt `his way'
      \ex \gll jeedeen Oort kreeg $[$sienen$]$, [\ldots].\\
                every kind got \phantom{$[$}his.\M.\SG.\ACC{}\\
           \glt `Every kind got its own, [\ldots].'
    \end{xlist}
  \ex\label{ex:WGermanic:70} Low Saxon \citep[§2.3, examples~2.112 and~2.177]{strunk04}
    \begin{xlist}
      \ex \gll de'n Jung sien Vadder\\
               the.\M.\SG.\ACC{} boy his.\M.\SG.\NOM{} father\\
      \glt `the boy's father'
    \ex \gll Korl sien\\
             Korl his.\N.\SG.\NOM{}\\
             \glt `Korl's'
               \end{xlist}
  \ex\label{ex:WGermanic:71} Low Saxon \citep[§6, example~61]{Strunk05}\label{ex:lowsaxonprenomgen}\\
    \gll $[$Hinnerk=s Huss$]$ iss groote den $[$Antje=s$]$.\\
             \phantom{$[$}Hinnerk=\POSS{} house is bigger than \phantom{$[$}Antje=\POSS{}\\
        \glt `Hinnerk's house is bigger than Antje's.'
\end{exe}
%
Example~(\ref{ex:WGermanic:69}a) contains a possessive pronoun in combination with
a noun expressing the possessum. The possessor is anaphorically given
as the referent of the possessive pronoun. Example~(\ref{ex:WGermanic:70}a) is a
case of \textit{possessive doubling}: as before, we have a possessive
pronoun and a noun, but now the possessive pronoun is directly
preceded by a nominal in the accusative, which explicitly supplies the
possessor. Finally, the first possessive expression in~(\ref{ex:WGermanic:71}) is
an instance of an \textit{-s} marked nominal realizing the possessor,
followed by the unmarked possessum. Examples~(\ref{ex:WGermanic:69}b),~(\ref{ex:WGermanic:70}b),
and the second possessive in~(\ref{ex:WGermanic:71}) show that the possessum can be
elided in each of these constructions.

\citet{Strunk05}\footnote{\citet{strunk04} is an
earlier version of this work, which contains a wealth of material on
Low Saxon possessives.} models the three constructions in a unified
way, crucially relying on optionally specified \PRED~`pro' values to
capture the differing amounts of explicitly realized referential
information. He assumes entries for possessive pronouns along the
lines of~(\ref{ex:WGermanic:72}a) and the possessive clitic in~(\ref{ex:WGermanic:72}b), as well as the
top level rule for the DP in~(\ref{ex:WGermanic:72}c).
%
\begin{exe}
  \ex\label{ex:WGermanic:72}
  \begin{xlist}
    \ex \catlexentry{sien}{D}{%
      $\big($(\UP\PRED) = \mbox{\rm`pro-of\arglist{\POSS}'}$\big)$\\
      (\UP\AGR) = \M.\SG\\
      (\UP\CASE) = \NOM\\
      $\big($(\UP\POSS\PRED) = \mbox{\rm`pro'}$\big)$\\
      (\UP\POSS\,\textsc{marking}) = {+}\\
      (\UP\POSS\AGR) = \M.3\SG\\
      (\UP\POSS\CASE) = \ACC\\
    }
    \ex \catlexentry{=s}{D}{%
      $\big($(\UP\PRED)= \mbox{\rm`pro-of\arglist{\POSS}'}$\big)$\\
      (\UP\POSS\,\textsc{marking}) = {+}
    }
    \ex \phraserule{DP}{%
      \rulenode{DP\\(\UP\POSS)=\DOWN\\
        (\UP\POSS\,\textsc{marking}) =$_c$ {+}}
      \rulenode{\BAR{D}\\\UP=\DOWN}
    }
  \end{xlist}
\end{exe}
%
The entry for a possessive pronoun constrains two ``regions'', the
first constraining the f-structure \UP for the DP it heads -- the
possessum -- and the second constraining the f-structure (\UP\POSS)
for the possessor. Both regions have agreement constraints, \UP from
inflectional morphology, (\UP\POSS) from the choice of the pronominal
root. The two regions also each have an optional equation defining
\PRED to be a pro-form. The four ways to satisfy these constraints
correspond to the four cases in~(\ref{ex:WGermanic:69}) and~(\ref{ex:WGermanic:70}). Finally,
\citet{Strunk05} analyses possessive \textit{-s} as a clitic which
also sits in D. Like the possessive pronoun this clitic can be
realized with or without a possessum in NP. Unlike the pronoun it must
be preceded by a DP that supplies a possessor, which explains the
absence of an optional \mbox{(\UP\POSS\PRED)=\mbox{`pro'}} in this
entry.

\hspace*{-2.5pt}\citet{berman-frank:1996} and \citet{dipper03} discuss the standard
German prenominal genitive possessive construction, exemplified
in~(\ref{ex:WGermanic:73}a). In contrast to the clitic-in-D analysis given above for
the (perhaps only superficially) similar Low Saxon possessive
\textit{-s}, these authors put the prenominal material completely in
Spec-DP.  \citet{berman-frank:1996} also treat the colloquial German
possessive doubling construction, which involves a prenominal
dative,\footnote{\citet[58--59]{kasper:2014} calls the prenominal dative possessive a
``non-standard German [construction] that is completely absent from
the standard but can be found in almost all regional
varieties/ dialects''. \citet{berman-frank:1996} discuss the
prenominal dative together with the prenominal genitive, and note that
the former ``allerdings eher in der gesprochenen Sprache auftritt''
[is however more likely to occur in spoken language] (p.~59).} found
in~(\ref{ex:WGermanic:73}b). As in the analysis proposed for the Lower Saxon
counterpart above, the (now dative-marked) possessor is located in
Spec-DP, and the possessive pronoun in D.
%
\begin{exe}
  \ex\label{ex:WGermanic:73} German \citep[§3.1.2, example 136, 141]{berman-frank:1996}
  \begin{xlist}
    \ex {\gll
      Peters (*das) Haus\\
      Peter's \phantom{(*}the house\\
      \glt `Peter's house'}
    \ex {\gll
      der Frau *(ihr) Haus\\
      the.\F.\SG.\ACC{} woman \phantom{*(}her house\\
      \glt `the woman's house'}
  \end{xlist}
  \end{exe}
%
In the analysis put forward by \citeauthor{berman-frank:1996}, the
main \textit{structural} difference between the the German prenominal genitive
and prenominal dative is that the former requires D to be
empty~(\ref{ex:WGermanic:73}a), whereas the latter requires D to be
filled~(\ref{ex:WGermanic:73}b). A binary feature on head realization is used to
control this.

\subsubsection{Declension and the status of quantifiers\label{sec:Germanic:dquant_aquant}}

Inside the German DP, determiners, adjectives and nouns show agreement
with respect to gender, number and case. \textit{Declension}
is another agreement dimension, found between determiners and
adjectives. Determiners have inherent declension: they can be
categorized as inflected (strong declension), uninflected or mixed. In the latter
case some cells are inflected/strong and others are
not. Inflected adjectives, on the other hand, have strong (more
distinctive morphology) and weak (less distinctive morphology)
declension paradigms. Adjective declension agrees with the inherent declension
of the determiner in the following way:
%
\begin{exe}
  \ex inflected (strong) determiner: weak adjective\\
      uninflected or no determiner: strong adjective
\end{exe}
%
This phenomenon is illustrated in~(\ref{ex:WGermanic:75}--\ref{ex:WGermanic:76}). For reasons of
exposition, the inflection is made explicit and we use a zero morpheme
to mark the lack of inflection. Note that \textit{ein} is a member of
the mixed declension class and appears both inflected (\ref{ex:WGermanic:75}b) and uninflected (\ref{ex:WGermanic:76}b).
\begin{exe}
  \ex\label{ex:WGermanic:75} German \citep[data from][§3.2, presentation/glosses changed]{Dipper05}
  \begin{xlist}
    \ex \gll d-er sü{\ss}-e rot-e Wein\\
        the-\M.\SG.\NOM{} sweet-\textsc{weak}.\SG.\NOM{} red-\textsc{weak}.\SG.\NOM{} wine(\M)\\
    \ex \gll ein-em sü{\ss}-en Wein\\
             a-\M.\SG.\DAT{} sweet-\textsc{weak}.\SG.\DAT{} wine(\M)\\
    \end{xlist}
  \ex\label{ex:WGermanic:76}
  \begin{xlist}
    \ex \gll sü{\ss}-er rot-er Wein\\
        sweet-\textsc{strong}.\M.\SG.\NOM{} red-\textsc{strong}.\M.\SG.\NOM{} wine(\M)\\
    \ex \gll ein-$\emptyset$ sü{\ss}-er Wein\\
             a-\M.\SG.\NOM{} sweet-\textsc{strong}.\M.\SG.\NOM{} wine(\M)\\
    \end{xlist}
\end{exe}
Determiners that do not inflect at all (for instance, \textit{allerlei} `every kind', \textit{solcherlei} `such') are not of
the strong declension, and adjectives that do not inflect at all
(\textit{lila} `purple', \textit{rosa} `pink') are ambiguous between
strong/weak declension.

\citet{Dipper05} models the facts about declension in the
following way: The f-structure projected from the DP has a feature
\textsc{decl}, whose value
is equated with \textsc{st-det} in strong determiners and with \textsc{st-adj} in strong
adjectives.  This captures the fact that
these two are never seen together. Weak adjectives constrain their containing DP's f-structure by 
$\textsc{decl}=_{c}\textsc{st-det}$, and therefore
only co-occur with strong determiners. Uninflected adjectives and
determiners do not constrain the \textsc{decl} feature at all.

As seen in~(\ref{ex:WGermanic:75}a) and~(\ref{ex:WGermanic:76}a), when the DP/NP contains
multiple adjectives, they show identical declension. \citet{Dipper05}
uses this fact to address the issue of the categorial status of
quantifiers like \textit{alle} `all' and \textit{mehrere} `multiple',
for which it is difficult to decide whether they are determiners or
adjectives. By inspecting the declension of adjectival material in the
presence of a quantifier, \citeauthor{Dipper05} is able to clearly
distinguish determiner-like and adjective-like quantifiers.



\subsubsection{Preposition-determiner contractions}

German has a number of lexical preposition-determiner (P-D) contractions, such as \textit{zum} `to the', and \textit{vom} `of the', shown in~(\ref{ex:WGermanic:77}). 
%
\begin{exe}
  \ex\label{ex:WGermanic:77}
  \begin{xlist}
    \ex[]{\gll zum König\\
      to.the.\M.\SG.\DAT{} king\\
    \glt `to the king'}
    \ex[]{\gll vom Bürgermeister seinem Gehalt\\
      of.the.\M.\SG.\DAT{} mayor his.\N.\SG.\DAT{} salary\\
    \glt `of the mayor's salary'}
  \end{xlist}
\end{exe}
%
In \citet{berman-frank:1996}, P-D contractions are treated as
prepositions that not only constrain their object DP in a P-like manner -- it
must have a given case -- but also in a D-like manner -- it is marked
definite and has certain agreement features. Most importantly, the
object DP may not itself realize its own D. This is enforced using the
binary head realization feature also used in the analysis of possessives,
sketched in~\sectref{sec:Germanic:posslowsax} above.

The example in~(\ref{ex:WGermanic:77}b) shows that this picture is too simplistic:
here, the D-like properties do not constrain the object DP, but the
prenominal dative of this DP. It is inside this prenominal dative that
D is left unrealized, and not in the object DP itself, which has
\textit{seinem} in D. The correct generalization about P-D
contractions must therefore include that the D inherent in the
contraction corresponds to a D leftmost in the object DP, and need not
be the object DP's head. \citet{wescoat2007} gives an analysis in
terms of lexical sharing that addresses exactly these points. In
\citeauthor{wescoat2007}'s lexical sharing model, one lexical terminal
can correspond to multiple adjacent preterminals. A slightly
simplified analysis of~(\ref{ex:WGermanic:77}b) is given in~(\ref{ex:WGermanic:78}).
%
\begin{exe}
  \ex\label{ex:WGermanic:78}
  \hspace*{-3mm}\begin{forest}
    [PP
      [\rulenode{\UP=\DOWN\\(\DOWN\OBJ\CASE)=\DAT\\(\DOWN\PRED)=`von\arglist{\OBJ}'\\P},name=P
        [{\hspace*{3.5em}{\gll
              vom\\
            of.the.\M.\SG.\DAT{} \\}\hspace*{-3.5em}}, name=notvom, no edge, tier=lex]]
      [\rulenode{(\UP\OBJ)=\DOWN\\DP}
        [\rulenode{(\UP\POSS)=\DOWN\\DP}
          [\rulenode{\UP=\DOWN\\(\DOWN\SPEC)=\textsc{the}\\(\DOWN\CASE)=\DAT\\(\DOWN\AGR)=\M.\SG\\D}, name=D [{}, no edge]]
          [\rulenode{\UP=\DOWN\\NP} [{\gll
                Bürgermeister\\
                mayor\\}, tier=lex]]
        ]
        [\rulenode{\UP=\DOWN\\\BAR{D}}
          [\rulenode{\UP=\DOWN\\D} [\hspace{.9em}{\gll
                seinem\\
                his.\N.\SG.\DAT{}\hspace*{-4em}\\}\hspace*{.9em}, tier=lex]]
          [\rulenode{\UP=\DOWN\\NP} [{\gll
                Gehalt\\
                salary\\}, l=8em, tier=lex]]
        ]
      ]
    ]
    \path (notvom) -- +(1em,.6em) node (vom) {\rule{0pt}{1.25em}};
    \draw (vom.north)--(P.south);
    \draw (vom.north)--(D.south);
  \end{forest}
\end{exe}
%
The fact that the preterminals projected from \textit{vom} need to be
adjacent solves the problem noted above that the generalization about
P-D contractions needs to include reference to the left edge of the
object DP. In the paper, \citeauthor{wescoat2007} describes
further constraints on the function of the projected D inside the
object DP.

\subsubsection{Indeterminacy of case\label{sec:Germanic:indeterminacy}}
\largerpage
The German nominal inflection paradigms show pervasive
syncretism. These syncretic forms can either be ambiguous or
indeterminate. Ambiguous forms can be used in different contexts, but
they can only be in one paradigmatic cell at a time. So \textit{sie}
in~(\ref{ex:WGermanic:79}), which in isolation is ambiguous between plural `they'
and feminine singular `she', can be used in either way~(\ref{ex:WGermanic:79}a,b), but not as both at once (\ref{ex:WGermanic:79}c). Indeterminate forms \textit{can}
function as if they are in different cells simultaneously. For
instance, \textit{Papageien} `parrots', which is indeterminate for
case, can at the same time be selected as an accusative object and a
dative object~(\ref{ex:WGermanic:80}).
%
\begin{exe}
  \ex\label{ex:WGermanic:79}
  \begin{xlist}
    \ex[]{\gll Sie hilft Papageien.\\
      she helps parrots\\}
    \ex[]{\gll Sie helfen Papageien.\\
      they help parrots\\}
    \ex[*]{Sie hilft und helfen Papageien.}
  \end{xlist}
  \ex\label{ex:WGermanic:80}
  \begin{xlist}
    \ex{\gll Sie hilft Papageien.\\
    she helps parrots.\DAT{}\\}
    \ex{\gll Sie findet Papageien.\\
    she finds parrots.\ACC{}\\}
    \ex{Sie findet und hilft Papageien.}
  \end{xlist}
\end{exe}
%
Although a simple disjunctive defining equation for a feature suffices
for the ambiguous cases, this is not enough to achieve indeterminacy,
since a disjunction does not change the fact that a feature can only
have one value at a time. \citet{DKS:Indeterminacy} represent
indeterminate features as bundles of binary features, one for each of
the values in the paradigmatic dimension. Compatibility with values is
given as a disjunction of \textit{positive} specifications,
incompatibility as \textit{negative} specifications. Two example lexical specifications are given in~(\ref{ex:WGermanic:81}).
%
\begin{exe}
  \ex\label{ex:WGermanic:81}
  \begin{xlist}
    \ex \lexentry{Papageien}{(\UP\CASE\,\{\NOM|\GEN|\DAT|\ACC\}) = {+}}
    \ex \lexentry{Männer}{(\UP\CASE\,\{\NOM|\GEN|\ACC\}) = {+}\\(\UP\CASE\DAT) = ${-}$}
  \end{xlist}
\end{exe}
%
These specifications state that \textit{Papageien} is completely
indeterminate with respect to case~(\ref{ex:WGermanic:81}a), whereas \textit{Männer} is
non-dative, but otherwise indeterminate with respect to case~(\ref{ex:WGermanic:81}b).

A selecting element then expresses its case requirements in positive terms only. The entries in~(\ref{ex:WGermanic:82}) illustrate this.
%
\begin{exe}
  \ex\label{ex:WGermanic:82}
  \begin{xlist}
    \ex \lexentry{hilft\textup{\footnotemark}}{(\UP\OBJ\CASE\DAT) = +}
    \ex \lexentry{findet}{(\UP\OBJ\CASE\ACC) = +}
  \end{xlist}
\end{exe}
%
\footnotetext{We follow here the presentation in the paper and gloss over
  the fact that \textit{helfen} `help' might be better analysed as taking an
  \OBJTHETA rather than an \OBJ, which would complicate modelling the
  coordination.}%
%
Since the case feature bundles for \textit{Papageien} defined
in~(\ref{ex:WGermanic:81}a) can satisfy both these requirements at the same time,
we can capture the coordination
of~(\ref{ex:WGermanic:80}c). \citeauthor{DKS:Indeterminacy} show that this approach
can also deal with additional material in the DP like adjectives, which further
constrain the case value, and with verbs which themselves are
indeterminate about their case requirements on selected arguments.

\section{Concluding remarks}\label{sec:Germanic:conclude}\largerpage[2]

This chapter has presented an overview of Lexical-Functional Grammar
studies of Continental West Germanic languages. The majority of the
work discussed here has dealt with German clausal syntax,
followed by discussions of Dutch clausal syntax. This reflects the status of the 
LFG field as a whole -- the nominal domain has received less attention than the 
clausal\slash verbal domain, an overview of LFG work on the former is given in 
\citetv{chapters/Nominal} – but it also
reflects the fact that the other CWG languages -- possibly, but not
only, minority, regional, and/or non-standardized languages -- do not
feature prominently in the LFG literature. I hope that the discussion
of existing work on the syntax of the two ``big'' CWG languages in the
current chapter may inspire further application of LFG to the other
members of the family.

Obviously, not every LFG study that touches upon CWG has been
mentioned in this chapter. There are some larger blind-spots that I
wish to mention here.
%
\begin{itemize}
\item \citet{Boegel2015} develops an LFG model of the
  prosody-syntax interface. Recent papers contain applications to
  Swabian \citep{BoegelRaach2020,boegel:2021:languages} and
  Standard German \citep{Boegel2020}. See also \citetv{chapters/Prosody} for a 
  discussion of the syntax-prosody interface in LFG.
\item A number of authors have used OT in combination with LFG,
  especially in the domain of word order variation and information
  structure. Examples are
  \citet{Choi1999,Choi2001}, \citet{cook:2001:revdiss}, \citet{CookPayne},\Citet{vanderbeek:2005}, and \citet{seiler:2007}. These
  have been mentioned in the text, but were not discussed in any
  detail. OT-LFG is dealt with in \citetv{chapters/OT}, and information structure is treated in \citetv{chapters/InformationStructure}.
\item German is blessed with a wide-coverage LFG grammar,
  implemented in the context of the ParGram project. This grammar can be
  queried in the interactive XLE-WEB interface.\footnote{\url{https://clarino.uib.no/iness/xle-web}, consulted July 2022} The project page for the
  ParGram project in Germany,\footnote{\url{https://www.ims.uni-stuttgart.de/en/research/projects/pargram}, consulted July 2022}
  contains older references. The research activities in and around
  this project have resulted in a long list of publications. Some of
  that work has already been discussed above. I will here list a small
  selection of further papers that also have direct relevance for
  theoretical debates:
  \citet{forst-rohrer:2009:lfg} and \citet{kuhn-etal:2010:lfg} discuss problems in
  the analysis of German VP coordination;
  \citet{rehbein-van-genabith:2006:german} and \citet{forstetal10} deal
  with the implementation of particle verbs; \citet{forst06} is
  a ``grammar writer's'' contribution to the \COMP-debate.  The desire
  for parallel structures in the context of ParGram is one of the
  forces behind the auxiliaries-as-features style of syntactic
  analysis in LFG. An early contribution and implementation can be
  found in \citet{butt-etal1996-coling}. Computational work on LFG is
  the topic of several chapters in Part~V of this volume.
\end{itemize} 
%
Omitting these studies from the main text was a conscious choice, intended to
keep the chapter accessible by not introducing too much conceptual
machinery and too many problem domains. I made this choice with the
knowledge that their topics would be touched upon in other
chapters. At the same time, I wish to underline their importance,
because exactly the fact that they span multiple domains and methods
means that they are excellent demonstrations of the flexibility and
precision that LFG offers.


\section*{Acknowledgements}

I thank Bozhil Hristov and two anonymous reviewers for the detailed
and valuable comments that greatly helped to improve this
chapter. Moreover, I am grateful to Mary Dalrymple for editing this
volume, inviting me to contribute, and her extraordinary patience
during the writing process.

\section*{Abbreviations}

Besides the abbreviations from the Leipzig Glossing Conventions, this
chapter uses the following abbreviations.\medskip

\noindent
\begin{tabularx}{.45\textwidth}{@{}lQ}
CWG & Continental West Germanic\\
\textsc{expl} & expletive\\
IPP & infinitivus participio\\
lb & left bracket\\
Mf & Mittelfeld\\
Nf & Nachfeld\\
\end{tabularx}
\begin{tabularx}{.45\textwidth}{lQ@{}}
OT  & Optimality Theory\\
rb & right bracket\\
\textsc{teinf} & (Dutch) infinitive with marker \textit{te}\\
Vf & Vorfeld\\
\textsc{zuinf} & (German) infinitive with marker \textit{zu}
\end{tabularx}

\printbibliography[heading=subbibliography,notkeyword=this]
\end{document}
