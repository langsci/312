\documentclass[output=paper,hidelinks]{langscibook}
\ChapterDOI{10.5281/zenodo.10185942}
\title{Agreement}
\author{Dag Haug\affiliation{University of Oslo}}
\abstract{This chapter surveys the treatment of agreement in LFG. We   show how theories of agreement can be classified by how they use   symmetry and feature sharing in their treatments and how LFG usually   opts for a symmetric but not feature sharing account. Other topics   include the \NINDEX/\CONCORD distinction, how non-f-structure such as   linear order and information structure impacts on agreement,   long-distance agreement and Wechsler's Agreement Marking Principle.}

\IfFileExists{../localcommands.tex}{
   \addbibresource{../localbibliography.bib}
   \addbibresource{thisvolume.bib}
   % add all extra packages you need to load to this file

\usepackage{tabularx}
\usepackage{multicol}
\usepackage{url}
\urlstyle{same}
%\usepackage{amsmath,amssymb}

% Tight underlining according to https://alexwlchan.net/2017/10/latex-underlines/
\usepackage{contour}
\usepackage[normalem]{ulem}
\renewcommand{\ULdepth}{1.8pt}
\contourlength{0.8pt}
\newcommand{\tightuline}[1]{%
  \uline{\phantom{#1}}%
  \llap{\contour{white}{#1}}}
  
\usepackage{listings}
\lstset{basicstyle=\ttfamily,tabsize=2,breaklines=true}

% \usepackage{langsci-basic}
\usepackage{langsci-optional}
\usepackage[danger]{langsci-lgr}
\usepackage{langsci-gb4e}
%\usepackage{langsci-linguex}
%\usepackage{langsci-forest-setup}
\usepackage[tikz]{langsci-avm} % added tikz flag, 29 July 21
% \usepackage{langsci-textipa}

\usepackage[linguistics,edges]{forest}
\usepackage{tikz-qtree}
\usetikzlibrary{positioning, tikzmark, arrows.meta, calc, matrix, shapes.symbols}
\usetikzlibrary{arrows, arrows.meta, shapes, chains, decorations.text}

%%%%%%%%%%%%%%%%%%%%% Packages for all chapters

% arrows and lines between structures
\usepackage{pst-node}

% lfg attributes and values, lines (relies on pst-node), lexical entries, phrase structure rules
\usepackage{packages/lfg-abbrevs}

% subfigures
\usepackage{subcaption}

% macros for small illustrations in the glossary
\usepackage{./packages/picins}

%%%%%%%%%%%%%%%%%%%%% Packages from contributors

% % Simpler Syntax packages
\usepackage{bm}
\tikzstyle{block} = [rectangle, draw, text width=5em, text centered, minimum height=3em]
\tikzstyle{line} = [draw, thick, -latex']

% Dependency packages
\usepackage{tikz-dependency}
%\usepackage{sdrt}

\usepackage{soul}

\usepackage[notipa]{ot-tableau}

% Historical
\usepackage{stackengine}
\usepackage{bigdelim}

% Morphology
\usepackage{./packages/prooftree}
\usepackage{arydshln}
\usepackage{stmaryrd}

% TAG
\usepackage{pbox}

\usepackage{langsci-branding}

   % %%%%%%%%% lang sci press commands

\newcommand*{\orcid}{}

\makeatletter
\let\thetitle\@title
\let\theauthor\@author
\makeatother

\newcommand{\togglepaper}[1][0]{
   \bibliography{../localbibliography}
   \papernote{\scriptsize\normalfont
     \theauthor.
     \titleTemp.
     To appear in:
     Dalrymple, Mary (ed.).
     Handbook of Lexical Functional Grammar.
     Berlin: Language Science Press. [preliminary page numbering]
   }
   \pagenumbering{roman}
   \setcounter{chapter}{#1}
   \addtocounter{chapter}{-1}
}

\DeclareOldFontCommand{\rm}{\normalfont\rmfamily}{\mathrm}
\DeclareOldFontCommand{\sf}{\normalfont\sffamily}{\mathsf}
\DeclareOldFontCommand{\tt}{\normalfont\ttfamily}{\mathtt}
\DeclareOldFontCommand{\bf}{\normalfont\bfseries}{\mathbf}
\DeclareOldFontCommand{\it}{\normalfont\itshape}{\mathit}
\makeatletter
\DeclareOldFontCommand{\sc}{\normalfont\scshape}{\@nomath\sc}
\makeatother

% Bug fix, 3 April 2021
\SetupAffiliations{output in groups = false,
                   separator between two = {\bigskip\\},
                   separator between multiple = {\bigskip\\},
                   separator between final two = {\bigskip\\}
                   }

% commands for all chapters
\setmathfont{LibertinusMath-Additions.otf}[range="22B8]

% punctuation between a sequence of years in a citation
% OLD: \renewcommand{\compcitedelim}{\multicitedelim}
\renewcommand{\compcitedelim}{\addcomma\space}

% \citegen with no parentheses around year
\providecommand{\citegenalt}[2][]{\citeauthor{#2}'s \citeyear*[#1]{#2}}

% avms with plain font, using langsci-avm package
\avmdefinestyle{plain}{attributes=\normalfont,values=\normalfont,types=\normalfont,extraskip=0.2em}
% avms with attributes and values in small caps, using langsci-avm package
\avmdefinestyle{fstr}{attributes=\scshape,values=\scshape,extraskip=0.2em}
% avms with attributes in small caps, values in plain font (from peter sells)
\avmdefinestyle{fstr-ps}{attributes=\scshape,values=\normalfont,extraskip=0.2em}

% reference to previous or following examples, from Stefan
%(\mex{1}) is like \next, referring to the next example
%(\mex{0}) is like \last, referring to the previous example, etc
\makeatletter
\newcommand{\mex}[1]{\the\numexpr\c@equation+#1\relax}
\makeatother

% do not add xspace before these
\xspaceaddexceptions{1234=|*\}\restrict\,}

% Several chapters use evnup -- this is verbatim from lingmacros.sty
\makeatletter
\def\evnup{\@ifnextchar[{\@evnup}{\@evnup[0pt]}}
\def\@evnup[#1]#2{\setbox1=\hbox{#2}%
\dimen1=\ht1 \advance\dimen1 by -.5\baselineskip%
\advance\dimen1 by -#1%
\leavevmode\lower\dimen1\box1}
\makeatother

% Centered entries in tables.  Requires array package.
\newcolumntype{P}[1]{>{\centering\arraybackslash}p{#1}}

% Reference to multiple figures, requested by Victoria Rosen
\newcommand{\figsref}[2]{Figures~\ref{#1}~and~\ref{#2}}
\newcommand{\figsrefthree}[3]{Figures~\ref{#1},~\ref{#2}~and~\ref{#3}}
\newcommand{\figsreffour}[4]{Figures~\ref{#1},~\ref{#2},~\ref{#3}~and~\ref{#4}}
\newcommand{\figsreffive}[5]{Figures~\ref{#1},~\ref{#2},~\ref{#3},~\ref{#4}~and~\ref{#5}}

% Semitic chapter:
\providecommand{\textchi}{χ}

% Prosody chapter
\makeatletter
\providecommand{\leftleadsto}{%
  \mathrel{\mathpalette\reflect@squig\relax}%
}
\newcommand{\reflect@squig}[2]{%
  \reflectbox{$\m@th#1$$\leadsto$}%
}
\makeatother
\newcommand\myrotaL[1]{\mathrel{\rotatebox[origin=c]{#1}{$\leadsto$}}}
\newcommand\Prosleftarrow{\myrotaL{-135}}
\newcommand\myrotaR[1]{\mathrel{\rotatebox[origin=c]{#1}{$\leftleadsto$}}}
\newcommand\Prosrightarrow{\myrotaR{135}}

% Core Concepts chapter
\newcommand{\anterm}[2]{#1\\#2}
\newcommand{\annode}[2]{#1\\#2}

% HPSG chapter
\newcommand{\HPSGphon}[1]{〈#1〉}
% for defining RSRL relations:
\newcommand{\HPSGsfl}{\enskip\ensuremath{\stackrel{\forall{}}{\Longleftarrow{}}}\enskip}
% AVM commands, valid only inside \avm{}
\avmdefinecommand {phon}[phon] { attributes=\itshape } % define a new \phon command
% Forest Set-up
\forestset
  {notin label above/.style={edge label={node[midway,sloped,above,inner sep=0pt]{\strut$\ni$}}},
    notin label below/.style={edge label={node[midway,sloped,below,inner sep=0pt]{\strut$\ni$}}},
  }

% Dependency chapter
\newcommand{\ua}{\ensuremath{\uparrow}}
\newcommand{\da}{\ensuremath{\downarrow}}
\forestset{
  dg edges/.style={for tree={parent anchor=south, child anchor=north,align=center,base=bottom},
                 where n children=0{tier=word,edge=dotted,calign with current edge}{}
                },
dg transfer/.style={edge path={\noexpand\path[\forestoption{edge}, rounded corners=3pt]
    % the line downwards
    (!u.parent anchor)-- +($(0,-l)-(0,4pt)$)-- +($(12pt,-l)-(0,4pt)$)
    % the horizontal line
    ($(!p.north west)+(0,l)-(0,20pt)$)--($(.north east)+(0,l)-(0,20pt)$)\forestoption{edge label};},!p.edge'={}},
% for Tesniere-style junctions
dg junction/.style={no edge, tikz+={\draw (!p.east)--(!.west) (.east)--(!n.west);}    }
}


% Glossary
\makeatletter % does not work with \newcommand
\def\namedlabel#1#2{\begingroup
   \def\@currentlabel{#2}%
   \phantomsection\label{#1}\endgroup
}
\makeatother


\renewcommand{\textopeno}{ɔ}
\providecommand{\textepsilon}{ɛ}

\renewcommand{\textbari}{ɨ}
\renewcommand{\textbaru}{ʉ}
\newcommand{\acutetextbari}{í̵}
\renewcommand{\textlyoghlig}{ɮ}
\renewcommand{\textdyoghlig}{ʤ}
\renewcommand{\textschwa}{ə}
\renewcommand{\textprimstress}{ˈ}
\newcommand{\texteng}{ŋ}
\renewcommand{\textbeltl}{ɬ}
\newcommand{\textramshorns}{ɤ}

\newbool{bookcompile}
\booltrue{bookcompile}
\newcommand{\bookorchapter}[2]{\ifbool{bookcompile}{#1}{#2}}




\renewcommand{\textsci}{ɪ}
\renewcommand{\textturnscripta}{ɒ}

\renewcommand{\textscripta}{ɑ}
\renewcommand{\textteshlig}{ʧ}
\providecommand{\textupsilon}{υ}
\renewcommand{\textyogh}{ʒ}
\newcommand{\textpolhook}{̨}

\renewcommand{\sectref}[1]{Section~\ref{#1}}

%\KOMAoptions{chapterprefix=true}

\renewcommand{\textturnv}{ʌ}
\renewcommand{\textrevepsilon}{ɜ}
\renewcommand{\textsecstress}{ˌ}
\renewcommand{\textscriptv}{ʋ}
\renewcommand{\textglotstop}{ʔ}
\renewcommand{\textrevglotstop}{ʕ}
%\newcommand{\textcrh}{ħ}
\renewcommand{\textesh}{ʃ}

% label for submitted and published chapters
\newcommand{\submitted}{{\color{red}Final version submitted to Language Science Press.}}
\newcommand{\published}{{\color{red}Final version published by
    Language Science Press, available at \url{https://langsci-press.org/catalog/book/312}.}}

% Treebank definitions
\definecolor{tomato}{rgb}{0.9,0,0}
\definecolor{kelly}{rgb}{0,0.65,0}

% Minimalism chapter
\newcommand\tr[1]{$<$\textcolor{gray}{#1}$>$}
\newcommand\gapline{\lower.1ex\hbox to 1.2em{\bf \ \hrulefill\ }}
\newcommand\cnom{{\llap{[}}Case:Nom{\rlap{]}}}
\newcommand\cacc{{\llap{[}}Case:Acc{\rlap{]}}}
\newcommand\tpres{{\llap{[}}Tns:Pres{\rlap{]}}}
\newcommand\fstackwe{{\llap{[}}Tns:Pres{\rlap{]}}\\{\llap{[}}Pers:1{\rlap{]}}\\{\llap{[}}Num:Pl{\rlap{]}}}
\newcommand\fstackone{{\llap{[}}Tns:Past{\rlap{]}}\\{\llap{[}}Pers:\ {\rlap{]}}\\{\llap{[}}Num:\ {\rlap{]}}}
\newcommand\fstacktwo{{\llap{[}}Pers:3{\rlap{]}}\\{\llap{[}}Num:Pl{\rlap{]}}\\{\llap{[}}Case:\ {\rlap{]}}}
\newcommand\fstackthr{{\llap{[}}Tns:Past{\rlap{]}}\\{\llap{[}}Pers:3{\rlap{]}}\\{\llap{[}}Num:Pl{\rlap{]}}} 
\newcommand\fstackfou{{\llap{[}}Pers:3{\rlap{]}}\\{\llap{[}}Num:Pl{\rlap{]}}\\{\llap{[}}Case:Nom{\rlap{]}}}
\newcommand\fstackonefill{{\llap{[}}Tns:Past{\rlap{]}}\\{\llap{[}}Pers:3{\rlap{]}}\\%
  {\llap{[}}Num:Pl{\rlap{]}}}
\newcommand\fstackoneint%
    {{\llap{[}}{\bf Tns:Past}{\rlap{]}}\\{\llap{[}}Pers:\ {\rlap{]}}\\{\llap{[}}Num:\ {\rlap{]}}}
\newcommand\fstacktwoint%
    {{\llap{[}}{\bf Pers:3}{\rlap{]}}\\{\llap{[}}{\bf Num:Pl}{\rlap{]}}\\{\llap{[}}Case:\ {\rlap{]}}}
\newcommand\fstackthrchk%
    {{\llap{[}}{\bf Tns:Past}{\rlap{]}}\\{\llap{[}}{Pers:3}{\rlap{]}}\\%
      {\llap{[}}Num:Pl{\rlap{]}}} 
\newcommand\fstackfouchk%
    {{\llap{[}}{\bf Pers:3}{\rlap{]}}\\{\llap{[}}{\bf Num:Pl}{\rlap{]}}\\%
      {\llap{[}}Case:Nom{\rlap{]}}}
\newcommand\uinfl{{\llap{[}}Infl:\ \ {\rlap{]}}}
\newcommand\inflpass{{\llap{[}}Infl:Pass{\rlap{]}}}
\newcommand\fepp{{\llap{[}}EPP{\rlap{]}}}
\newcommand\sepp{{\llap{[}}\st{EPP}{\rlap{]}}}
\newcommand\rdash{\rlap{\hbox to 24em{\hfill (dashed lines represent
      information flow)}}}


% Computational chapter
\usepackage{./packages/kaplan}
\renewcommand{\red}{\color{lsLightWine}}

% Sinitic
\newcommand{\FRAME}{\textsc{frame}\xspace}
\newcommand{\arglistit}[1]{{\textlangle}\textit{#1}{\textrangle}}

%WestGermanic
\newcommand{\streep}[1]{\mbox{\rule{1pt}{0pt}\rule[.5ex]{#1}{.5pt}\rule{-1pt}{0pt}\rule{-#1}{0pt}}}

\newcommand{\hspaceThis}[1]{\hphantom{#1}}


\newcommand{\FIG}{\textsc{figure}}
\newcommand{\GR}{\textsc{ground}}

%%%%% Morphology
% Single quote
\newcommand{\asquote}[1]{`{#1}'} % Single quotes
\newcommand{\atrns}[1]{\asquote{#1}} % Translation
\newcommand{\attrns}[1]{(\asquote{#1})} % Translation
\newcommand{\ascare}[1]{\asquote{#1}} % Scare quotes
\newcommand{\aqterm}[1]{\asquote{#1}} % Quoted terms
% Double quote
\newcommand{\adquote}[1]{``{#1}''} % Double quotes
\newcommand{\aquoot}[1]{\adquote{#1}} % Quotes
% Italics
\newcommand{\aword}[1]{\textit{#1}}  % mention of word
\newcommand{\aterm}[1]{\textit{#1}}
% Small caps
\newcommand{\amg}[1]{{\textsc{\MakeLowercase{#1}}}}
\newcommand{\ali}[1]{\MakeLowercase{\textsc{#1}}}
\newcommand{\feat}[1]{{\textsc{#1}}}
\newcommand{\val}[1]{\textsc{#1}}
\newcommand{\pred}[1]{\textsc{#1}}
\newcommand{\predvall}[1]{\textsc{#1}}
% Misc commands
\newcommand{\exrr}[2][]{(\ref{ex:#2}{#1})}
\newcommand{\csn}[3][t]{\begin{tabular}[#1]{@{\strut}c@{\strut}}#2\\#3\end{tabular}}
\newcommand{\sem}[2][]{\ensuremath{\left\llbracket \mbox{#2} \right\rrbracket^{#1}}}
\newcommand{\apf}[2][\ensuremath{\sigma}]{\ensuremath{\langle}#2,#1\ensuremath{\rangle}}
\newcommand{\formula}[2][t]{\ensuremath{\begin{array}[#1]{@{\strut}l@{\strut}}#2%
                                         \end{array}}}
\newcommand{\Down}{$\downarrow$}
\newcommand{\Up}{$\uparrow$}
\newcommand{\updown}{$\uparrow=\downarrow$}
\newcommand{\upsigb}{\mbox{\ensuremath{\uparrow\hspace{-0.35em}_\sigma}}}
\newcommand{\lrfg}{L\textsubscript{R}FG} 
\newcommand{\dmroot}{\ensuremath{\sqrt{\hspace{1em}}}}
\newcommand{\amother}{\mbox{\ensuremath{\hat{\raisebox{-.25ex}{\ensuremath{\ast}}}}}}
\newcommand{\expone}{\ensuremath{\xrightarrow{\nu}}}
\newcommand{\sig}{\mbox{$_\sigma\,$}}
\newcommand{\aset}[1]{\{#1\}}
\newcommand{\linimp}{\mbox{\ensuremath{\,\multimap\,}}}
\newcommand{\fsfunc}{\ensuremath{\Phi}\hspace*{-.15em}}
\newcommand{\cons}[1]{\ensuremath{\mbox{\textbf{\textup{#1}}}}}
\newcommand{\amic}[1][]{\cons{MostInformative$_c$}{#1}}
\newcommand{\amif}[1][]{\cons{MostInformative$_f$}{#1}}
\newcommand{\amis}[1][]{\cons{MostInformative$_s$}{#1}}
\newcommand{\amsp}[1][]{\cons{MostSpecific}{#1}}

%Glue
\newcommand{\glues}{Glue Semantics} % macro for consistency
\newcommand{\glue}{Glue} % macro for consistency
\newcommand{\lfgglue}{LFG$+$Glue} 
\newcommand{\scare}[1]{`{#1}'} % Scare quotes
\newcommand{\word}[1]{\textit{#1}}  % mention of word
\newcommand{\dquote}[1]{``{#1}''} % Double quotes
\newcommand{\high}[1]{\textit{#1}} % highlight (italicize)
\newcommand{\laml}{{L}} 
% Left interpretation double bracket
\newcommand{\Lsem}{\ensuremath{\left\llbracket}} 
% Right interpretation double bracket
\newcommand{\Rsem}{\ensuremath{\right\rrbracket}} 
\newcommand{\nohigh}[1]{{#1}} % nohighlight (regular font)
% Linear implication elimination
\newcommand{\linimpE}{\mbox{\small\ensuremath{\multimap_{\mathcal{E}}}}}
% Linear implication introduction, plain
\newcommand{\linimpI}{\mbox{\small\ensuremath{\multimap_{\mathcal{I}}}}}
% Linear implication introduction, with flag
\newcommand{\linimpIi}[1]{\mbox{\small\ensuremath{\multimap_{{\mathcal{I}},#1}}}}
% Linear universal elimination
\newcommand{\forallE}{\mbox{\small\ensuremath{\forall_{{\mathcal{E}}}}}}
% Tensor elimination
\newcommand{\tensorEij}[2]{\mbox{\small\ensuremath{\otimes_{{\mathcal{E}},#1,#2}}}}
% CG forward slash
\newcommand{\fs}{\ensuremath{/}} 
% s-structure mapping, no space after                                     
\newcommand{\sigb}{\mbox{$_\sigma$}}
% uparrow with s-structure mapping, with small space after  
\newcommand{\upsig}{\mbox{\ensuremath{\uparrow\hspace{-0.35em}_\sigma\,}}}
\newcommand{\fsa}[1]{\textit{#1}}
\newcommand{\sqz}[1]{#1}
% Angled brackets (types, etc.)
\newcommand{\bracket}[1]{\ensuremath{\left\langle\mbox{\textit{#1}}\right\rangle}}
% glue logic string term
\newcommand{\gterm}[1]{\ensuremath{\mbox{\textup{\textit{#1}}}}}
% abstract grammatical formative
\newcommand{\gform}[1]{\ensuremath{\mbox{\textsc{\textup{#1}}}}}
% let
\newcommand{\llet}[3]{\ensuremath{\mbox{\textsf{let}}~{#1}~\mbox{\textsf{be}}~{#2}~\mbox{\textsf{in}}~{#3}}}
% Word-adorned proof steps
\providecommand{\vformula}[2]{%
  \begin{array}[b]{l}
    \mbox{\textbf{\textit{#1}}}\\%[-0.5ex]
    \formula{#2}
  \end{array}
}

%TAG
\newcommand{\fm}[1]{\textsc{#1}}
\newcommand{\struc}[1]{{#1-struc\-ture}}
\newcommand{\func}[1]{\mbox{#1-function}}
\newcommand{\fstruc}{\struc{f}}
\newcommand{\cstruc}{\struc{c}}
\newcommand{\sstruc}{\struc{s}}
\newcommand{\astruc}{\struc{a}}
\newcommand{\nodelabels}[2]{\rlap{\ensuremath{^{#1}_{#2}}}}
\newcommand{\footnode}{\rlap{\ensuremath{^{*}}}}
\newcommand{\nafootnode}{\rlap{\ensuremath{^{*}_{\nalabel}}}}
\newcommand{\nanode}{\rlap{\ensuremath{_{\nalabel}}}}
\newcommand{\AdjConstrText}[1]{\textnormal{\small #1}}
\newcommand{\nalabel}{\AdjConstrText{NA}}

%Case
\newcommand{\MID}{\textsc{mid}{}\xspace}

%font commands added April 2023 for Control and Case chapters
\def\textthorn{þ}
\def\texteth{ð}
\def\textinvscr{ʁ}
\def\textcrh{ħ}
\def\textgamma{ɣ}

% Coordination
\newcommand{\CONJ}{\textsc{conj}{}\xspace}
\newcommand*{\phtm}[1]{\setbox0=\hbox{#1}\hspace{\wd0}}
\newcommand{\ggl}{\hfill(Google)}
\newcommand{\nkjp}{\hfill(NKJP)}

% LDDs
\newcommand{\ubd}{\attr{ubd}\xspace}
% \newcommand{\disattr}[1]{\blue \attr{#1}}  % on topic/focus path
% \newcommand{\proattr}[1]{\green\attr{#1}}  % On Q/Relpro path
\newcommand{\disattr}[1]{\color{lsMidBlue}\attr{#1}}  % on topic/focus path
\newcommand{\proattr}[1]{\color{lsMidGreen}\attr{#1}}  % On Q/Relpro path
\newcommand{\eestring}{\mbox{$e$}\xspace}
\providecommand{\disj}[1]{\{\attr{#1}\}}
\providecommand{\estring}{\mb{\epsilon}}
\providecommand{\termcomp}[1]{\attr{\backslash {#1}}}
\newcommand{\templatecall}[2]{{\small @}(\attr{#1}\ \attr{#2})}
\newcommand{\xlgf}[1]{(\leftarrow\ \attr{#1})} 
\newcommand{\xrgf}[1]{(\rightarrow\ \attr{#1})}
\newcommand{\rval}[2]{\annobox {\xrgf{#1}\teq\attr{#2}}}
\newcommand{\memb}[1]{\annobox {\downarrow\, \in \xugf{#1}}}
\newcommand{\lgf}[1]{\annobox {\xlgf{#1}}}
\newcommand{\rgf}[1]{\annobox {\xrgf{#1}}}
\newcommand{\rvalc}[2]{\annobox {\xrgf{#1}\teqc\attr{#2}}}
\newcommand{\xgfu}[1]{(\attr{#1}\uparrow)}
\newcommand{\gfu}[1]{\annobox {\xgfu{#1}}}
\newcommand{\nmemb}[3]{\annobox {{#1}\, \in \ngf{#2}{#3}}}
\newcommand{\dgf}[1]{\annobox {\xdgf{#1}}}
\newcommand{\predsfraise}[3]{\annobox {\xugf{pred}\teq\semformraise{#1}{#2}{#3}}}
\newcommand{\semformraise}[3]{\annobox {\textrm{`}\hspace{-.05em}\attr{#1}\langle\attr{#2}\rangle{\attr{#3}}\textrm{'}}}
\newcommand{\teqc}{\hspace{-.1667em}=_c\hspace{-.1667em}} 
\newcommand{\lval}[2]{\annobox {\xlgf{#1}\teq\attr{#2}}}
\newcommand{\xgfd}[1]{(\attr{#1}\downarrow)}
\newcommand{\gfd}[1]{\annobox {\xgfd{#1}}}
\newcommand{\gap}{\rule{.75em}{.5pt}\ }
\newcommand{\gapp}{\rule{.75em}{.5pt}$_p$\ }

% Mapping
% Avoid having to write 'argument structure' a million times
\newcommand{\argstruc}{argument structure}
\newcommand{\Argstruc}{Argument structure}
\newcommand{\emptybracks}{\ensuremath{[\;\;]}}
\newcommand{\emptycurlybracks}{\ensuremath{\{\;\;\}}}
% Drawing lines in structures
\newcommand{\strucconnect}[6]{%
\draw[-stealth] (#1) to[out=#5, in=#6] node[pos=#3, above]{#4} (#2);%
}
\newcommand{\strucconnectdashed}[6]{%
\draw[-stealth, dashed] (#1) to[out=#5, in=#6] node[pos=#3, above]{#4} (#2);%
}
% Attributes for s-structures in the style of lfg-abbrevs.sty
\newcommand{\ARGnum}[1]{\textsc{arg}\textsubscript{#1}}
% Drawing mapping lines
\newcommand{\maplink}[2]{%
\begin{tikzpicture}[baseline=(A.base)]
\node(A){#1\strut};
\node[below = 3ex of A](B){\pbox{\textwidth}{#2}};
\draw ([yshift=-1ex]A.base)--(B);
% \draw (A)--(B);
\end{tikzpicture}}
% long line for extra features
\newcommand{\longmaplink}[2]{%
\begin{tikzpicture}[baseline=(A.base)]
\node(A){#1\strut};
\node[below = 3ex of A](B){\pbox{\textwidth}{#2}};
\draw ([yshift=2.5ex]A.base)--(B);
% \draw (A)--(B);
\end{tikzpicture}%
}
% For drawing upward
\newcommand{\maplinkup}[2]{%
\begin{tikzpicture}[baseline=(A.base)]
\node(A){#1};
\node[above = 3ex of A, anchor=base](B){#2};
\draw (A)--(B);
\end{tikzpicture}}
% Above with arrow going down (for argument adding processes)
\newcommand{\argumentadd}[2]{%
\begin{tikzpicture}[baseline=(A.base)]
\node(A){#1};
\node[above = 3ex of A, anchor=base](B){#2};
\draw[latex-] ([yshift=2ex]A.base)--([yshift=-1ex]B.center);
\end{tikzpicture}}
% Going up to the left
\newcommand{\maplinkupleft}[2]{%
\begin{tikzpicture}[baseline=(A.base)]
\node(A){#1};
\node[above left = 3ex of A, anchor=base](B){#2};
\draw (A)--(B);
\end{tikzpicture}}
% Going up to the right
\newcommand{\maplinkupright}[2]{%
\begin{tikzpicture}[baseline=(A.base)]
\node(A){#1};
\node[above right = 3ex of A, anchor=base](B){#2};
\draw (A)--(B);
\end{tikzpicture}}
% Argument fusion
\newenvironment{tikzsentence}{\begin{tikzpicture}[baseline=0pt, 
  anchor=base, outer sep=0pt, ampersand replacement=\&
   ]}{\end{tikzpicture}}
\newcommand{\Subnode}[2]{\subnode[inner sep=1pt]{#1}{#2\strut}}
\newcommand{\connectbelow}[3]{\draw[inner sep=0pt] ([yshift=0.5ex]#1.south) -- ++ (south:#3ex)
  -| ([yshift=0.5ex]#2.south);}
\newcommand{\connectabove}[3]{\draw[inner sep=0pt] ([yshift=0ex]#1.north) -- ++ (north:#3ex)
  -| ([yshift=0ex]#2.north);}
  
\newcommand{\ASNode}[2]{\tikz[remember picture,baseline=(#1.base)] \node [anchor=base] (#1) {#2};}

% Austronesian
\newcommand{\LV}{\textsc{lv}\xspace}
\newcommand{\IV}{\textsc{iv}\xspace}
\newcommand{\DV}{\textsc{dv}\xspace}
\newcommand{\PV}{\textsc{pv}\xspace}
\newcommand{\AV}{\textsc{av}\xspace}
\newcommand{\UV}{\textsc{uv}\xspace}

\apptocmd{\appendix}
         {\bookmarksetup{startatroot}}
         {}
         {%
           \AtEndDocument{\typeout{langscibook Warning:}
                          \typeout{It was not possible to set option 'staratroot'}
                          \typeout{for appendix in the backmatter.}}
         }

   %% hyphenation points for line breaks
%% Normally, automatic hyphenation in LaTeX is very good
%% If a word is mis-hyphenated, add it to this file
%%
%% add information to TeX file before \begin{document} with:
%% %% hyphenation points for line breaks
%% Normally, automatic hyphenation in LaTeX is very good
%% If a word is mis-hyphenated, add it to this file
%%
%% add information to TeX file before \begin{document} with:
%% %% hyphenation points for line breaks
%% Normally, automatic hyphenation in LaTeX is very good
%% If a word is mis-hyphenated, add it to this file
%%
%% add information to TeX file before \begin{document} with:
%% \include{localhyphenation}
\hyphenation{
Aus-tin
Bel-ya-ev
Bres-nan
Chom-sky
Eng-lish
Geo-Gram
INESS
Inkelas
Kaplan
Kok-ko-ni-dis
Lacz-kó
Lam-ping
Lu-ra-ghi
Lund-quist
Mcho-mbo
Meu-rer
Nord-lin-ger
PASSIVE
Pa-no-va
Pol-lard
Pro-sod-ic
Prze-piór-kow-ski
Ram-chand
Sa-mo-ye-dic
Tsu-no-da
WCCFL
Wam-ba-ya
Warl-pi-ri
Wes-coat
Wo-lof
Zae-nen
accord-ing
an-a-phor-ic
ana-phor
christ-church
co-description
co-present
con-figur-ation-al
in-effa-bil-ity
mor-phe-mic
mor-pheme
non-com-po-si-tion-al
pros-o-dy
referanse-grammatikk
rep-re-sent
Schätz-le
term-hood
Kip-ar-sky
Kok-ko-ni
Chi-che-\^wa
au-ton-o-mous
Al-si-na
Ma-tsu-mo-to
}

\hyphenation{
Aus-tin
Bel-ya-ev
Bres-nan
Chom-sky
Eng-lish
Geo-Gram
INESS
Inkelas
Kaplan
Kok-ko-ni-dis
Lacz-kó
Lam-ping
Lu-ra-ghi
Lund-quist
Mcho-mbo
Meu-rer
Nord-lin-ger
PASSIVE
Pa-no-va
Pol-lard
Pro-sod-ic
Prze-piór-kow-ski
Ram-chand
Sa-mo-ye-dic
Tsu-no-da
WCCFL
Wam-ba-ya
Warl-pi-ri
Wes-coat
Wo-lof
Zae-nen
accord-ing
an-a-phor-ic
ana-phor
christ-church
co-description
co-present
con-figur-ation-al
in-effa-bil-ity
mor-phe-mic
mor-pheme
non-com-po-si-tion-al
pros-o-dy
referanse-grammatikk
rep-re-sent
Schätz-le
term-hood
Kip-ar-sky
Kok-ko-ni
Chi-che-\^wa
au-ton-o-mous
Al-si-na
Ma-tsu-mo-to
}

\hyphenation{
Aus-tin
Bel-ya-ev
Bres-nan
Chom-sky
Eng-lish
Geo-Gram
INESS
Inkelas
Kaplan
Kok-ko-ni-dis
Lacz-kó
Lam-ping
Lu-ra-ghi
Lund-quist
Mcho-mbo
Meu-rer
Nord-lin-ger
PASSIVE
Pa-no-va
Pol-lard
Pro-sod-ic
Prze-piór-kow-ski
Ram-chand
Sa-mo-ye-dic
Tsu-no-da
WCCFL
Wam-ba-ya
Warl-pi-ri
Wes-coat
Wo-lof
Zae-nen
accord-ing
an-a-phor-ic
ana-phor
christ-church
co-description
co-present
con-figur-ation-al
in-effa-bil-ity
mor-phe-mic
mor-pheme
non-com-po-si-tion-al
pros-o-dy
referanse-grammatikk
rep-re-sent
Schätz-le
term-hood
Kip-ar-sky
Kok-ko-ni
Chi-che-\^wa
au-ton-o-mous
Al-si-na
Ma-tsu-mo-to
}

   \togglepaper[6]%%chapternumber
}{}

\begin{document}
\maketitle
\label{chap:Agreement}

\section{Introduction}

\emph{Agreement} is the linguistic phenomenon whereby a set of features
is realized morphologically on two different syntactic tokens, as we see in (\ref{ex:Agreement:1}).

\ea\label{ex:Agreement:1} The boy loves the girl.  \z Both the word \textit{boy} and the
word \textit{loves} realize a singular number feature.\footnote{We are
relying here on an inferential, realizational view of morphology
whereby \textit{boy} is morphologically singular even if there is no
singular morpheme.} However, this feature is only meaningful on
\textit{boy}, where it indicates that that the noun phrase refers to a
single boy; \textit{loves} merely agrees, in this case with its
subject. Agreement is therefore a directed phenomenon: the
\textit{controller} (`boy') has a set of meaningful features and the
\textit{target} (`loves') agrees with these.

``Meaningful'' must be taken with a grain of salt. We can also have
agreement in purely syntactic features such as \textsc{case} or in
features that are inherent in the controller but do not carry any
obvious meaning, such as \textsc{gender}. But even in such cases, we
observe directionality. Consider (\ref{ex:rosaspinosa}) from Latin.

\ea\label{ex:rosaspinosa} Latin\\
\gll rosa spinosa floruit\\
rose:\textsc{nom;f;sg} thorny:\textsc{nom;f;sg} bloomed:\textsc{pst;3sg}\\
\glt `The thorny rose bloomed.'
\z
%
The nominative case feature that is realized on \textit{rosa} and
\textit{spinosa} is only meaningful on \textit{rosa} because it
indicates the grammatical function (subject) of the noun phrase. By
contrast the grammatical function (adjunct) of \textit{spinosa} is
given by the fact that its case agrees with that of its head, rather
than by a specific case feature: if the NP was in object position
instead, the case of \textit{rosa} would change because the
grammatical function of the noun phrase would change; and the case of
the adjective \textit{spinosa} would also change, despite its
grammatical function as adjunct remaining the same. Finally, the
feminine gender feature in (\ref{ex:rosaspinosa}) is an inherent, purely formal
property of the controller: it does not provide any information about
the syntactic function or the meaning of the noun phrase headed by
\textit{rosa}, but is a non-variable feature of \textit{rosa} which is
part of the information conveyed by the lexeme. By contrast, the
adjective \textit{spinosa} inflects for this feature and can assume
other gender features, depending on the inherent gender of its
controller.

There are three main areas where languages display agreement phenomena. First, there is agreement in predicate-argument structures, where one or more arguments typically act as controllers and the predicate is the target. Second, we observe agreement inside NPs, where typically the head noun controls agreement on targets like determiners, quantifiers, adjectives and other modifiers. Third, we have “anaphoric agreement” between anaphors and antecedents. The latter type of agreement has attracted little attention in LFG work and will consequently largely be ignored here, except that it is relevant as a diachronic source of predicate-argument agreement.

In \sectref{sec:symmetry}, we show how theories of agreement can be classified by how they use symmetry and feature sharing in their treatments. In \sectref{sec:indexconcord} we discuss the \textsc{index}/\textsc{concord} distinction that is drawn in much LFG work on agreement. While agreement is generally treated at f-structure in LFG, \sectref{sec:outsidef} discusses how linear order and information structure impacts on agreement. \sectref{sec:diachrony} discusses the diachrony of agreement markers. \sectref{sec:domains} discusses long-distance agreement, a phenomenon which suggests there may be a role for feature sharing in agreement to preserve syntactic locality. Finally, \sectref{sec:agreementmarking} discusses Wechsler's Agreement Marking Principle, which is a challenge to symmetric accounts of agreement. 

\section{Agreement in unification grammars} \label{sec:symmetry}
The basic treatment of agreement in unification-based grammars is very
straightforward as we simply need to make sure that the relevant
features of the controller and the target unify. This is usually done
by specifying functional descriptions that put the features in the
same position in the functional structure, namely that of the
controller. The specifications of (\ref{ex:rosaspinosa}) are shown in (\ref{ex:f-descriptions}) and
yield the f-structure in (\ref{ex:bloomingrosef}). Only relevant features are shown.

\ea\label{ex:f-descriptions}
\lexentry{rosa}{(\UP\ \textsc{pred}) = \textsc{`rose'} \\
                 (\UP\ \textsc{num}) = \textsc{sg} \\
                 (\UP\ \textsc{case}) = \textsc{nom} \\
                 (\UP\ \textsc{gend}) = \textsc{fem}} \\
\lexentry{floruit}{(\UP\ \textsc{pred}) = \textsc{`bloom$<$subj$>$'} \\
                (\UP\ \textsc{subj num}) = \textsc{sg} \\
                (\UP\ \textsc{subj case}) = \textsc{nom} \\
                (\UP\ \textsc{subj pers}) = \textsc{3}} \\
\lexentry{spinosa}{(\UP\ \textsc{pred}) = \textsc{`thorny'} \\
                ((\textsc{adj} $\in$ \UP) \textsc{num}) = \textsc{sg} \\
                ((\textsc{adj} $\in$ \UP) \textsc{case}) = \textsc{nom} \\
                ((\textsc{adj} $\in$ \UP) \textsc{gend}) = \textsc{fem}}
\z
%
\eabox{\label{ex:bloomingrosef}
{\avm[style=fstr]{
  [pred & `bloom\arglist{subj}'\\
    subj & [pred & `rose'\\
      num & sg\\
      case & nom\\
      gend & fem\\
      pers & 3\\
      adj & \{ [ pred & `thorny' ] \} ] ] }}
}
%
In this approach to agreement, there is \emph{symmetry} between the
controller and target features in that it does not matter whether a
feature value originates from a functional description associated with
the controller or the target or both. However, agreement features are
\emph{not} shared (in the technical sense of structure sharing in
f-structures), but only represented in a single position in the
f-structure, that of the controller, reflecting the directedness of
agreement. It is this symmetric, yet not feature-sharing approach to agreement that gives the standard LFG analysis its specific flavor, different from analyses that are often found in the
derivational tradition starting from \citet{Pollock:89} and in HPSG \citep[chapter~2]{pollard1994head-driven}.

In current derivational approaches, controller features are
inter\-pret\-able and target features are un\-inter\-pret\-able. The Agree
mechanism matches un\-inter\-pret\-able features to their interpretable
counterparts and deletes them. If uninterpretable features remain, the
derivation crashes. Hence all target features must be available on the
controller. But in Latin, which is a pro-drop language, this forces us
to postulate several null subjects differing only in their
interpretable \textsc{pers} and \textsc{num} values, merely to check
off the matching uninterpretable features on the verb. The same point
is made by \citet{Barlow:PhD} and \citet[64]{pollard1994head-driven}. Pollard and
Sag give the Polish examples in (\ref{pro-dropagreement}), where the verb would be assumed to agree with a
null subject.

\ea\label{pro-dropagreement}Polish\\
\begin{tabular}[t]{lll}
  kocha\l{}em & kocha\l{}e\'{s} & kocha\l \\
  I.{\M} loved & you.{\M} loved & he loved \\
  kocha\l{}am & kocha\l{}a\'{s} & kocha\l{}a\\
  I.{\F} loved & you.{\F} loved & she loved \\
\end{tabular}
\z
%
To maintain an asymmetric view of agreement, we are essentially forced
to assume that the examples in (\ref{pro-dropagreement}) involve a multiplicity of
phonetically null pronominals, one for each distinct form of the
verb.

By contrast, on the standard LFG analysis, target features can themselves
provide information. Going back to the Latin example from
(\ref{ex:rosaspinosa}), we would get the f-structure in (\ref{ex:bloomingrosepro}) if the
subject is pro-dropped to give the simple sentence \textit{floruit} `It blooms'.

\eabox{\label{ex:bloomingrosepro}
{\avm[style=fstr]{
    [pred & `bloom\arglist{subj}'\\
    subj & [ pred & `pro'\\
      num & sg\\
      case & nom\\
      pers & 3 ]]}}
}
%
This f-structure arises directly from the f-descriptions of
\textit{floruit} in (\ref{ex:f-descriptions}) plus an optional
description (\UP\SUBJ\PRED) = \textsc{`pro'} associated with the
verb. The \textsc{num, case} and \textsc{pers} features are specified
by the target (the verb) directly, with no need for matching features
on the null subject, so that we do not need to multiply covert
elements.  Few LFG practitioners have therefore adopted an asymmetric
mechanism for matching target and controller features, although the
LFG framework offers such a mechanism in the form of constraining
equations. Nevertheless, we will see in \sectref{sec:indexconcord}
that some theories of feature indeterminacy and coordination actually
require the use of constraining equations, at least to deal with
feature resolution. More substantially, \citet{Wechsler:Mixed} has
argued that absence of controller features has grammatical
effects. This requires a deeper commitment to asymmetry. We discuss
his proposal in \sectref{sec:agreementmarking}.

While it contrasts with Minimalism in that target and controller
features are taken to be symmetric, the standard LFG treatment also
differs from an approach that is often seen in HPSG based on
structure sharing of the agreement features between the target and the
controller. In an LFG setting, we could get such an analysis e.g. by
embedding agreement features in a feature \textsc{agr} to be structure
shared between the target and the controller. This would yield the
f-structure in (\ref{ex:bloomingrosef:shared}) instead of (\ref{ex:bloomingrosef}), if we assume
that both predicate-argument agreement and NP-internal agreement
involve structure sharing.

\eabox{\label{ex:bloomingrosef:shared}
{\avm[style=fstr]{
    [ pred & `bloom\arglist{subj}'\\
      agr & \rnode{t1}{\strut}\\
      subj & [ pred & `rose'\\
        agr & \rnode{c}{[num & sg\\
                         case & nom\\
                         gend & fem\\
                         pers & 3]}\smallskip\\
        adj & \{ [ pred & `thorny'\\
                agr & \rnode{t2}{\strut} ] \} ] ] }
     \CURVE[2]{-2pt}{0}{c}{0pt}{0}{t2}
     \CURVE[2]{-2pt}{0}{c}{0pt}{0}{t1}}
}
Within the HPSG tradition, \textcite{Kathol:Agreement} argues for such
an approach. His main argument is that in many cases, target and
controller morphology is arguably ``the same'' (such as the
-\textit{a} ending in \textit{ros-a} and \textit{spinos-a}). This is
particularly common in noun phrase-internal agreement, but
occasionally happens also in predicate-argument agreement, cf. (\ref{swahili}).

\ea\label{swahili}Swahili (\citealt[ex. 14]{Kathol:Agreement}, originally from \citealt[171]{Welmers1973})
\ea\label{swahili1}
\gll\textbf{Ki}kapu \textbf{ki}kubwa \textbf{ki}moja \textbf{ki}lianguka.\\
basket large one fell\\
\glt`One large basket fell.'
\ex\label{swahili2}
\gll\textbf{Vi}kapu \textbf{vi}kubwa \textbf{vi}tatu \textbf{vi}lianguka.\\
baskets large three fell\\
\glt`Three large baskets fell.'
\z\z
%
In such cases, although the morphology is the same, it has to
contribute different functional descriptions in the various positions, because the agreement
construction is built into the equations. By contrast, if we assume
structure sharing, the mapping from morphology to functional
descriptions becomes uniform: \textit{-a} in Latin and
\textit{ki-} in Swahili always contribute their features to the
\textsc{agr} feature structure of the item where they are realized,
and agreement will be captured by requiring structure sharing of
\textsc{agr} structure in the appropriate configurations.

We can assume that all agreement works in this way, but since
morphological identity of target and controller features is much more
common in noun phrase-internal agreement, it is possible to assume
feature sharing only here and not in predicate-argument agreement. This is
illustrated in (\ref{ex:bloomingrosef:halfshared}).

\eabox{\label{ex:bloomingrosef:halfshared}
\avm[style=fstr]{
    [ pred & `bloom\arglist{subj}'\\
      subj & [ pred & `rose'\\
        agr & \rnode{c}{[num & sg\\
                         case & nom\\
                         gend & fem\\
                         pers & 3]}\smallskip\\
        adj & \{ [ pred & `thorny'\\
                agr & \rnode{t2}{\strut} ] \} ] ] }
     \CURVE[1.8]{-2pt}{0}{c}{0pt}{0}{t2}
} The \textsc{agr} feature bundle is structure shared inside the NP
but not between the verb and the NP. This is the option taken in much
HPSG work, e.g. \citet{pollard1994head-driven} and
\citet{WechslerZlatic:Agreement2003}. It is natural to connect this
difference to the \textsc{index/concord} distinction that we discuss
in \sectref{sec:indexconcord}: on that view, the \textsc{agr}
feature of (\ref{ex:bloomingrosef:halfshared}) will be split in two feature bundles,
\textsc{concord} (typically relevant for NP-internal agreement) and
\textsc{index} (typically between predicates and arguments) and we can
assume that only \textsc{concord} agreement involves structure
sharing.\footnote{Note that \citet[145]{WechslerZlatic:Agreement2003}
say that ``subject-verb agreement\dots is modeled in terms of
structure-sharing'', although it is clear from
\citet[21]{WechslerZlatic:Agreement2003} that they do not assume the
verb bears its own person and number features. I assume that
``structure sharing'' is used loosely here in the sense of cospecification of
features.}

Kathol's argument is essentially an architectural argument about how
to best capture the morphology-syntax interface. It has not been
picked up in the LFG tradition. The most explicit work on the topic,
\textcite[Chapter 12]{DLM:LFG} assumes the traditional
LFG approach and consequently postulates complex so-called m-features
(morphological features that are to be mapped to functional
descriptions). That is, a first person plural form of the verb is
associated with the m-feature in (\ref{ex:Agreement:11}).

\ea\label{ex:Agreement:11} \textsc{m-agr}:$\langle$\textsc{agr(su):\{pers:1, num:pl\}}$\rangle$
\z
%
The form, then, carries information not just about the features it
contributes (first person and plural number) but also \emph{where} it
contributes those features (in this case, to the subject). Therefore,
there cannot be a uniform representation of -\textit{a} in
\textit{ros-a} and \textit{spinos-a} (or \textit{ki-} and \textit{vi-}
in \ref{swahili}), since they contribute the same
feature to \emph{different} locations. In a structure sharing account
we \emph{can} have a uniform representation (of the relevant morphemes
or paradigmatic inferences, depending on your view of morphology),
where e.g. \textit{-a} is simply associated with nominative,
singular, feminine features and the feature sharing that forces
agreement stems from the relevant agreement construction. But as
(\ref{ex:Agreement:11}) shows, we do not \emph{need} structure sharing: we can capture
the same facts without it, but at the cost of a (slight) complication
of the morphology-syntax interface.

\hspace*{-2pt}In addition to the architectural issue, the structure sharing approach
also makes different empirical predictions in some cases, because the
same syntactic position can be simultaneously target and controller for
two different agreement processes involving the same feature and hence
give rise to so-called long distance agreement. We return to this in
\sectref{sec:domains}.

To sum up, the standard LFG treatment is symmetric but not feature-sharing: it
is based on features contributed by defining equations from
(potentially) several sources (the controller and one or more targets)
to a single syntactic position. While there has been little pressure to
change this except for special constructions, the complexities of
agreement phenomena cross-linguistically has led to expansions in many
different directions.

\section{\textsc{index, concord} and coordination}\label{sec:indexconcord} 
%
It is possible for nominal controllers to trigger different values for
the same feature on different targets, as in the Serbo-Croatian example (\ref{ex:indexconcord}) from \citet[5]{WechslerZlatic:Agreement2003}.

\ea\label{ex:indexconcord}Serbo-Croatian\\
\gll Ta dobra deca su došla.\\
that:\textsc{f;sg} good:\textsc{f;sg} children:(\textsc{f;sg}) \textsc{aux;3pl} come:\textsc{prf;ptcp;n;pl}\\
\glt `Those good children came.'
\z
Here the noun \textit{deca} `children' triggers feminine singular
agreement on the determiner and the adjective, but neuter plural
agreement on the predicate.\footnote{Note, though, that the feminine singular and the neuter plural are syncretic in Serbo-Croatian. See \textcite{Alsina:Two,Alsina:Third} and \textcite{Wechsler:Wrong} for discussion.}  Such examples require that we postulate
two different bundles of agreement features, generally called
\textsc{index} and \textsc{concord}
\citep{pollard1994head-driven,Kathol:Agreement,WechslerZlatic:Agreement2003}. Both
\textsc{index} and \textsc{concord} are syntactic features, modelled
at f-structure in LFG, but the intuition is that \textsc{index}
features are more closely related to semantics and are the ones that are
related to the reference of a noun phrase, typically \textsc{gender,
  person} and \textsc{number} (but not \textsc{case}). By contrast, \textsc{concord} features
are more closely related to morphological class and typically include
\textsc{gender, number} and \textsc{case} (but not \textsc{person}). According to
\textcite{Wechsler:Mixed} this division reflects the historical origin
of the morphology on the agreement targets, which typically comes from
incorporated pronouns in the case of \textsc{index} agreement, but
from nominal classifiers (and other sources) in the case of
\textsc{concord} agreement. \textsc{concord} and \textsc{index} are
also different in that \textsc{concord} agreement is generally found
inside NPs whereas \textsc{index} features are typically relevant to
predicate-argument agreement.

Since \textsc{gender} and \textsc{number}
are present both in \textsc{index} and \textsc{concord}, they may take
different values in those contexts and that is what happens in
(\ref{ex:indexconcord}). The f-structure for \textit{ta dobra deca} in
this example is shown in (\ref{ex:tadobradeca}).

\eabox{\label{ex:tadobradeca}
{\avm[style=fstr]{
    [concord & [ gend & fem \\
                 num & sg \\
                 case & nom ] \\
     index & [ gend & neut \\
                 num & pl\\
                 pers & 3 ]]}}
}
%
It is worth pointing out that although \textsc{index} is in some sense
“closer” to the semantics than \textsc{concord}, both are syntactic
features, represented at f-structure. In addition to these two kinds
of agreement it is necessary to postulate a third, semantic/pragmatic
kind of agreement. This is particularly common in pronoun-antecedent
agreement. For example, the Serbian/Croatian diminutive noun
\textit{devoj\v{c}e} `girl' may be referred to with a neuter pronoun
(reflecting its \textsc{index gend} feature), or with a feminine
pronoun, reflecting the meaning of its anteceedent.

Much work in LFG uses representations like (\ref{ex:bloomingrosef}) as
a simplification when the \textsc{index/concord} distinction is not
relevant, but actual work on agreement has generally assumed the
distinction. However, \textcite{Alsina:Two,Alsina:Third} argued
against having two sets of syntactic agreement features. For
counterarguments defending the \textsc{index/concord} distinction, see
\textcite{Wechsler:Wrong} and \textcite{Hristov:LFG13}.

While some words like \textit{deca} appear to be lexically specified
with dif\-ferent \textsc{index} and \textsc{concord} features, another
important motivation for the distinction comes from different
behaviour in coordinate structures. Consider (\ref{ex:Agreement:14}) from \citet[36]{belyaev-etal2015}

\ea\label{ex:Agreement:14}
This/*These man and woman are/*is eating sushi.
\z
The coordinate noun phrase in (\ref{ex:Agreement:14}) consists of two singular
nouns. The determiner must agree in singular number with each of these
nouns, whereas the predicate must agree in plural number with the
coordination as a whole. This indicates that \textsc{concord num},
relevant for NP-internal agreement, is singular, but \textsc{index
  num}, relevant for predicate agreement, is plural.

To derive this \textsc{concord/index} distinction in number,
\textcite{kingdalrymple04} proposed that \textsc{index} features are
nondistributive, i.e. they are features not just of the individual
conjuncts but also of the conjunction as a whole, based on rules of
feature resolution; whereas \textsc{concord} features are
distributive, i.e. properties of the individual conjuncts but not of
the conjunction as a whole. That is, a conjunction of two singular NPs
such as \textit{man and woman} cannot trigger a plural determiner
(\textit{*These man and woman}) because the determiner agrees in
\textsc{concord}.  However, it does trigger plural number agreement on the
verb (if it is the subject) because the conjunction as a whole has a
\textsc{num pl} feature in the \textsc{index}, different from the
singular feature of the two conjuncts, as shown in (\ref{ex:Agreement:15}).

\eabox{\label{ex:Agreement:15}
\avm[style=fstr]{
  [ index & [num & pl]\\
    \{ [ pred & `woman'\\
      concord & [num & sg]\\
      index & [num & sg ]] & 
     [ pred & `woman'\\
      concord & [num & sg]\\
      index & [num & sg ]] \} ]}
  }
This raises the question of how the features of a coordination are
related to those of the conjuncts. The distinction between
distributive and nondistributive features was originally introduced by
\citet{DalrympleKaplan2000} who used set-valued features to model both
indeterminacy and feature resolution in coordination. For example, the
\textsc{person} feature is treated in terms of sets over the atomic
markers $S$ (for ``speaker'') and $H$ (for ``hearer''). In a language
like English or Spanish, with no exclusive/inclusive distinction in
the first person plural, sets over these atoms are interpreted as in (\ref{ex:Agreement:16}).\footnote{The system of \citet{DalrympleKaplan2000} can also capture the first person exclusive as $\{S\}$ in languages where this is needed.}

\ea\label{ex:Agreement:16}
\begin{tabular}[t]{rl}
  $\{S,H\}$ & first person \\
  $\{H\}$ &  second person \\
  $\{\}$ &  third person \\
\end{tabular}
\z
On this interpretation, feature resolution corresponds to set union
and can be encoded in the phrase structure rule for coordination as in
(\ref{ex:Agreement:17}).

\ea\label{ex:Agreement:17}
\phraserule{NP}{
  \rulenode{NP\\\UP=\DOWN\\(\DOWN\ \textsc{person}) $\subseteq$ (\UP\ \textsc{person})}
  CONJ
  \rulenode{NP\\\UP=\DOWN\\(\DOWN\ \textsc{person}) $\subseteq$ (\UP\ \textsc{person})}}
\z
Because the values in (\ref{ex:Agreement:16}) are ordered by set inclusion we get a
hierarchy effect in resolution, where second and third person resolves
to second person, and first and second/third person resolves to first
person.

It is worth pointing out that this requires the target features to be
stated with a constraining equation as in the sample first person
entry in (\ref{ex:Agreement:18}).

\ea\label{ex:Agreement:18}
(\UP\ \textsc{person}) $=_c \{S,H\}$
\z
If the target features were stated constructively, as in the standard
approach, a first person verb would be compatible with the
coordination of two second person forms, because the first person from would set the \textsc{person} feature to $\{S,H\}$ and each conjunct would simply check that $\{H\}$ is a subset of that. In other words, the set-based approach requires us to give up the symmetric approach to agreement and would therefore run into similar problems with e.g. pro-drop as other asymmetric approaches to agreement, as discussed above.

Alternative accounts of feature resolution that are based on ordinary
feature structures rather than sets seem at first sight not to require
constraining equations. In particular, \citet{DKS:Indeterminacy}
suggests using ordinary LFG features to encode what would be set
membership in the analysis of \citet{DalrympleKaplan2000} and to deal with
feature indeterminacy that way. \citet{Sadler:Underspecification}
extends that approach to coordination. For example, in a language like
Icelandic, where any coordination of nouns with different genders
resolve to neuter gender, the set-based approach would assume values
as in (\ref{ex:Agreement:19}).

\ea\label{ex:Agreement:19}
\begin{tabular}[t]{rl}
  $\{M,F\}$ & neuter gender \\
  $\{M\}$ &  masculine gender \\
  $\{F\}$ &  feminine gender \\
\end{tabular}
\z
This can be translated into standard feature structures by decomposing
gender into two features, \textsc{m} and \textsc{f}, as follows.

\ea
\ea neuter gender: \avm[style=fstr]{[m & $-$\\f & $-$]}
\ex masculine gender: \avm[style=fstr]{[m & $+$\\f & $-$]}
\ex feminine gender: \avm[style=fstr]{[m & $-$\\f & $+$]}
\z \z The resolution rule will then specify that for each gender feature, if all
the conjuncts are $+$, the set is also assigned $+$; otherwise the set
is assigned $-$. However, as it turns out, stating this resolution
rule explicitly requires the use of constraining equations, namely an
implicational constraint.\footnote{See \citet[640]{DLM:LFG} for a
formalisation of the required resolution rule.} Still, the situation
is different from the set-based solution in that the equations on both
the target and on the controller conjuncts are constructive. It is
only the resolution rule that makes use of constraining equations,
suggesting that even in a declarative theory like LFG, feature
resolution requires a procedural approach:\footnote{The use of constraining equations in LFG in general has been taken to be a ``dynamic residue that resists a purely declarative analysis'' \citep[44]{blackburn1995a-specification}.} first, we construct the
conjuncts and then we can compute the features of the coordination.
On the other hand, the agreement mechanism itself does not require
constraining equations, and since the target features are still
specified constructively we do not run into problems with pro-drop.



%\section{Decomposition of agreement features}\label{sec:decomposition}
%Coordination:
%Sadler 1999, Kuhn and Sadler 1997, Dalrymple and Hristov 1997
%King and Dalrymple 2004, Belyaev et al. 2015.
%Decomposition:
%Dalrymple and Kaplan 2000, Vincent and Börjars 2007, Otoguro 2015, Sadler 2011, Arka 2011, 2012, Dalrymple 2012

\section{Factors outside the f-structure}\label{sec:outsidef}
While agreement is generally determined in terms of f-structure
relations, it is widely acknowledged that other factors are also
relevant, in particular linear order/c-structure and information
structure.

\subsection{Linear order}
That linear order can be relevant for agreement is shown by so-called
single conjunct agreement. (\ref{ex:Agreement:21}--\ref{ex:brazilian}) show some examples from \citet{KuhnSadler:LFG07}.

\ea\label{ex:Agreement:21} Czech\\
\gll Na rohožce seděla kočka a pes.\\
on mat was.sitting:\textsc{f;sg} cat:\textsc{f;sg} and dog:\textsc{m;sg}\\
\glt `The cat and the dog were sitting on the mat.'
\z
\ea\label{ex:brazilian} Portuguese\\
\gll os [mitos e lendas] brasileiras\\
the:\textsc{m;pl} \phantom{[}myth:\textsc{m;pl} and legend:\textsc{f;pl} Brazilian:\textsc{f;pl} \\
\glt `the Brazilian myths and legends'
\z
In (\ref{ex:Agreement:21}), from Czech, the predicate \textit{seděla} agrees only with the closest
subject conjunct, \textit{kočka}. In (\ref{ex:brazilian}), from Portuguese, the determiner agrees with
its closest conjunct, the first one, whereas the postposed adjective
agrees with the second conjunct, which again is the closest
one. Examples such as (\ref{ex:brazilian}) show that we cannot simply pick out a
single distinguished conjunct and make that available for agreement:
what is relevant is the distance between the target and the
controller.

\citet{KuhnSadler:LFG07} discuss earlier approaches to single conjunct
agreement and propose a solution based on dividing features into not
only the standard distributive/nondistributive classification, but
also to distinguish left-peripheral, right-peripheral and
proximity-based features. \citet{DH:Agr} dispense with the need for
dividing features this way and instead provide definitions of new
f-structure path descriptions. For example, $f_{(L)}$ is defined as in (\ref{ex:Agreement:23}).

\ea\label{ex:Agreement:23}
\begin{tabular}[t]{lllc}
  $f_{(L)}$ & $\equiv$ & $f$ & $\in^*$ \\
           &          &     & $\neg [ (\leftarrow\ \in) \prec_f\ \rightarrow] $
\end{tabular}
\z Here, $\in^*$ picks out an arbitrarily embedded member of the set
(to account for nested coordination); the Kleene star also allows zero
levels of embedding, which would make $f_{(L)}$ refer simply to
$f$. However, in case we pick a set member, it must be the leftmost
member of $f$. This is accomplished by the off-path constraint $\neg [
  (\leftarrow\ \in) \prec_f\ \rightarrow] $, which says that at any
point in the path of (potentially nested) coordinations, there must
not be other conjuncts ($\leftarrow\ \in$) that f-precede ($\prec_f$)
the one we pick ($\rightarrow$). Hence, if $f$ is
not a set, $f_{(L)}$ equals $f$, but if $f$ \emph{is} a set, $f_{(L)}$
can be either the whole set $f$ or its leftmost member. This allows
modelling of optional left conjunct agreement. We can also capture
obligatory left conjunct agreement by defining $f_{L}$ just like
$f_{(L)}$ except it can never refer to a set. (So $f_{L}$ always picks
the leftmost member of $f$.)  Similarly we can define $f_{R}$ and
$f_{(R)}$ by reversing the f-precedence relation and finally $f_C$
(closest conjunct) as $f_{L}$ if \DOWN\ f-precedes $f_{L}$ and $f_{R}$
if $f_{R}$ f-precedes \DOWN. This solution makes it possible to
describe (optional or obligatory) single conjunct agreement
irrespective of whether the relevant agreement feature(s) are
distributive or not; and it does so without altering the LFG
formalism.

Consider the f-structure for (\ref{ex:brazilian}).

\eabox{
\avm[style=fstr]{
  [ \{ [pred & `myth'\\
        concord & [num & pl \\ gend & m]] &
       [pred & `legend'\\
         concord & [num & pl \\ gend & f]] \} & {} \smallskip\\
       adj ~ \{ [ pred & `Brazilian' ] & {}\} ]}
}
This f-structure satisfies the following functional description of
\textit{brasileiras}.

\ea\begin{tabular}[t]{l}
  (({\ADJ}\ \UP)$_C$ \textsc{concord num}) = {\PL} \\
  (({\ADJ}\ \UP)$_C$ \textsc{concord gend}) = {\F} \\
\end{tabular}
\z
({\ADJ}~\UP) refers in the normal way to the f-structure of
the head, and the subscript $C$ then makes sure we select the closest
conjunct; if ({\ADJ}~\UP) was not a set, the subscript $C$ would simply have no effect.

\subsection{Information structure}
Besides c-structure/linear order, information structure is also
relevant for agreement processes in many languages, as discussed by
\citet{DN}. In their architecture, discourse functions are
modelled as features at s-structure and can be accessed from the
f-structure through the $\sigma$-projection. \citet[123]{DN}
provide the specification in (\ref{ex:Agreement:26}) of the third person singular
topical oblique agreement marker in Itelmen.

\ea\label{ex:Agreement:26}\begin{tabular}[t]{lll}
  (\UP\ \textsc{obl pers}) = 3 \\
  (\UP\ \textsc{obl num}) = \textsc{sg} \\
  ((\UP\ \textsc{obl})$_\sigma$\ \textsc{df}) = \textsc{topic}
\end{tabular}
\z
More complicated patterns are also possible. Object agreement in
Itelmen is only optionally an indicator of the topicality of the
object, but it does indicate that there is no oblique topic. This is
captured by the description in (\ref{ex:Agreement:27}) of the first person singular
object agreement marker.

\ea\label{ex:Agreement:27}\begin{tabular}[t]{lll}
  (\UP\ \textsc{obj pers}) = 1 \\
  (\UP\ \textsc{obj num}) = \textsc{sg} \\
  $\neg$[((\UP\ \textsc{obl})$_\sigma$\ \textsc{df}) = \textsc{topic}] \\
  (((\UP\ \textsc{obj})$_\sigma$\ \textsc{df}) = \textsc{topic})
\end{tabular}
\z
In addition to precedence and information structure role, LFG analyses
have shown that agreement can be sensitive to other factors such as
adjacency (direct precedence) and various prominence hierarchies based
on person and grammatical functions. \citet{BroadwellEtAl11} and
\citet{belyaev2013-agr} analyse such patterns in Kaqchikel and Dargwa
respectively and show how they be captured with LFG augmented with
Optimality Theory (OT).

\section{Diachrony: grammatical and anaphoric agreement}\label{sec:diachrony}
It is a long-standing observation from comparative linguistics
\citep{Bopp:Vergleichende} that agreement markers in
predicate-argument structures (i.e. \textsc{index} agreement) arise
from incorporated pronouns. That is, we have an evolution from
anaphoric agreement with a dislocated noun phrase (\emph{The man, he
came}) to grammatical agreement (\emph{The man he-came}). As pointed out
by \citet{BM87}, LFG is well placed to capture this
development because (unlike what happens in many other formal
frameworks), pronouns and agreement markers are very similar, yet also
distinct in a way which generates clear predictions about differences
between anaphoric and grammatical agreement. In particular,
incorporated pronouns always introduce a semantic form (\textsc{pred
  `pro'}), while agreement markers do not introduce a semantic form or
do so only optionally (if the language allows pro-drop). Otherwise,
both agreement markers and incorporated pronouns introduce the
relevant agreement features. \citet{BM87} argue that in
Chichewa, subject agreement is grammatical and obligatory whereas
object agreement is anaphoric and optional. They represent subject
markers (SM) and object markers (OM) with the lexical entries in
(\ref{ex:Agreement:28}).\footnote{We adopt the convention of treating sublexical units
  such as the subject and object marker as if they were nodes in a
  syntactic tree, with \DOWN\ designating their own f-structure and
  \UP\ that of the lexical item they attach to, as is done also in the
  presentation in \citet{BresnanEtAl2016}.}

\ea\label{ex:Agreement:28}
\begin{tabular}[t]{ll}
  SM- & (\UP\ \textsc{subj}) = \DOWN \\
     & (\DOWN\ \textsc{index}) = $\alpha$ \\
     & ((\DOWN\ \textsc{pred}) = `pro') \\\hline
  OM- & (\UP\ \textsc{obj}) = \DOWN \\
     & (\DOWN\ \textsc{index}) = $\alpha$ \\
     & (\DOWN\ \textsc{pred}) = `pro' \\
\end{tabular}
\z
From a diachronic point of view, the subject marker and the object
marker reflect different points on a grammaticalization path from
pronouns to agreement morphology: the object marker has lost its
c-structure independence, but is still in all respects a pronoun at
f-structure, contributing its own \textsc{pred} value. The subject
marker has evolved one step further in that the \textsc{pred} value
contribution has become optional. There is a clear connection between
the formal representations at the two stages, and the relation between
them fits well with the intuitive notion of “bleaching” or “loss of
content” in grammaticalization processes.

At the same time, the subtle difference between the two
representations, along with some other independent properties of
Chichewa, suffice to predict a number of differences between subject
and object agreement.  First, because the Chichewa
sentence structure consists of a subject NP, a head-initial VP and a
topic NP (in any order), the NP object must appear directly after the verb
(i.e. inside the VP) whenever there is no object marker. When there is an object marker, however, that marker is the actual object, whereas the apparent NP object is an anaphorically linked topic, which can therefore appear anywhere in the clause.

Second, because the object marker is a light (i.e.\ incorporated)
anaphoric pronoun, it blocks the use of the independent pronoun in
this function, with the effect that the independent object pronoun is
reserved for cases of focus and contrastive topics.\footnote{Though as
a reviewer remarks, this blocking effect is not formalized in
\citet{BM87}.} No such effect is found with the subject marker. Third,
objects can be questioned in situ but only when there is no object
marker.  All these predictions are borne out in Chichewa.

\hspace*{-.5mm}In sum, the LFG framework makes it possible to understand fundamental
differences between grammatical agreement with governed functions and
anaphoric agreement with discourse functions, while at the same time
providing a plausible diachronic pathway from the latter to the
former, in line with what we observe in language change. Notice that
the analysis relies crucially on treating the subject marker as
ambiguous between a true pronoun (with a \textsc{pred `pro'} feature)
and an agreement marker (without it). This holds for LFG analyses of
pro-drop generally. \textcite{Toivonen:FinnPoss} provides motivation
for this kind of “lexical split” analysis by pointing to the case of
Finnish possessives, where the agreement marker and the suffixal
pronoun differ in other features as well.  For more on the LFG
analysis of pro-drop, see \citetv{chapters/Incorporation}.




\section{A role for feature sharing? -- Agreement domains}\label{sec:domains}
In line with the general philosophy of LFG, the formalism itself does
not in any way constrain how agreement domains are defined. We could
easily write constraints that would enforce purely linear agreement
(e.g.\ agree with closest NP irrespective of grammatical function) or
agreement across unbounded domains (e.g.\ agree with \textsc{comp*
  subj}). An advantage of this is that LFG has no problems capturing
surprising agreement relations such as those found in Archi, where
agreement targets include a mixed bag of a number of first person
forms, some adverbial elements, an emphatic particle and one
postposition, which all agree with the absolutive element in their clause. (\ref{ex:Agreement:29})
shows how the lexical entry for the first person dative pronoun looks
according to \citet{sadler16}, assuming the absolutive argument
bears the grammatical function \textsc{piv}.

\ea\label{ex:Agreement:29}
\lexentry{d-ez}{
   (\UP\ \textsc{pred}) = `\textsc{pro}'  \\
   (\UP\ \textsc{num}) = \textsc{sg}  \\
   (\UP\ \textsc{pers}) = \textsc{1}  \\
   (\UP\ \textsc{case}) = \textsc{dat}  \\
   ((\textsc{PathOut} \UP) \textsc{piv gend}) = \textsc{ii} \\
   ((\textsc{PathOut} \UP) \textsc{piv num}) = \textsc{sg}}
\z
That is, the first person dative pronoun agrees with a \textsc{piv}
argument that is found by first going up \textsc{PathOut}, which is
defined as $\{\textsc{subj}|\textsc{obj}|\textsc{obl}|\textsc{obl
  obj}\}$. (\ref{ex:Agreement:30}) shows an example where a first person pronoun
embedded in PP (\textsc{obl obj}) agrees with the absolutive.

\ea\label{ex:Agreement:30} Archi\\
\gll d-ez χir d-e<r>qˁa-r-ši d-i \\
\textsc{ii.sg}-\textsc{1sg.dat} behind \textsc{ii.sg}-<\textsc{ipfv}>go-\textsc{ipfv}-\textsc{cvb} \textsc{ii.sg}-be.\textsc{pres} \\
\glt `She goes after me.'
\z
%
The first person dative pronoun bears the noun class \textsc{ii}
(essentially human feminine) marker \textit{d-} because it agrees with
the absolutive argument \textit{she} (only expressed through agreement
on the verb), irrespective of the gender of the speaker. The equation
((\textsc{obl obj} \UP) \textsc{piv gend}) = \textsc{ii} captures
that. But the use of inside-out functional uncertainties may be
problematic in cases where it does not refer uniquely because of
structure sharing. More work is needed on this kind of complex
agreement paths.

The approach of \citet{sadler16} can in principle be extended with
paths that cross clausal boundaries (so-called long distance
agreement). However, the fact that we \emph{can} write such equations does not mean
that we \emph{should}. Locality of grammatical processes remains an important
theoretical concern in LFG even if it is not hardwired into the
formalism. \citet{haug-nikitina2015} argue that several cases of
so-called long distance agreement can be given a local treatment if
the agreement process is assumed to be structure-sharing. Their main
example concerns the so-called ``dominant participle'' construction in
Latin,\footnote{\citet{haug-nikitina2015} also argue that the same analysis may work for long distance agreement in Tsez, Passamaquoddy and Innu-Aimûn, which has been widely discussed in the generative literature \citep{Branigan:Altruism,Bruening:PhD,Polinsky2001:Agreement}.} where a noun and a participle form a non-finite clause which is
headed by the participle but bears the agreement features of the noun.

\newpage
\ea\label{subjectdom2} Latin\\
\gll ne eum Lentulus et Cethegus {\dots} deprehensi terrerent\\
lest him:{\ACC} L.:\textsc{nom;m} and C.:\textsc{nom;m} {} captured:\textsc{nom;m;pl} frighten:\textsc{impf;subj;3pl}\\
\glt`lest the capture of Lentulus and Cethegus should frighten him' (Sall., Cat 48.4)
\z
According to the analysis in \citet{haug-nikitina2015},
\textit{Lentulus et Cethegus\dots de\-pre\-hensi} (`that Lentulus and
Cethegus were captured') is a clause which acts as the subject of the
matrix verb \textit{terrerent}. Yet unlike other clausal subjects in
Latin, it does not trigger default third person singular agreement on the
predicate. Instead, the matrix verb is plural, meaning that it either
agrees with the embedded subject \textit{Lentulus et Cethegus}, or the
plural feature of the embedded subject has somehow been transferred
to the predicate \textit{deprehensi}. \textit{Deprehensi} does bear
morphological plural marking, but on the standard, non-feature sharing
approach to agreement this feature would only be active in the 
subject (controller) position. If instead we suppose that features
in this kind of agreement are active in both the target and the
controller, the target may in turn serve as the controller for another
agreement process with the matrix verb as the target. This yields the
f-structure in (\ref{ex:Agreement:32}).

\eabox{\label{ex:Agreement:32}
\avm[style=fstr]{
  [ pred & `frighten\arglist{subj,obj}'\\
    obj & [ pred & `pro']\\
    subj & [ pred & `be.captured\arglist{subj}'\\
             agr & \rnode{t1}{\strut}\\
             subj & [ pred & `L.-and-C.'\\
               agr & \rnode{t2}{\strut} ] ]]}
\CURVE[3]{0pt}{0}{t2}{0pt}{0}{t1}
}
Structure sharing agreement between \textit{Lentulus et Cethegus} and
\textit{deprehensi} makes the agreement features available in the f-structure
which is \textsc{subj agr} relative to the matrix verb, so that there
can be normal predicate--subject agreement in the matrix clause. In
principle, that agreement could also be structure sharing, but as the
apparent long-distance agreement can only be positively demonstrated
in participial clauses, \citet{haug-nikitina2015} remain agnostic
on the matter. However, a similar feature-sharing account of agreement
was extended to finite verb agreement by
\citet{AV:LFG14,AlsinaVigo17}. Interestingly, their arguments for
adopting structure sharing are different: in some cases, such as
copular inversion in Catalan and raising constructions in Icelandic,
the controller cannot be specified lexically, but is determined by OT
constraints over the global f-structure. This, they hold, argues for a
view that targets and controllers lexically specify features of their
own \textsc{agr} and then OT constraints decide which \textsc{agr} structures should be linked to each other. Finally, a feature sharing approach to agreement is also adopted by \citet{Sadler:NomMod} to account for an adjectival construction in Modern Standard Arabic where the target adjective agrees with two distinct controllers.

\section{A challenge to symmetry: The Agreement Marking Principle}\label{sec:agreementmarking}
\citet{Wechsler:Mixed} proposes the principle in (\ref{ex:Agreement:33}), called the \textit{Agreement Marking Principle}.

\ea\label{ex:Agreement:33}\label{ex:AgreementMarking}
Agreement is driven by a syntactic feature of the controller, if
the controller has such a feature. If the controller lacks such a
feature, then the target agreement inflection is semantically
interpreted as characterizing the controller denotation.
\z
With this principle, Wechsler seeks to explain so-called mixed
agreement, i.e.\ cases where a polite plural pronoun triggers plural
agreement on the verb, but singular agreement on some other target,
e.g.\ a predicative adjective as in (\ref{ex:Agreement:34}) from French.

\ea\label{ex:Agreement:34} French\\
\gll Vous \^etes loyal.\\
you.\textsc{pl} are.\textsc{2pl} loyal.\textsc{m.sg}\\
\glt `You (singular, formal, male) are loyal.'
\z
This pattern follows from the Agreement Marking Principle on the
assumption that \textit{vous} bears an \textsc{index num pl} feature
that is able control \textsc{index} agreement on the verb, but
\emph{no} \textsc{concord num} feature, which leaves the predicative
adjective without an agreement controller, thereby licensing semantic
agreement. Moreover, the Agreement Marking Principle gives us an
\emph{explanation} of the so-called “polite plural generalization”,
that there are no languages\footnote{See \textcite[Section
    2.1]{Wechsler:Mixed} for the typological data.}  with the opposite
pattern, i.e. where the polite plural pronoun triggers plural
agreement on the adjective but allows singular agreement on the verb,
or more generally, following Wechsler, on any target that has the
\textsc{person} feature. This polite plural generalization follows
because pronouns by necessity have \textsc{index} features and any
\textsc{person} target must be an \textsc{index} target.

Formalizing the Agreement Marking Principle requires use of
constraining equations. Wechsler's analysis of the French feminine
definite article \textit{la} is given in (\ref{ex:Agreement:35}), where
\textbf{female}(\UPS) is a simplified representation for the
relevant semantic resource that will ensure that the referent is
interpreted as female.

\ea\label{ex:Agreement:35}\begin{tabular}{ll}
  \textit{la} & (\UP\ \textsc{gend})$=_c$ \textsc{f} $\lor$ [\textbf{female}(\UPS) $\land \neg$\,(\UP\ \textsc{gend})]
\end{tabular}
\z
%
The idea is that when \textit{la} combines with a noun that is
lexically specified as feminine gender, such as \textit{sentinelle}
`sentry', the feminine feature is \emph{not} semantically interpreted;
but when it combines with a noun that does not have a gender feature,
such as \textit{professeur}, it \emph{will} be interpreted semantically.
However, this entails a move away from the traditional symmetric approach to
agreement in LFG to the asymmetric approach associated with
derivational syntax.

As pointed out by Wechsler, the Agreement Marking Principle is not in
itself a descriptive generalization, since the presence versus absence
of a given agreement feature on the controller NP is not always directly
observable, but rather depends upon the grammatical analysis of the
NP. However, the radically symmetric nature of the standard LFG
analysis allows for cases where there is \emph{no controller} NP at all. This
is what we saw in the standard analysis of \textit{floruit} in
(\ref{ex:bloomingrosepro}). The lexical entry of the verb on the standard analysis will be as in (\ref{ex:Agreement:36}).

\ea\label{ex:Agreement:36}
\lexentry{floruit}{(\UP\ \textsc{pred}) = \textsc{`bloom$<$subj$>$'} \\
                (\UP\ \textsc{subj case}) = \textsc{nom} \\
                (\UP\ \textsc{subj num}) = \textsc{sg} \\
                (\UP\ \textsc{subj pers}) = \textsc{3} \\
                ((\UP\ \textsc{subj pred}) = \textsc{`pro'})}
\z
On the traditional LFG analysis, which also underlies the diachronic analysis of
anaphoric agreement discussed in \sectref{sec:diachrony}, there
simply \emph{is} no controller: it is constructed by the target. If we
change (\ref{ex:Agreement:36}) to interpret the number and person agreement along the
lines of the Agreement Marking Principle, we get (\ref{ex:Agreement:37}), where
\textbf{\textsc{non-participant}}(\UPS) is shorthand for some semantic
resource that ensures the subject referent is distinct from the
discourse participants (speaker or hearer).

\eabox{\label{ex:Agreement:37}
\lexentry{floruit}{(\UP\ \textsc{pred}) = \textsc{`bloom\arglist{subj}'} \\
(\UP\SUBJ\CASE) = \textsc{nom} \\
(\UP\SUBJ\PERS)$=_c$ \textsc{3} $\lor$ [\textbf{non-participant}(\UPS) $\land \neg$\,(\UP\SUBJ\PERS)]\\
(\UP\SUBJ\NUM)$=_c$ \textsc{sg} $\lor$ [\textbf{non-participant}(\UPS) $\land \neg$\,(\UP\SUBJ\NUM)] \\
((\UP\SUBJ\PRED) = \textsc{`pro'})}
}%
If we want to maintain the Agreement Marking Principle there are a
number of ways we can go.  First, we can take (\ref{ex:Agreement:37}) at face value and
assume that since there is no controller, the agreement features are
interpreted semantically. This would yield the prediction that in
pro-drop structures, agreement features are always semantically
interpreted, which is a strong and quite probably false
assumption.\footnote{In fact, a reviewer offers a counterexample from Spanish, where second person plural forms can be used for very elevated addressees in a very formal register and crucially the interpretation does not change whether the subject is expressed by means of
the pronoun \textit{vos} or is null:

\ea Spanish\\
\gll (Vos) sois muy bondadoso.\\
you.\textsc{pl} are.\textsc{2.pl} very kind.\textsc{m.sg}\\
\glt `You (singular, formal, male) are very kind'
\z
} Second, we can exploit the fact that the LFG formalism
cannot faithfully express the Agreement Marking Principle as
formulated in (\ref{ex:AgreementMarking}). (\ref{ex:AgreementMarking})
says that agreement in some feature is syntactic, ``if the controller
has such a feature''. However, the LFG formalism offers no way of
checking where a feature originates. Constraining equations check
whether some feature is present in the minimal solution to the
f-description, irrespective of where they originate. Therefore, we can
add the constructive equations (\UP\ \textsc{subj person}) = 3 and
(\UP\ \textsc{subj num}) = \textsc{sg} to the optional part of
(\ref{ex:Agreement:37}). This preserves the formalization of the Agreement Marking
Principle, but arguably not its spirit, since the same lexical item
provides both target and controller features. Finally, we could
envisage a c-structure controller (with the appropriate features) in
pro-drop structures, although this seems at odds with all standard
assumptions of LFG.

In sum, it is not clear how to best integrate the Agreement Marking
Principle in LFG. More generally, symmetry between target and
controller features does important work in LFG's traditional theory of agreement
and it requires substantial work to alter this fundamental setup.


\section{Agreement and semantics}
A general question which has not received much attention in the LFG
literature concerns how f-structure agreement features relate to the
semantic content that they (sometimes) encode. In the standard LFG
architecture, levels of linguistic description as found in the
projection architecture are related by codescription, where
linguistic items simultaneously describe different structures,
including syntax and semantics. For example, the lexical entry for a
singular noun might look like (\ref{ex:Agreement:38}), where $1(x)$ is a cardinality
test on the referent.


\ea\label{ex:Agreement:38}
\lexentry{horse}{(\UP\ \textsc{pred}) = \textsc{`horse'} \\
(\UP\ \textsc{index num}) = \textsc{sg} \\
(\UP\ \textsc{concord num}) = \textsc{sg} \\
$\lambda x.\IT{horse}^*(x) \land 1(x): v \multimap r$}
\z
This lexical entry simultaneously specifies syntactic singular number
(in the form of f-structure features) and semantic singular number
(simplified as a cardinality check on $x$). On the alternative, so-called
“description-by-analysis” approach \citep{halvorsen83}, semantics is
not cospecified together with syntax, but is instead read off the
constructed f-structure.

Although codescription is the standard, \citet{andrews2008} points to
two problems for this approach, both having to do with agreement. The
first and most obvious problem is that in lexical entries like (\ref{ex:Agreement:38}),
there is no necessary connection between the syntactic and semantic
singular number features: yet outside the limited class of
\textit{pluralia tantum} these are closely connected in a way we would
predict more clearly if we simply had semantics read the f-structure
features. There is to my knowledge no theory of how this connection
would work in a codescription approach, but it seems conceivable that
the morphology-syntax interface developed in \textcite[Chapter
  12]{DLM:LFG} could also take care of the interface
with semantics and restrict the mappings in a principled way.

The second problem for codescription, according to
\citet{andrews2008}, is that it creates the need to decide which of
the various lexical entries introducing a given feature-value
occurrence is the one that is introducing the semantic
constructor. This again relates to the question of symmetry or not
between target and controller features. Andrews considers an Italian example with possible pro-drop (\ref{ex:Agreement:39}).

\ea\label{ex:Agreement:39} Italian\\
\gll(le ragazze) vengono\\
\phantom{(}the.\textsc{fem.pl} girls.\textsc{fem.pl} come.\textsc{3pl}\\
\glt`The girls/they are coming.'
\z
If the subject is present, we presumably want the noun to introduce
the plural meaning constructor and the verb not to, but if the subject
is omitted, then the verb presumably provides the
constructor. However, we already need to make sure that the
\textsc{pred} feature of the subject is instantiated only once, so it
is not clear that this is a deep problem, although as Andrews points
out, it does open the door to some stipulation.

NP-internal agreement raises more tricky problems. As discussed by
\citet{belyaev-etal2015}, there are languages where a plural head noun can
take two coordinated singular adjectives as modifiers, as in (\ref{resolvingAgreement})
from Russian.

\ea\label{resolvingAgreement}Russian\\
\gll 
krasnyj i belyj flagi\\
red.\textsc{sg} and white.\textsc{sg} flag.\textsc{pl} \\
\glt`(the) red and (the) white flags' [2 flags total: one red, one white]
\z
\citet{belyaev-etal2015} call this pattern “resolving agreement”. On their
analysis, it has the f-structure in (\ref{ex:Agreement:41}).\footnote{See \citet{belyaev-etal2015} for the details of how this f-structure arises. In short, the relevant rule for adjective coordination creates two incomplete (\textsc{pred}-less) NPs, to which each adjective contributes their \textsc{concord} features, including singular number. The \textsc{pred} feature originating in the noun is distributive and gets copied into each conjunct.} Notice that this treats \textsc{concord} as non-distributive; according to \citet{belyaev-etal2015} the distributivity of \textsc{concord} is subject to variation across languages, and even across different constructions within particular languages.

\eabox{\label{ex:Agreement:41}
\avm[style=fstr]{
  [ concord ~ [num & pl] & {}\\
    index ~  [num & pl] & {} \\
    conj ~ and & {}\\
    \{ [ pred & `flag'\\
         concord & [num & sg]\\
         adj & \{ [pred & `white'] \} ] &
       [ pred & `flag'\\
         concord & [num & sg]\\
         adj & \{ [pred & `red'] \} ] \} & {} ]}
}
\citet{belyaev-etal2015} do not offer an explicit semantics in their account,
but it is clear that we will have to interpret agreement features from
the target (the adjectives) one way or another. Notice that the
analysis does not provide an \textsc{index num sg} feature on the
conjuncts and it would not be trivial to get that. So on a description
by analysis approach, we need to interpret the \textsc{concord num sg}
features of the conjuncts, although \textsc{concord} features are
normally understood as meaningless. The (\textsc{index}) \textsc{num
  pl} feature of the whole noun phrase would be superfluous but not
harmful, just like in other cases of group formation from two singular
nouns.

On a codescription approach, we cannot directly exploit the fact that there are
two singular flags in the f-structure in (\ref{ex:Agreement:41}). Instead it seems
likely that the lexical entry of the singular adjectives themselves
will introduce singular number constraints. The special phrase
structure rule for resolving agreement might also play a role in
constraining when an adjective's number feature is interpreted, to avoid
problems of interpreting adjective number features when they agree
with e.g. a \textit{plurale tantum}.

We cannot address this issue in further detail here, but we can
conclude that in one way or another, the morphological singular
feature that occurs on the adjectives in (\ref{resolvingAgreement}) will have
to be interpreted. Although details remain unclear, this supports the
general symmetric approach to agreement in LFG.

\section{Summary}
We have seen that the standard treatment of agreement in LFG relies heavily on
unification: the controller and the target co-specify a piece of
functional structure. There is therefore symmetry between controller
and target features, as both contribute grammatical information on an
equal footing. On the other hand, the piece of functional structure
that is co-specified is usually found only in the syntactic position of the
controller (except when feature sharing is assumed), accounting for the directed nature of agreement. To
account for certain phenomena in coordination and with special lexical
items, it has proven necessary to operate with two such positions
(f-structure features), \textsc{index} and
\textsc{concord}. While the phenomenon of agreement is thus handled at
f-structure, the projection architecture makes it possible to model
interactions with other aspects of grammatical structure, notably
c-structure and information structure, as has proven necessary for
several phenomena.

The symmetric but not feature sharing theory of agreement has proven
successful for example in accounting for the diachrony of agreement
marking. Nevertheless, there are some constructions that seem to
suggest modifications of the basic framework: long distance agreement
across clause boundaries can be analyzed as local agreement if we
allow structure sharing at least for (some) instances of
\textsc{concord} agreement, whereas Wechsler's Agreement Marking
Principle suggests that target and controller features are not
symmetric. On the other hand, the semantic contribution that target
features sometimes make seem to support the traditional, symmetric
analysis.


\section*{Acknowledgements}
I thank Mary Dalrymple and the three reviewers for very helpful feedback on earlier versions of this chapter.

\sloppy
\printbibliography[heading=subbibliography,notkeyword=this]

\end{document}
