\documentclass[output=paper,hidelinks]{langscibook}
\ChapterDOI{10.5281/zenodo.10185952}
\title{Clause structure and configurationality}
\author{Avery D. Andrews\affiliation{The Australian National University}}
\abstract{LFG differs strongly from Mainstream Generative Grammar in basing its theory of clause structure on overt surface appearance, as would be input to a parser, rather than the outputs of a derivational process that might produce these structures.  This leads to a number of differences, such as a much smaller number of functional projections, and more emphasis on a typology of overt structures, including the inclusion of special provisions for `non-configurationality'. In this chapter, we examine LFG analyses resulting from this perspective from the beginning of LFG in \citet{bresnan82} through to the theory as presented in \citet{BresnanEtAl2016}.}
\IfFileExists{../localcommands.tex}{
   \addbibresource{../localbibliography.bib}
   \addbibresource{thisvolume.bib}
   % add all extra packages you need to load to this file

\usepackage{tabularx}
\usepackage{multicol}
\usepackage{url}
\urlstyle{same}
%\usepackage{amsmath,amssymb}

% Tight underlining according to https://alexwlchan.net/2017/10/latex-underlines/
\usepackage{contour}
\usepackage[normalem]{ulem}
\renewcommand{\ULdepth}{1.8pt}
\contourlength{0.8pt}
\newcommand{\tightuline}[1]{%
  \uline{\phantom{#1}}%
  \llap{\contour{white}{#1}}}
  
\usepackage{listings}
\lstset{basicstyle=\ttfamily,tabsize=2,breaklines=true}

% \usepackage{langsci-basic}
\usepackage{langsci-optional}
\usepackage[danger]{langsci-lgr}
\usepackage{langsci-gb4e}
%\usepackage{langsci-linguex}
%\usepackage{langsci-forest-setup}
\usepackage[tikz]{langsci-avm} % added tikz flag, 29 July 21
% \usepackage{langsci-textipa}

\usepackage[linguistics,edges]{forest}
\usepackage{tikz-qtree}
\usetikzlibrary{positioning, tikzmark, arrows.meta, calc, matrix, shapes.symbols}
\usetikzlibrary{arrows, arrows.meta, shapes, chains, decorations.text}

%%%%%%%%%%%%%%%%%%%%% Packages for all chapters

% arrows and lines between structures
\usepackage{pst-node}

% lfg attributes and values, lines (relies on pst-node), lexical entries, phrase structure rules
\usepackage{packages/lfg-abbrevs}

% subfigures
\usepackage{subcaption}

% macros for small illustrations in the glossary
\usepackage{./packages/picins}

%%%%%%%%%%%%%%%%%%%%% Packages from contributors

% % Simpler Syntax packages
\usepackage{bm}
\tikzstyle{block} = [rectangle, draw, text width=5em, text centered, minimum height=3em]
\tikzstyle{line} = [draw, thick, -latex']

% Dependency packages
\usepackage{tikz-dependency}
%\usepackage{sdrt}

\usepackage{soul}

\usepackage[notipa]{ot-tableau}

% Historical
\usepackage{stackengine}
\usepackage{bigdelim}

% Morphology
\usepackage{./packages/prooftree}
\usepackage{arydshln}
\usepackage{stmaryrd}

% TAG
\usepackage{pbox}

\usepackage{langsci-branding}

   % %%%%%%%%% lang sci press commands

\newcommand*{\orcid}{}

\makeatletter
\let\thetitle\@title
\let\theauthor\@author
\makeatother

\newcommand{\togglepaper}[1][0]{
   \bibliography{../localbibliography}
   \papernote{\scriptsize\normalfont
     \theauthor.
     \titleTemp.
     To appear in:
     Dalrymple, Mary (ed.).
     Handbook of Lexical Functional Grammar.
     Berlin: Language Science Press. [preliminary page numbering]
   }
   \pagenumbering{roman}
   \setcounter{chapter}{#1}
   \addtocounter{chapter}{-1}
}

\DeclareOldFontCommand{\rm}{\normalfont\rmfamily}{\mathrm}
\DeclareOldFontCommand{\sf}{\normalfont\sffamily}{\mathsf}
\DeclareOldFontCommand{\tt}{\normalfont\ttfamily}{\mathtt}
\DeclareOldFontCommand{\bf}{\normalfont\bfseries}{\mathbf}
\DeclareOldFontCommand{\it}{\normalfont\itshape}{\mathit}
\makeatletter
\DeclareOldFontCommand{\sc}{\normalfont\scshape}{\@nomath\sc}
\makeatother

% Bug fix, 3 April 2021
\SetupAffiliations{output in groups = false,
                   separator between two = {\bigskip\\},
                   separator between multiple = {\bigskip\\},
                   separator between final two = {\bigskip\\}
                   }

% commands for all chapters
\setmathfont{LibertinusMath-Additions.otf}[range="22B8]

% punctuation between a sequence of years in a citation
% OLD: \renewcommand{\compcitedelim}{\multicitedelim}
\renewcommand{\compcitedelim}{\addcomma\space}

% \citegen with no parentheses around year
\providecommand{\citegenalt}[2][]{\citeauthor{#2}'s \citeyear*[#1]{#2}}

% avms with plain font, using langsci-avm package
\avmdefinestyle{plain}{attributes=\normalfont,values=\normalfont,types=\normalfont,extraskip=0.2em}
% avms with attributes and values in small caps, using langsci-avm package
\avmdefinestyle{fstr}{attributes=\scshape,values=\scshape,extraskip=0.2em}
% avms with attributes in small caps, values in plain font (from peter sells)
\avmdefinestyle{fstr-ps}{attributes=\scshape,values=\normalfont,extraskip=0.2em}

% reference to previous or following examples, from Stefan
%(\mex{1}) is like \next, referring to the next example
%(\mex{0}) is like \last, referring to the previous example, etc
\makeatletter
\newcommand{\mex}[1]{\the\numexpr\c@equation+#1\relax}
\makeatother

% do not add xspace before these
\xspaceaddexceptions{1234=|*\}\restrict\,}

% Several chapters use evnup -- this is verbatim from lingmacros.sty
\makeatletter
\def\evnup{\@ifnextchar[{\@evnup}{\@evnup[0pt]}}
\def\@evnup[#1]#2{\setbox1=\hbox{#2}%
\dimen1=\ht1 \advance\dimen1 by -.5\baselineskip%
\advance\dimen1 by -#1%
\leavevmode\lower\dimen1\box1}
\makeatother

% Centered entries in tables.  Requires array package.
\newcolumntype{P}[1]{>{\centering\arraybackslash}p{#1}}

% Reference to multiple figures, requested by Victoria Rosen
\newcommand{\figsref}[2]{Figures~\ref{#1}~and~\ref{#2}}
\newcommand{\figsrefthree}[3]{Figures~\ref{#1},~\ref{#2}~and~\ref{#3}}
\newcommand{\figsreffour}[4]{Figures~\ref{#1},~\ref{#2},~\ref{#3}~and~\ref{#4}}
\newcommand{\figsreffive}[5]{Figures~\ref{#1},~\ref{#2},~\ref{#3},~\ref{#4}~and~\ref{#5}}

% Semitic chapter:
\providecommand{\textchi}{χ}

% Prosody chapter
\makeatletter
\providecommand{\leftleadsto}{%
  \mathrel{\mathpalette\reflect@squig\relax}%
}
\newcommand{\reflect@squig}[2]{%
  \reflectbox{$\m@th#1$$\leadsto$}%
}
\makeatother
\newcommand\myrotaL[1]{\mathrel{\rotatebox[origin=c]{#1}{$\leadsto$}}}
\newcommand\Prosleftarrow{\myrotaL{-135}}
\newcommand\myrotaR[1]{\mathrel{\rotatebox[origin=c]{#1}{$\leftleadsto$}}}
\newcommand\Prosrightarrow{\myrotaR{135}}

% Core Concepts chapter
\newcommand{\anterm}[2]{#1\\#2}
\newcommand{\annode}[2]{#1\\#2}

% HPSG chapter
\newcommand{\HPSGphon}[1]{〈#1〉}
% for defining RSRL relations:
\newcommand{\HPSGsfl}{\enskip\ensuremath{\stackrel{\forall{}}{\Longleftarrow{}}}\enskip}
% AVM commands, valid only inside \avm{}
\avmdefinecommand {phon}[phon] { attributes=\itshape } % define a new \phon command
% Forest Set-up
\forestset
  {notin label above/.style={edge label={node[midway,sloped,above,inner sep=0pt]{\strut$\ni$}}},
    notin label below/.style={edge label={node[midway,sloped,below,inner sep=0pt]{\strut$\ni$}}},
  }

% Dependency chapter
\newcommand{\ua}{\ensuremath{\uparrow}}
\newcommand{\da}{\ensuremath{\downarrow}}
\forestset{
  dg edges/.style={for tree={parent anchor=south, child anchor=north,align=center,base=bottom},
                 where n children=0{tier=word,edge=dotted,calign with current edge}{}
                },
dg transfer/.style={edge path={\noexpand\path[\forestoption{edge}, rounded corners=3pt]
    % the line downwards
    (!u.parent anchor)-- +($(0,-l)-(0,4pt)$)-- +($(12pt,-l)-(0,4pt)$)
    % the horizontal line
    ($(!p.north west)+(0,l)-(0,20pt)$)--($(.north east)+(0,l)-(0,20pt)$)\forestoption{edge label};},!p.edge'={}},
% for Tesniere-style junctions
dg junction/.style={no edge, tikz+={\draw (!p.east)--(!.west) (.east)--(!n.west);}    }
}


% Glossary
\makeatletter % does not work with \newcommand
\def\namedlabel#1#2{\begingroup
   \def\@currentlabel{#2}%
   \phantomsection\label{#1}\endgroup
}
\makeatother


\renewcommand{\textopeno}{ɔ}
\providecommand{\textepsilon}{ɛ}

\renewcommand{\textbari}{ɨ}
\renewcommand{\textbaru}{ʉ}
\newcommand{\acutetextbari}{í̵}
\renewcommand{\textlyoghlig}{ɮ}
\renewcommand{\textdyoghlig}{ʤ}
\renewcommand{\textschwa}{ə}
\renewcommand{\textprimstress}{ˈ}
\newcommand{\texteng}{ŋ}
\renewcommand{\textbeltl}{ɬ}
\newcommand{\textramshorns}{ɤ}

\newbool{bookcompile}
\booltrue{bookcompile}
\newcommand{\bookorchapter}[2]{\ifbool{bookcompile}{#1}{#2}}




\renewcommand{\textsci}{ɪ}
\renewcommand{\textturnscripta}{ɒ}

\renewcommand{\textscripta}{ɑ}
\renewcommand{\textteshlig}{ʧ}
\providecommand{\textupsilon}{υ}
\renewcommand{\textyogh}{ʒ}
\newcommand{\textpolhook}{̨}

\renewcommand{\sectref}[1]{Section~\ref{#1}}

%\KOMAoptions{chapterprefix=true}

\renewcommand{\textturnv}{ʌ}
\renewcommand{\textrevepsilon}{ɜ}
\renewcommand{\textsecstress}{ˌ}
\renewcommand{\textscriptv}{ʋ}
\renewcommand{\textglotstop}{ʔ}
\renewcommand{\textrevglotstop}{ʕ}
%\newcommand{\textcrh}{ħ}
\renewcommand{\textesh}{ʃ}

% label for submitted and published chapters
\newcommand{\submitted}{{\color{red}Final version submitted to Language Science Press.}}
\newcommand{\published}{{\color{red}Final version published by
    Language Science Press, available at \url{https://langsci-press.org/catalog/book/312}.}}

% Treebank definitions
\definecolor{tomato}{rgb}{0.9,0,0}
\definecolor{kelly}{rgb}{0,0.65,0}

% Minimalism chapter
\newcommand\tr[1]{$<$\textcolor{gray}{#1}$>$}
\newcommand\gapline{\lower.1ex\hbox to 1.2em{\bf \ \hrulefill\ }}
\newcommand\cnom{{\llap{[}}Case:Nom{\rlap{]}}}
\newcommand\cacc{{\llap{[}}Case:Acc{\rlap{]}}}
\newcommand\tpres{{\llap{[}}Tns:Pres{\rlap{]}}}
\newcommand\fstackwe{{\llap{[}}Tns:Pres{\rlap{]}}\\{\llap{[}}Pers:1{\rlap{]}}\\{\llap{[}}Num:Pl{\rlap{]}}}
\newcommand\fstackone{{\llap{[}}Tns:Past{\rlap{]}}\\{\llap{[}}Pers:\ {\rlap{]}}\\{\llap{[}}Num:\ {\rlap{]}}}
\newcommand\fstacktwo{{\llap{[}}Pers:3{\rlap{]}}\\{\llap{[}}Num:Pl{\rlap{]}}\\{\llap{[}}Case:\ {\rlap{]}}}
\newcommand\fstackthr{{\llap{[}}Tns:Past{\rlap{]}}\\{\llap{[}}Pers:3{\rlap{]}}\\{\llap{[}}Num:Pl{\rlap{]}}} 
\newcommand\fstackfou{{\llap{[}}Pers:3{\rlap{]}}\\{\llap{[}}Num:Pl{\rlap{]}}\\{\llap{[}}Case:Nom{\rlap{]}}}
\newcommand\fstackonefill{{\llap{[}}Tns:Past{\rlap{]}}\\{\llap{[}}Pers:3{\rlap{]}}\\%
  {\llap{[}}Num:Pl{\rlap{]}}}
\newcommand\fstackoneint%
    {{\llap{[}}{\bf Tns:Past}{\rlap{]}}\\{\llap{[}}Pers:\ {\rlap{]}}\\{\llap{[}}Num:\ {\rlap{]}}}
\newcommand\fstacktwoint%
    {{\llap{[}}{\bf Pers:3}{\rlap{]}}\\{\llap{[}}{\bf Num:Pl}{\rlap{]}}\\{\llap{[}}Case:\ {\rlap{]}}}
\newcommand\fstackthrchk%
    {{\llap{[}}{\bf Tns:Past}{\rlap{]}}\\{\llap{[}}{Pers:3}{\rlap{]}}\\%
      {\llap{[}}Num:Pl{\rlap{]}}} 
\newcommand\fstackfouchk%
    {{\llap{[}}{\bf Pers:3}{\rlap{]}}\\{\llap{[}}{\bf Num:Pl}{\rlap{]}}\\%
      {\llap{[}}Case:Nom{\rlap{]}}}
\newcommand\uinfl{{\llap{[}}Infl:\ \ {\rlap{]}}}
\newcommand\inflpass{{\llap{[}}Infl:Pass{\rlap{]}}}
\newcommand\fepp{{\llap{[}}EPP{\rlap{]}}}
\newcommand\sepp{{\llap{[}}\st{EPP}{\rlap{]}}}
\newcommand\rdash{\rlap{\hbox to 24em{\hfill (dashed lines represent
      information flow)}}}


% Computational chapter
\usepackage{./packages/kaplan}
\renewcommand{\red}{\color{lsLightWine}}

% Sinitic
\newcommand{\FRAME}{\textsc{frame}\xspace}
\newcommand{\arglistit}[1]{{\textlangle}\textit{#1}{\textrangle}}

%WestGermanic
\newcommand{\streep}[1]{\mbox{\rule{1pt}{0pt}\rule[.5ex]{#1}{.5pt}\rule{-1pt}{0pt}\rule{-#1}{0pt}}}

\newcommand{\hspaceThis}[1]{\hphantom{#1}}


\newcommand{\FIG}{\textsc{figure}}
\newcommand{\GR}{\textsc{ground}}

%%%%% Morphology
% Single quote
\newcommand{\asquote}[1]{`{#1}'} % Single quotes
\newcommand{\atrns}[1]{\asquote{#1}} % Translation
\newcommand{\attrns}[1]{(\asquote{#1})} % Translation
\newcommand{\ascare}[1]{\asquote{#1}} % Scare quotes
\newcommand{\aqterm}[1]{\asquote{#1}} % Quoted terms
% Double quote
\newcommand{\adquote}[1]{``{#1}''} % Double quotes
\newcommand{\aquoot}[1]{\adquote{#1}} % Quotes
% Italics
\newcommand{\aword}[1]{\textit{#1}}  % mention of word
\newcommand{\aterm}[1]{\textit{#1}}
% Small caps
\newcommand{\amg}[1]{{\textsc{\MakeLowercase{#1}}}}
\newcommand{\ali}[1]{\MakeLowercase{\textsc{#1}}}
\newcommand{\feat}[1]{{\textsc{#1}}}
\newcommand{\val}[1]{\textsc{#1}}
\newcommand{\pred}[1]{\textsc{#1}}
\newcommand{\predvall}[1]{\textsc{#1}}
% Misc commands
\newcommand{\exrr}[2][]{(\ref{ex:#2}{#1})}
\newcommand{\csn}[3][t]{\begin{tabular}[#1]{@{\strut}c@{\strut}}#2\\#3\end{tabular}}
\newcommand{\sem}[2][]{\ensuremath{\left\llbracket \mbox{#2} \right\rrbracket^{#1}}}
\newcommand{\apf}[2][\ensuremath{\sigma}]{\ensuremath{\langle}#2,#1\ensuremath{\rangle}}
\newcommand{\formula}[2][t]{\ensuremath{\begin{array}[#1]{@{\strut}l@{\strut}}#2%
                                         \end{array}}}
\newcommand{\Down}{$\downarrow$}
\newcommand{\Up}{$\uparrow$}
\newcommand{\updown}{$\uparrow=\downarrow$}
\newcommand{\upsigb}{\mbox{\ensuremath{\uparrow\hspace{-0.35em}_\sigma}}}
\newcommand{\lrfg}{L\textsubscript{R}FG} 
\newcommand{\dmroot}{\ensuremath{\sqrt{\hspace{1em}}}}
\newcommand{\amother}{\mbox{\ensuremath{\hat{\raisebox{-.25ex}{\ensuremath{\ast}}}}}}
\newcommand{\expone}{\ensuremath{\xrightarrow{\nu}}}
\newcommand{\sig}{\mbox{$_\sigma\,$}}
\newcommand{\aset}[1]{\{#1\}}
\newcommand{\linimp}{\mbox{\ensuremath{\,\multimap\,}}}
\newcommand{\fsfunc}{\ensuremath{\Phi}\hspace*{-.15em}}
\newcommand{\cons}[1]{\ensuremath{\mbox{\textbf{\textup{#1}}}}}
\newcommand{\amic}[1][]{\cons{MostInformative$_c$}{#1}}
\newcommand{\amif}[1][]{\cons{MostInformative$_f$}{#1}}
\newcommand{\amis}[1][]{\cons{MostInformative$_s$}{#1}}
\newcommand{\amsp}[1][]{\cons{MostSpecific}{#1}}

%Glue
\newcommand{\glues}{Glue Semantics} % macro for consistency
\newcommand{\glue}{Glue} % macro for consistency
\newcommand{\lfgglue}{LFG$+$Glue} 
\newcommand{\scare}[1]{`{#1}'} % Scare quotes
\newcommand{\word}[1]{\textit{#1}}  % mention of word
\newcommand{\dquote}[1]{``{#1}''} % Double quotes
\newcommand{\high}[1]{\textit{#1}} % highlight (italicize)
\newcommand{\laml}{{L}} 
% Left interpretation double bracket
\newcommand{\Lsem}{\ensuremath{\left\llbracket}} 
% Right interpretation double bracket
\newcommand{\Rsem}{\ensuremath{\right\rrbracket}} 
\newcommand{\nohigh}[1]{{#1}} % nohighlight (regular font)
% Linear implication elimination
\newcommand{\linimpE}{\mbox{\small\ensuremath{\multimap_{\mathcal{E}}}}}
% Linear implication introduction, plain
\newcommand{\linimpI}{\mbox{\small\ensuremath{\multimap_{\mathcal{I}}}}}
% Linear implication introduction, with flag
\newcommand{\linimpIi}[1]{\mbox{\small\ensuremath{\multimap_{{\mathcal{I}},#1}}}}
% Linear universal elimination
\newcommand{\forallE}{\mbox{\small\ensuremath{\forall_{{\mathcal{E}}}}}}
% Tensor elimination
\newcommand{\tensorEij}[2]{\mbox{\small\ensuremath{\otimes_{{\mathcal{E}},#1,#2}}}}
% CG forward slash
\newcommand{\fs}{\ensuremath{/}} 
% s-structure mapping, no space after                                     
\newcommand{\sigb}{\mbox{$_\sigma$}}
% uparrow with s-structure mapping, with small space after  
\newcommand{\upsig}{\mbox{\ensuremath{\uparrow\hspace{-0.35em}_\sigma\,}}}
\newcommand{\fsa}[1]{\textit{#1}}
\newcommand{\sqz}[1]{#1}
% Angled brackets (types, etc.)
\newcommand{\bracket}[1]{\ensuremath{\left\langle\mbox{\textit{#1}}\right\rangle}}
% glue logic string term
\newcommand{\gterm}[1]{\ensuremath{\mbox{\textup{\textit{#1}}}}}
% abstract grammatical formative
\newcommand{\gform}[1]{\ensuremath{\mbox{\textsc{\textup{#1}}}}}
% let
\newcommand{\llet}[3]{\ensuremath{\mbox{\textsf{let}}~{#1}~\mbox{\textsf{be}}~{#2}~\mbox{\textsf{in}}~{#3}}}
% Word-adorned proof steps
\providecommand{\vformula}[2]{%
  \begin{array}[b]{l}
    \mbox{\textbf{\textit{#1}}}\\%[-0.5ex]
    \formula{#2}
  \end{array}
}

%TAG
\newcommand{\fm}[1]{\textsc{#1}}
\newcommand{\struc}[1]{{#1-struc\-ture}}
\newcommand{\func}[1]{\mbox{#1-function}}
\newcommand{\fstruc}{\struc{f}}
\newcommand{\cstruc}{\struc{c}}
\newcommand{\sstruc}{\struc{s}}
\newcommand{\astruc}{\struc{a}}
\newcommand{\nodelabels}[2]{\rlap{\ensuremath{^{#1}_{#2}}}}
\newcommand{\footnode}{\rlap{\ensuremath{^{*}}}}
\newcommand{\nafootnode}{\rlap{\ensuremath{^{*}_{\nalabel}}}}
\newcommand{\nanode}{\rlap{\ensuremath{_{\nalabel}}}}
\newcommand{\AdjConstrText}[1]{\textnormal{\small #1}}
\newcommand{\nalabel}{\AdjConstrText{NA}}

%Case
\newcommand{\MID}{\textsc{mid}{}\xspace}

%font commands added April 2023 for Control and Case chapters
\def\textthorn{þ}
\def\texteth{ð}
\def\textinvscr{ʁ}
\def\textcrh{ħ}
\def\textgamma{ɣ}

% Coordination
\newcommand{\CONJ}{\textsc{conj}{}\xspace}
\newcommand*{\phtm}[1]{\setbox0=\hbox{#1}\hspace{\wd0}}
\newcommand{\ggl}{\hfill(Google)}
\newcommand{\nkjp}{\hfill(NKJP)}

% LDDs
\newcommand{\ubd}{\attr{ubd}\xspace}
% \newcommand{\disattr}[1]{\blue \attr{#1}}  % on topic/focus path
% \newcommand{\proattr}[1]{\green\attr{#1}}  % On Q/Relpro path
\newcommand{\disattr}[1]{\color{lsMidBlue}\attr{#1}}  % on topic/focus path
\newcommand{\proattr}[1]{\color{lsMidGreen}\attr{#1}}  % On Q/Relpro path
\newcommand{\eestring}{\mbox{$e$}\xspace}
\providecommand{\disj}[1]{\{\attr{#1}\}}
\providecommand{\estring}{\mb{\epsilon}}
\providecommand{\termcomp}[1]{\attr{\backslash {#1}}}
\newcommand{\templatecall}[2]{{\small @}(\attr{#1}\ \attr{#2})}
\newcommand{\xlgf}[1]{(\leftarrow\ \attr{#1})} 
\newcommand{\xrgf}[1]{(\rightarrow\ \attr{#1})}
\newcommand{\rval}[2]{\annobox {\xrgf{#1}\teq\attr{#2}}}
\newcommand{\memb}[1]{\annobox {\downarrow\, \in \xugf{#1}}}
\newcommand{\lgf}[1]{\annobox {\xlgf{#1}}}
\newcommand{\rgf}[1]{\annobox {\xrgf{#1}}}
\newcommand{\rvalc}[2]{\annobox {\xrgf{#1}\teqc\attr{#2}}}
\newcommand{\xgfu}[1]{(\attr{#1}\uparrow)}
\newcommand{\gfu}[1]{\annobox {\xgfu{#1}}}
\newcommand{\nmemb}[3]{\annobox {{#1}\, \in \ngf{#2}{#3}}}
\newcommand{\dgf}[1]{\annobox {\xdgf{#1}}}
\newcommand{\predsfraise}[3]{\annobox {\xugf{pred}\teq\semformraise{#1}{#2}{#3}}}
\newcommand{\semformraise}[3]{\annobox {\textrm{`}\hspace{-.05em}\attr{#1}\langle\attr{#2}\rangle{\attr{#3}}\textrm{'}}}
\newcommand{\teqc}{\hspace{-.1667em}=_c\hspace{-.1667em}} 
\newcommand{\lval}[2]{\annobox {\xlgf{#1}\teq\attr{#2}}}
\newcommand{\xgfd}[1]{(\attr{#1}\downarrow)}
\newcommand{\gfd}[1]{\annobox {\xgfd{#1}}}
\newcommand{\gap}{\rule{.75em}{.5pt}\ }
\newcommand{\gapp}{\rule{.75em}{.5pt}$_p$\ }

% Mapping
% Avoid having to write 'argument structure' a million times
\newcommand{\argstruc}{argument structure}
\newcommand{\Argstruc}{Argument structure}
\newcommand{\emptybracks}{\ensuremath{[\;\;]}}
\newcommand{\emptycurlybracks}{\ensuremath{\{\;\;\}}}
% Drawing lines in structures
\newcommand{\strucconnect}[6]{%
\draw[-stealth] (#1) to[out=#5, in=#6] node[pos=#3, above]{#4} (#2);%
}
\newcommand{\strucconnectdashed}[6]{%
\draw[-stealth, dashed] (#1) to[out=#5, in=#6] node[pos=#3, above]{#4} (#2);%
}
% Attributes for s-structures in the style of lfg-abbrevs.sty
\newcommand{\ARGnum}[1]{\textsc{arg}\textsubscript{#1}}
% Drawing mapping lines
\newcommand{\maplink}[2]{%
\begin{tikzpicture}[baseline=(A.base)]
\node(A){#1\strut};
\node[below = 3ex of A](B){\pbox{\textwidth}{#2}};
\draw ([yshift=-1ex]A.base)--(B);
% \draw (A)--(B);
\end{tikzpicture}}
% long line for extra features
\newcommand{\longmaplink}[2]{%
\begin{tikzpicture}[baseline=(A.base)]
\node(A){#1\strut};
\node[below = 3ex of A](B){\pbox{\textwidth}{#2}};
\draw ([yshift=2.5ex]A.base)--(B);
% \draw (A)--(B);
\end{tikzpicture}%
}
% For drawing upward
\newcommand{\maplinkup}[2]{%
\begin{tikzpicture}[baseline=(A.base)]
\node(A){#1};
\node[above = 3ex of A, anchor=base](B){#2};
\draw (A)--(B);
\end{tikzpicture}}
% Above with arrow going down (for argument adding processes)
\newcommand{\argumentadd}[2]{%
\begin{tikzpicture}[baseline=(A.base)]
\node(A){#1};
\node[above = 3ex of A, anchor=base](B){#2};
\draw[latex-] ([yshift=2ex]A.base)--([yshift=-1ex]B.center);
\end{tikzpicture}}
% Going up to the left
\newcommand{\maplinkupleft}[2]{%
\begin{tikzpicture}[baseline=(A.base)]
\node(A){#1};
\node[above left = 3ex of A, anchor=base](B){#2};
\draw (A)--(B);
\end{tikzpicture}}
% Going up to the right
\newcommand{\maplinkupright}[2]{%
\begin{tikzpicture}[baseline=(A.base)]
\node(A){#1};
\node[above right = 3ex of A, anchor=base](B){#2};
\draw (A)--(B);
\end{tikzpicture}}
% Argument fusion
\newenvironment{tikzsentence}{\begin{tikzpicture}[baseline=0pt, 
  anchor=base, outer sep=0pt, ampersand replacement=\&
   ]}{\end{tikzpicture}}
\newcommand{\Subnode}[2]{\subnode[inner sep=1pt]{#1}{#2\strut}}
\newcommand{\connectbelow}[3]{\draw[inner sep=0pt] ([yshift=0.5ex]#1.south) -- ++ (south:#3ex)
  -| ([yshift=0.5ex]#2.south);}
\newcommand{\connectabove}[3]{\draw[inner sep=0pt] ([yshift=0ex]#1.north) -- ++ (north:#3ex)
  -| ([yshift=0ex]#2.north);}
  
\newcommand{\ASNode}[2]{\tikz[remember picture,baseline=(#1.base)] \node [anchor=base] (#1) {#2};}

% Austronesian
\newcommand{\LV}{\textsc{lv}\xspace}
\newcommand{\IV}{\textsc{iv}\xspace}
\newcommand{\DV}{\textsc{dv}\xspace}
\newcommand{\PV}{\textsc{pv}\xspace}
\newcommand{\AV}{\textsc{av}\xspace}
\newcommand{\UV}{\textsc{uv}\xspace}

\apptocmd{\appendix}
         {\bookmarksetup{startatroot}}
         {}
         {%
           \AtEndDocument{\typeout{langscibook Warning:}
                          \typeout{It was not possible to set option 'staratroot'}
                          \typeout{for appendix in the backmatter.}}
         }

   %% hyphenation points for line breaks
%% Normally, automatic hyphenation in LaTeX is very good
%% If a word is mis-hyphenated, add it to this file
%%
%% add information to TeX file before \begin{document} with:
%% %% hyphenation points for line breaks
%% Normally, automatic hyphenation in LaTeX is very good
%% If a word is mis-hyphenated, add it to this file
%%
%% add information to TeX file before \begin{document} with:
%% %% hyphenation points for line breaks
%% Normally, automatic hyphenation in LaTeX is very good
%% If a word is mis-hyphenated, add it to this file
%%
%% add information to TeX file before \begin{document} with:
%% \include{localhyphenation}
\hyphenation{
Aus-tin
Bel-ya-ev
Bres-nan
Chom-sky
Eng-lish
Geo-Gram
INESS
Inkelas
Kaplan
Kok-ko-ni-dis
Lacz-kó
Lam-ping
Lu-ra-ghi
Lund-quist
Mcho-mbo
Meu-rer
Nord-lin-ger
PASSIVE
Pa-no-va
Pol-lard
Pro-sod-ic
Prze-piór-kow-ski
Ram-chand
Sa-mo-ye-dic
Tsu-no-da
WCCFL
Wam-ba-ya
Warl-pi-ri
Wes-coat
Wo-lof
Zae-nen
accord-ing
an-a-phor-ic
ana-phor
christ-church
co-description
co-present
con-figur-ation-al
in-effa-bil-ity
mor-phe-mic
mor-pheme
non-com-po-si-tion-al
pros-o-dy
referanse-grammatikk
rep-re-sent
Schätz-le
term-hood
Kip-ar-sky
Kok-ko-ni
Chi-che-\^wa
au-ton-o-mous
Al-si-na
Ma-tsu-mo-to
}

\hyphenation{
Aus-tin
Bel-ya-ev
Bres-nan
Chom-sky
Eng-lish
Geo-Gram
INESS
Inkelas
Kaplan
Kok-ko-ni-dis
Lacz-kó
Lam-ping
Lu-ra-ghi
Lund-quist
Mcho-mbo
Meu-rer
Nord-lin-ger
PASSIVE
Pa-no-va
Pol-lard
Pro-sod-ic
Prze-piór-kow-ski
Ram-chand
Sa-mo-ye-dic
Tsu-no-da
WCCFL
Wam-ba-ya
Warl-pi-ri
Wes-coat
Wo-lof
Zae-nen
accord-ing
an-a-phor-ic
ana-phor
christ-church
co-description
co-present
con-figur-ation-al
in-effa-bil-ity
mor-phe-mic
mor-pheme
non-com-po-si-tion-al
pros-o-dy
referanse-grammatikk
rep-re-sent
Schätz-le
term-hood
Kip-ar-sky
Kok-ko-ni
Chi-che-\^wa
au-ton-o-mous
Al-si-na
Ma-tsu-mo-to
}

\hyphenation{
Aus-tin
Bel-ya-ev
Bres-nan
Chom-sky
Eng-lish
Geo-Gram
INESS
Inkelas
Kaplan
Kok-ko-ni-dis
Lacz-kó
Lam-ping
Lu-ra-ghi
Lund-quist
Mcho-mbo
Meu-rer
Nord-lin-ger
PASSIVE
Pa-no-va
Pol-lard
Pro-sod-ic
Prze-piór-kow-ski
Ram-chand
Sa-mo-ye-dic
Tsu-no-da
WCCFL
Wam-ba-ya
Warl-pi-ri
Wes-coat
Wo-lof
Zae-nen
accord-ing
an-a-phor-ic
ana-phor
christ-church
co-description
co-present
con-figur-ation-al
in-effa-bil-ity
mor-phe-mic
mor-pheme
non-com-po-si-tion-al
pros-o-dy
referanse-grammatikk
rep-re-sent
Schätz-le
term-hood
Kip-ar-sky
Kok-ko-ni
Chi-che-\^wa
au-ton-o-mous
Al-si-na
Ma-tsu-mo-to
}

   \togglepaper[9]%%chapternumber
}{}

\begin{document}
\maketitle
\label{chap:Cstr}

\section{Introduction}
Because LFG is based on using phrase-structure rules (PS rules, with a substantial
involvement of universal principles) to provide a direct
description of overt structure, with more abstract levels such as f-structure
determined by annotations on these rules, they carry a major burden in
describing the organization of clause-structure.  In particular, it is not possible
to invoke `movement' mechanisms to get things into their surface positions.  Rather,
with one plausible minor exception,\footnote
 {``Second position'' items, as discussed below.}
the PS rules have to put everything
in the exact positions where they are found overtly, albeit
with the possible help of filtering by other components of the grammar.

Partly for this reason, in LFG. the treatment of clause structure has been from the
beginning closely involved with the concept of `non-configurationality', a term coined
by \citet{Hale81} to refer to situations where linear order does not determine
grammatical relations in any clear way, and where, in addition, referring expressions
and other clausal constituents are sometimes discontinuous.  Transformational Grammar
and its more direct descendants, sometimes called `Mainstream Generative Grammar',
avoided having a problem with this by proposing that underlying clause structures were
transformed into overt ones by the application of `scrambling' rules (in later work,
sometimes relegated to the `phonology').  But when phrase structure rules are to be used
for providing a direct characterization of overt structures (with filtering by other
components), this is not possible.

Another relevant issue is the position of subjects. If a language
appears to have verb phrases that exclude an apparent `subject' NP argument, then
the PS rules have to provide a position for NP external to VP, while if a
putative subject is freely intermixed with other arguments, then we probably
do not want to have a full-sized VP containing the verb and other arguments,
but rather have the verb and the arguments appear
directly under S.\footnote
 {As we shall see, languages sometimes have a smaller verbal phrase containing
 the verb and certain other material, but not, normally, the object; this is
 sometimes treated as a VP, and sometimes as a different kind of verbal phrase,
 often symbolized as $\overline{\hbox{V}}$.}
The theory should then plausibly provide two
possibilities along the lines of (a) and (b) below, where (a) puts an
NP in front of a VP, while (b) has no VP:\footnote
 {\label{IDLPfn}We don't use the  `ID/LP' notation of \citet{Gazdar1982}, first applied to LFG by
   \citet{Falk84}, to allow the daughters of S to appear an any order in (\ref{ex:Cstr:1}b) (\textphraserule{S}{NP$^*$, V}) because of the plausibility of interpreting this as a possibly null string of NPs, either before or after one V.  There are issues worth looking into further here, but not in this chapter.}

\ea
\begin{xlist}\label{ex:Cstr:1}
\item \phraserule{S}{NP ~ VP}
\item \phraserule{S}{NP$^*$ ~ V ~ NP$^*$}
\end{xlist}
\z
A comma could also be included in the first rule to allow the NP to precede or follow the VP, as in Makua \citep{Stucky1983}.

A further general consequence of the archicture is that because many phenomena
including agreement, case-marking and anaphora can be largely or entirely
described in terms of the more abstract level of f-structure, the sources
of evidence for phrase structure are more limited than they are in Mainstream
Generative Grammar.  We cannot, for example, easily use coreference phenomena to motivate
phrase structures in which one object in a double object construction c-commands
another, but would need a very extensive (and therefore fragile) argument
to show that other levels such as f-structure are not sufficient.

In this chapter, I take a predominantly historical approach to clause structure
in LFG, on the basis that a reader might want to engage in literature from any time
from the early 1980s to the present, and therefore find useful some discussion of what
kinds of proposals were being made at different times.
I will divide the history of clause structure in LFG into three periods so far,
with the possibility of a new one starting now.  In the first, from the
beginnings of LFG in the early 1980s to the early 1990s, some version of the
X-bar theory was assumed, but there was little explicit discussion about
exactly what that version was.  The 90s constitute a transitional period, in which both
the `extended projections' from Minimalism and ideas from Optimality Theory are
taken on.  The third period plausibly begins with Bresnan's
(\citeyear{bresnan2001lexical}) theory of structure-function mappings, which can be seen
as a consolidation of the work of the transitional period, based on a division
between `endocentric' and `exocentric' structure, the former obeying the X-bar
theory with functional structures, the latter not, along with some principles
derived from Optimality Theory, such as Economy of Expression.  This approach
has persisted with little alteration through \citet{BresnanEtAl2016} to the
present.
Since it is the result of multiple analyses of different languages by a number
of people, I will call it the `2001 Synthesis'.  More recently, a fourth
period may have begun with \citet{low:lov:20} and \citet{Lovestrand2022},
a thorough revision of
the underlying phrase-structure theory making greater use of architectural ideas
of LFG rather than simply applying some version of mainstream X-bar theory.  However
there hasn't yet been substantial work on a variety of
clause structures in this new approach.

In this chapter, I consider early LFG in the first section, the transitional
period in the second, and the 2001 Synthesis in the third.  Then, in the fourth
section, I review some of the earlier and transitional systems in light
of the 2001 Synthesis, and discuss the revisions that are thereby motivated,
and conclude with a speculation about S derived from a modification
in the new X-bar framework made in \citet{Lovestrand2022}.

\section{Early LFG}\label{earlyLFG}
In early LFG, it was assumed that some version of the X-bar theory was correct,
but no attempt was made to seriously formalize or revise the proposals that were standard
at the time. \citet[354-356]{bresnan1982control-complementation}, which also appeared in the foundational LFG collection
 \citet{bresnan82}, developed a fairly permissive theory of ‘structure-function mappings’,
many provisions of which have persisted until now.  This theory constrains how c-structure
nodes can be annotated to produce f-structures, and shows the influence of Ken Hale’s ideas
about `configurational’ vs.\ `non-configurational' languages. The other papers in
the \citeyear{bresnan82} collection tended to conform to these ideas without discussing them explicitly.

In these papers, the languages treated as configurational
(where grammatical relations are largely coded by phrase-structure position)
were English, French, Russian and Icelandic, all of them analysed as SVO, while
the one analysed as non-configurational was Malayalam, predominantly verb-final.
Most of the SVO languages were analysed with distinct S and VP rules like these:
\ea\label{vsorule}
\begin{xlist}
\item \phraserule{S}{\rulenode{NP \\ {(\UP \SUBJ)=\DOWN}} \rulenode{VP \\ \UP=\DOWN}}
\item \phraserule{VP}{\rulenode{V \\ \UP=\DOWN} \rulenode{NP \\ {(\UP \OBJ)=\DOWN}} $\ldots$}
\end{xlist}
\z
But for Icelandic, Andrews used a flat S rule with subject first (consistent with
Bresnan's \citeyear[354/296]{bresnan1982control-complementation}\footnote{The page before the `/' is for the journal version, the one after for the
 page in the 1982 volume.}
schema, which he does not however cite or discuss).

From that time to the present, a major concern of LFG authors has been to marshal arguments that c-structure
relationships were neither a necessary nor sufficient basis for assigning grammatical
relations in the general manner suggested by
\citet{chomsky1965aspects} and persisting through to \citet{chomsky1981lectures}, which appeared in the early days of LFG, and beyond.
So considerable attention was paid to arguing for the nonexistence of a VP
node in languages that appeared not to have a VP constituent, and a contrastive sketch
of configurational English versus non-configurational Warlpiri constitutes the
first chapter of the high-level LFG textbook \citep{BresnanEtAl2016}.

In the following two subsections, we consider first Malayalam, and then Warlpiri,
the other non-configurational language to which considerable attention was paid
in this period; discussion of Warlpiri was included in \citet{bresnan1982control-complementation} and several
other chapters in the \citet{bresnan82} collection.\footnote
  {Warlpiri is discussed in \citet{SimpBres1983}, but the focus of that article
    is grammatical relations rather than clause structure as such.}

\subsection{Malayalam}
Non-configurational Malayalam, an essentially verb-final language, was analysed by
\citet{mohanan1982} using the clausal category S,
introducing its arguments as an unstructured, possibly empty, string of NPs.  It
was proposed to have a verb-final S rule, with
the NPs introduced by a rule whose first version was
\citep[507]{mohanan1982}:

\ea\label{noncfgpsr}
\phraserule{S}{\rulenode{NP$^*$} \rulenode{V}}
\z
Annotations for this version of the rule were not specified.
Later \citep[542--643]{mohanan1982}, a series of annotations are proposed associating specific
grammatical functions with cases and other properties such as animacy, for example:
\ea
\begin{tabular}[t]{l}
{(\UP \OBJ)=\DOWN}\\
{(\DOWN \CASE)=\ACC}~\\
{(\DOWN \ANIM)=+}
\end{tabular}
\z
This is one of the alternative annotations to the NP in rule
(\ref{noncfgpsr}), allowing any NP in the series to have any grammatical
function, subject to filtering by other constraints of LFG.

The flatness of the structure given by (\ref{noncfgpsr}), with its absence of
a VP dominating objects, is motivated by several arguments, one of which is
the fact that the bearers of the grammatical relations can appear in any order.
But the most important one, according to \citet[526-533]{mohanan1982}, is the workings of a clefting
phenomenon which allows all and only direct daughters of the S to be clefted,
but not subconstituents of anything, such as possessors or objects of prepositions.
The construction is effected by suffixing {\it -aanə} `is' to the last word
of the clefted constituent, and {\it -ṯə} `is' to the verb.  Some relevant
examples are:
\ea Malayalam \citep[528-529]{mohanan1982}
\begin{xlist}
\item
\gll Kuṭṭi iṉṉale ammakkə anaayey-aaṇə koṭuṯṯa-ṯə.\\
child  yesterday mother.{\DAT} elephant-is gave-it\\
\glt{`It was an elephant that the child gave to the mother yesterday.'}
\item
\gll kuḷaṯṯil wecc-aaṇə jooṇinte kuṭṭi aanaye ṇuḷḷia-ṯə.\\
pond at-is John.{\GEN} child elephant pinched-it\\
\glt{`It was at the pond that John's child pinched the elephant.'}
\item
\gll *kuḷaṯṯil aaṇə weccə jooṇinte kuṭṭi aanaye ṇuḷḷia-ṯə\\
pond is at John.{\GEN} child elephant pinched-it\\
\end{xlist}
\z
In (a), the object is clefted, in (b) the PP, but when we try to cleft the
object of the PP (or, not shown, the possessor of the object), the result is
bad.  The proposed generalization is that you can cleft a direct constituent of
S, but not a subconstituent of S, which precludes the existence of a VP sitting on
top of the verb and its object.

There are two further significant elaborations in Mohanan's analysis.  First, it turns out
that Malayalam is not actually strictly verb-final: in main clauses, the verb
can be followed by additional NPs, but this normally requires putting a heavy
nuclear pitch on the verb, wiping out the word melodies on the following NPs,
and lengthening the vowel of the verb, evidently with some kind of contrastive
meaning \citep[511]{mohanan1982}.  This is furthermore not possible with certain
kinds of subordinate clauses.  Mohanan suggests an analysis involving a `scrambling
rule', which applies to the S rule, but this is not an option that is available
in the LFG formalism, and these NPs need to be introduced in their surface positions,
presumably with annotations connecting them to discourse functions (\citetv{chapters/InformationStructure}).

The second elaboration
is that there is a kind of verb phrase, but it contains only the
verb and certain additional elements, such as NPs and PPs used to form Copula 
Constructions and Complex Predicates \citep[513-534]{mohanan1982}.
An example is:
\eabox{\begin{forest}
[S
    [$\overline{\hbox{N}}$
      [N
        [kaʐuṯa \\ donkey
        ]
      ]
    ]
    [$\overline{\hbox{V}}$
       [$\overline{\hbox{N}}$
           [Det [or̄ə \\ one] ]
           [N   [mrəgan \\ animal] ]
       ]
       [V   [aaṇə \\ is]]
    ]
]
\end{forest}

`The donkey is an animal.' \citep[513--534]{mohanan1982}
}
These are plausibly VPs (maximal projections of V, Malayalam having only one
phrasal projection layer), which are however very restricted in what kinds of constituents
and bearers of grammatical relations they can introduce. On the other hand,
some superficially similar complex verbal constituents in other languages
do not appear to contain any maximal projections, and so can be analysed as
V$^0$ nodes with adjoined `non-projecting' lexical nodes \citep{Toivonen2001},
also discussed in \citetv{chapters/CoreConcepts}. One example  is Complex
Predicate constructions in Japanese (\citealt{Ishikawa1985}, \citealt{Matsumoto1996}),
discussed in \citetv{chapters/ComplexPreds}, and another is Warlpiri preverbs,
considered shortly below.

Another important characteristic of Malayalam and other non-configurational languages
is that all or most arguments of verbs can be freely omitted, and understood as
if they were represented by pronouns. \citet[544]{mohanan1982} discusses this briefly,
and provides a few examples including:
\newpage
\ea Malayalam \citep[544]{mohanan1982}\\
\gll Roṭṭi ewiṭe? Kuṭṭi ṯiṉṉu.\\
bread where? Child ate.\\
\glt{`Where is the bread?  The child ate it.'}
\z
The LFG treatment of this kind of phenomenon uses lexical rules of `anaphoric
control', developed in \citet{bresnan1982control-complementation}, which optionally
add a
pronominal f-structure specification to the lexical entries of verbs.  In the
above case, this would be {(\UP\OBJ\PRED)=`\textsc{pro}'}.  In Malayalam, anaphoric control
applies to the grammatical relations \SUBJ, \OBJ, and \OBJTHETA (`indirect object';
\OBJ2 in the original). Anaphoric control is not restricted to non-configurational
languages, and is subject to numerous variations in different languages.  In English,
anaphoric control is predominantly used with subjects of nonfinite verbs, but also applies in some other, more limited, circumstances.
For example, an inanimate subject can be omitted but understood as if it refers
to something that the group of people being addressed are looking at:
\ea
Looks bad! [said by one of a group of people staring at an engine with smoke
coming out of it]
\z
The typological range of such constructions and their semantics deserves further
investigation in LFG.

\subsection{Warlpiri}
A more extreme form of non-configurationality was addressed in the comprehensive
analysis of Warlpiri provided by \citet{Simpson1983}, later published with substantial
revisions as \citet{Simpson1991}, and also discussed by \citetv{chapters/Australian}.  Warlpiri differs from Malayalam (and Japanese)
in a number of ways:
\ea\label{warldiff}
\begin{xlist}
\item In finite clauses, there is no constraint on NPs coming after the verb.
\item NPs can be discontinuous, with different components appearing separated by other
constituents of the clause.
\item There is an `auxiliary', obeying a complex `second position' constraint.
\end{xlist}
\z
All of these phenomena are illustrated in this example:
\newpage
\ea\label{warlp}
Warlpiri\\
\gll Kurdu-jarra-rlu ka-pala maliki wajilipi-nyi wita-jarra-rlu.\\
child-\DU-{\ERG} \PRS-{\DU} dog(\ABS) chase-\textsc{npast} small-\DU-\ERG\\
\glt{`(The) two small children are chasing the dog.'\\
`(The) two children are chasing the dog and they are small.'}
\z
The theoretically most interesting point is (\ref{warldiff}b).

\citet{Hale81} described the two nominals `child' and `small' in
(\ref{warlp}) as being interpretable
in two ways, one being the `merged' interpretation, shown in the upper gloss,
where the two components are interpreted in the same way as a normal NP in
English, and the other being the `unmerged' interpretation, shown in the lower gloss, in which
the second nominal is interpreted as a secondary predicate giving additional
information.

To capture the merged interpretation, Simpson proposed that an NP (in her analysis,
for Warlpiri, an $\overline{\hbox{N}}$, the language having no evidence for either a specifier
level for lexical projections, or any kind of DP) could expand
to an adjunct alone with no head (as well, as of course, to a single head), so
that two independently introduced components of an f-structural
NP-correspondent could merge, as reflected in the annotations on the tree for
example (\ref{warlp}):\footnote
 {Note that Simpson assumed a rule or convention that would copy agreement features from
 the f-structure correspondent of an NP down to those of its {\sc adj}-members. There
 is furthermore a problem with the positioning of AUX that we will consider shortly.}
\eabox{\label{warlptree}
\hspace*{-1cm}\resizebox{\textwidth}{!}{\begin{forest}
[S
    [{(\UP \SUBJ)=\DOWN}\\{$\overline{\hbox{N}}$}
      [{\UP=\DOWN\\N}
        [{(\UP \PRED)=\textsc{`child'}} \\ {(\UP \CASE)=\ERG} \\ {(\UP \NUM)=\DU}\\ kurdu-jarra-rlu
        ]
      ]
    ]
    [{\UP=\DOWN\\AUX}
        [{{(\UP \ASP)=\PRS}\\ {(\UP \SUBJ \PERS)=1}\\ {(\UP \SUBJ \NUM)=\DU}\\ka=pala}
        ]
    ]
    [{(\UP \OBJ)=\DOWN}\\{$\overline{\hbox{N}}$}
      [{\UP=\DOWN\\N}
        [{(\UP \PRED)=\textsc{`dog'}}\\ maliki
        ]
      ]
    ]
    [{\UP=\DOWN \\ $\overline{\hbox{V}}$}
      [{\UP=\DOWN\\V}
        [{(\UP \PRED)=\textsc{`chase\arglist{{(\UP \SUBJ)},{(\UP \OBJ)}}'}} \\ {(\UP \TENSE)=\NPST} \\ {(\UP \PRED)=\textsc{`pro'}} \\ wajili-pi-nyi
        ]
      ]
    ]
    [{{(\UP \SUBJ)=\DOWN}\\$\overline{\hbox{N}}$}
      [{\DOWN $\in$ (\UP \ADJ)}\\N
        [{{(\UP \PRED)=\textsc{`small'}} \\ {(\UP \CASE)=\ERG} \\ {(\UP \NUM)=\DU}\\wita-jarra-lu}
        ]
      ]
    ]
]
\end{forest}
}

~\hfill \citep[283]{Simpson1991}
}
  
The annotations on the first and last $\overline{\hbox{N}}$s allow them to unify into a single
f-substructure, the value of {\SUBJ}: 
\eabox{
\avm[style=fstr]{
[ \SUBJ & [ \PRED & \textsc{`child'}\\
           \ADJ & \{[\PRED & \textsc{`small'}\\
                     \CASE & \textsc{erg}]\}\\
           \CASE & \ERG]\\
  \PRED & `chase\arglist{{(\UP \SUBJ)},{(\UP \OBJ)}}'\\
  \ASP  & \PRS\\
  \TENSE & \NPST\\
  \OBJ & [\PRED & \textsc{`dog'}]]
  }
}
Similarly to Malayalam, Warlpiri also makes extensive use of anaphoric control, although
many arguments not expressed as NPs will receive morphological registration in the AUX
constituent, which we will discuss shortly.

There are two further characteristics of Simpson's analysis that interact with each
other, and have been important in later developments.
The first is that similarly to Malayalam, Simpson analyses Warlpiri
as having an inner VP, symbolized as $\overline{\hbox{V}}$, containing the verb and certain
other elements, especially `preverbs', as discussed in \citet[111]{Simpson1991}.
However none of these items can contain complex phrasal constituents, and it is
therefore probably better to treat them as non-projecting words adjoined to
V.

The second additional feature of Warlpiri, already seen in the Warlpiri tree structure
(\ref{warlptree}), is the `AUX' constituent.  This was postulated for Warlpiri\footnote
 {AUX as a node-type was widely proposed at that time for the analysis of
 many other languages, including English, as discussed extensively for
 example by \citet{AkmajianSteeleWasow1979}.}
in the classic article of
\citet[310]{Hale1973}, as a constituent containing three kinds of constituents,
all optional.  First comes a `complementizer', which has a variety of functions,
later called the `augment' by \citet{Laughren2002}.  We will follow this usage.
Next comes the `base', which is one of the tense-aspect markers {\it -ka}
`present imperfective' or {\it -lpa} `past imperfective'.\footnote
 {Hale treated the future marker {\it kapi} as a base, but \citet{Legate2008warl2p}
shows that it is actually an augment.}
Finally come agreement markers, for subject
and object.  \citet[312]{Hale1973} proposed  that if the augment+base sequence was less
than two syllables in length, then the auxiliary could
not appear in initial position, but only after some other, evidently first, element
of the clause.\footnote
 {With the exception that certain items, such as topics set off with a pause,
  were seen as appearing outside the basic clause structure, allowing the AUX
  to appear in apparent third position if these items were included.}

\citet[83]{Simpson1991} proposes that the underlying position of the AUX is initial,
as specified by this rule:
\newpage
\ea
\phraserule{S}{(AUX) \makebox[1em]{} $\alpha$ \makebox[1em]{} $(\alpha)^*$}\\
$\alpha=\overline{\hbox{V}}, \overline{\hbox{N}}$, Particle\\
(\UP\textsc{aspect}) $\leftrightarrow ($\UP\TENSE)~\\
Assign (\UP \textsc{g})=\DOWN freely~\\
(where \textsc{g} stands for any grammatical function)
\z
The second line tells us that $\alpha$ can be any of three kinds of constituents,
while the third adds the information that \textsc{aspect} is specified if and only
if \TENSE\ is, a move that has the effect of requiring a verb to be present
if an AUX is, by mechanisms we will not consider here.  Finally, the last two
lines allow constituents to be annotated
freely as either heads (\UP=\DOWN) or arguments bearing any grammatical
function, providing a high degree of non-configurationality, including generating
multiple $\overline{\hbox{N}}$ nodes with the {(\UP \SUBJ)=\DOWN} annotation, allowing
NP-splitting.

AUX is then put into second position in most examples by first allowing all
AUXs to be classified as enclitics, but obligatorily so for the ones with monosyllabic
bases \citep[69]{Simpson1991}.  Then the clitics are postposed to a position after
the first phonological unit by a rule of sentence-phonology:
\ea
Encliticization Rule:\\
]AUX $\;$ $[\alpha]$ $\;$ $[\alpha]^*$ $\;\;\rightarrow\;\; \alpha+$AUX $\;[\alpha]^*$\\
(the `]' in front of AUX represents that the AUX has enclitic status)
\z
It is perhaps worth noting that for all examples where AUX appears in second
position, the trees are also written with AUX in second position, including the
effects of the Encliticization Rule in the diagram.
In later work, various aspects of this proposal are questioned and revisions proposed,
as we will see later, in \sectref{ausnonconfig} of this chapter.

A final observation I will make about Warlpiri concerns the treatment of discontinuous
NPs.  The LFG analysis permits an NP to be split into any number of separated
components, all of which can contribute to a single f-structure with a nominal
\PRED-feature, subject to no constraints of any kind.  There is considerable work
showing that this appears to be false, including \citet{SchuSima12}, \citet{Schultze-Berndt2022}
and \citet{LouaVers16}. Rather, discontinuous NPs appear to be associated
with a range of specific discourse functions (and to furthermore be rather rare, probably
not more than 1\% of NPs), and examples where an NP is split into three or more
components at the same level in f-structure (e.g. demonstrative, modifier(s) and head noun)
do not appear to be attested in the literature so far.  
Unfortunately, none of this recent literature discusses Warlpiri, but I am aware of no
triply split NPs in \citet{Simpson1983, Simpson1991}, \citet{Nash1986} or
\citet{Laughren1989},\footnote
 {Who also provides an example of discontinuous participial VPs, which \citet{Simpson1991}
 argues are nominalized.}
nor in the discussion of discontinuous NPs in Latin and Classical Greek provided by
\citet{DevineStephens2000}. Therefore, the proposition that discontinuous constituents are
limited to two components is a proposition worth further investigation.

Similarly to Mohanan, Simpson provides some arguments for a flat structure and no VP,
but they are more complex than Mohanan's cleft argument, one involving coreference,
another involving nonfinite constructions.  These will not be discussed here.

Concluding our discussion of the first period, we find a basic distinction between configurational
and non-configurational encoding, the former associated with SVO languages,
usually associated with a VP, the latter with verb-final or verb-anywhere languages,
often with flat structure.  It was usual to assume some kind of X-bar
theory, without being very specific about the details.  There were however some
intimations of later developments, such as Falk's (\citeyear{Falk84}) analysis
of the English Auxiliary system, in which, influenced by \citet{jackendoff1977}, he treats
auxiliaries as a lexical category M, taking VP as a complement and the subject
as a specifier.  This can be seen as an early version of the idea of the functional
projection IP, with its binary  branching auxiliary structure as opposed to the
flat ternary structure NP AUX VP proposed for S by \citet{AkmajianSteeleWasow1979},
which is taken up in the third period, the 2001 Synthesis (\sectref{synthsec}).

\section{The transitional period}
The characteristics of the transitional period are (a) the introduction of the
concept of `functional categories' from the GB and Minimalist frameworks, a feature which
has remained; (b) considerable experimentation with ideas from Optimality Theory,
which appears to have fallen off to some degree, although it is still being explored
(\citetv{chapters/OT}).
The dating of the period is difficult,  since the use of functional categories could be
said to have been anticipated by \citet{Falk84}, while drafts of what I take
to be the initiation of the third period, \citet{bresnan2001lexical}, were
available to some workers as early as 1996 \citep[15]{nordlinger1998constructive}.
But I will here take it to begin with \citet{Kroeger93},\footnote
 {The Stanford PhD thesis upon which the book is based is from 1991.}
where the functional
categories I and C are adopted from Government-Binding theory,
and continue until the publication of \citet{bresnan2001lexical}.  
Many of the features of what I will in the next section call the `2001 Synthesis' are
present in the analyses of the transitional period, to the point that some discussions
could be put in either section.  But here we take an essentially chronological view
surveying phenomena that lead to the 2001 Synthesis, presenting the resulting system
in \sectref{synthsec}, together with some new analyses as well as possible updates
to older ones.

\subsection{\citet{Kroeger93} on Tagalog}
Tagalog is a verb-initial language with preverbal discourse positions, and, according
to Kroeger, no evidence for a VP, and some evidence against (we will consider a different view below),
but evidence for some other predicate phrases, namely,
PP, AP and NP.  Kroeger analyses these patterns by taking from \citet{ChungMcCloskey1987}
the idea of a special category `S' that can constitute a predication (`small clause')
without providing TAM information, and combining this with the notion of
a `functional projection' IP, where the S appears as the complement of I.  He
also departs from \citeauthor{ChungMcCloskey1987} to allow S to expand to a lexical predicate
and multiple arguments, rather than only to a subject NP and a predicate
phrase.

For clauses with an aspect-marked verb, this verb appears in the I
head of the functional projection
IP (INFL for Kroeger), 
while the arguments and adjuncts appear in free order under S, although there
are some tendencies (p. 111):
\ea
\begin{xlist}
\item The `Actor' (non-nominative Agent marked with {\it ng}, or {\it ni} if
a proper name) tends to come first.
\item The `Nominative' (marked with {\it ang}, or {\it si} if a proper name) tends
  to come after the other arguments.
\item ``Heavier'' NP's tend to follow ``lighter'' NPs.
\end{xlist}
\z
Kroeger does not actually give a structure for a sentence with multiple NP
arguments: the closest is one with a clitic Actor and a focussed adjunct in the specifier of IP. Clitics are however subject to very interesting positional restrictions which
in this case put the clitic {\it ko} after the SPEC of IP, as indicated in this example
\citep[129]{Kroeger93}, where the original tree has INFL rather than just I:\footnote
 {The clitic rule, discussed in \citet[119-123]{Kroeger93}, is: ``Clitics
 appear immediately after the first daughter of the smallest maximal projection that
 contains them'' (but there are some apparent exceptions).  The `object-focus' suffix
 glossed \textsc{ov} indicates that the Patient of the verb is the `grammatical
 subject', traditionally called the `focus' in Philippine linguistics, analysed
 by Kroeger as the {\SUBJ} grammatical function in his LFG analysis.}

\newpage
\ea Tagalog
\begin{xlist}
\item\label{tagfin}
\gll [Para kay=Pedro] ko binili ang=laruan.\\
for \DAT=Pedro {1\SG.\GEN} \PRF.buy.\textsc{ov} \NOM=toy\\
\glt{`For Pedro I bought the toy.'}
\item\label{tagfintree}
%
% didn't want to take the time to figure out how to do this in forest right now
%
%\input{trees/tagfintree.tex}
\begin{forest}
[IP
    [PP
        [{para kay=Pedro\\for \DAT=Pedro}, roof]
    ]
    [NP, phantom [,name=target]]
    [I$'$
        [I
            [binili\\\PRF.buy.\textsc{ov}]
        ]
        [S
            [NP
                [ko, circle, draw]
                {\draw[->] () to[out=south west,in=south west] (target);}
            ]
            [NP
                 [{ang=laruan\\\NOM=toy},roof]
            ]
        ]
     ]
]
\end{forest}
\end{xlist}
\z
In this tree, the {\it ko} is ascribed to a position under S where a full
NP argument could appear (initial in accord with the
ordering tendencies noted above), with the arrow indicating
some kind of clitic displacement to after the first constituent of IP.
Another thing to note is the use of prime notation rather than bars, so
I$'$ instead of $\overline{\hbox{I}}$.  In this chapter, I will use whichever notation
is employed by the original author. 

Since functional projections have both the head and the complement annotated with
{\UP=\DOWN}, there is no problem with assembling the f-structure for a c-struc\-ture
such as (\ref{tagfintree}).  The initial PP is an instance of what Kroeger and much
subsequent work has called `Adjunct Fronting', which applies to the bearers of
non-core GFs, that is, adjuncts, oblique arguments and adverbials
\citep[43]{Kroeger93}.  We can analyse this by allowing SPEC of IP to receive
one of the non-term grammatical functions, together with some sort of focus-like discourse
function.  I can't find an explicit statement of this in the literature, but it
appears to be an implication of the discussion in Kroeger and other sources such as
\citet{GerassimovaSells2008} that the construction is clause-bounded,
since only subjects are said to be extractable from subordinate clauses, and
only from ones that are themselves subjects \citep[210,~215-221]{Kroeger93}.

So we can propose an annotated structure like (\ref{philanntree}) below for the example,
with the clitic {\it ko} placed overtly in its second position, without concern here for what
constraints put it there, an issue discussed extensively by \citet{Kaufman2010},
but too complex 
to attempt to provide an updated account of here.   We will notate it
as `\textsc{a-sb}', for the non-subject
Agent in Philippine languages, following the choice of \citet{Manning1996} for the
Agent in syntactically ergative languages such as Inuit:
\ea\label{philanntree}
\begin{forest}
[IP
    [ {\DOWN $\in$ (\UP \ADJ)}\\ {(\UP \FOC)=\DOWN}\\ PP
        [{para kay=Pedro}\\for \DAT=Pedro, roof]
    ]
    [{(\UP \textsc{a-sb)=\DOWN}}\\Cl 
        [I\\ko]
    ]
    [{\UP=\DOWN}\\ I$'$ 
        [{\UP=\DOWN}\\I
            [binili\\\PRF.buy.\textsc{ov}]
        ]
        [{\UP=\DOWN}\\ S
            [{(\UP \SUBJ)=\DOWN}\\ NP
                 [{ang=laruan}\\{\NOM=toy},roof]
            ]
        ]
     ]
]
\end{forest}
\z

Given appropriate lexical entries, this will produce the following f-structure:\footnote
  {This structure uses the older treatment of discourse functions such as {\FOC}us as
  grammatical functions in f-structure, along the lines of \citet[97-98]{bresnan2001lexical}
  or \citet[97]{BresnanEtAl2016}.   \citet{Kroeger93} does not provide any specific f-structures.
  For contemporary views, see \citetv{chapters/InformationStructure}.}
%%%%%%% 
\ea
\avm[style=fstr]{
  [ \FOC  & \rnode{a}{[\PRED & `for'\\
                      \OBJ & [\PRED & `Pedro']]}\\\smallskip
    \SUBJ & [ \PRED & `toy'\\
              \CASE & \NOM\\
              \textsc{spec} & \DEF
             ]\\
    \PRED & `buy<{{(\UP \sc a-sb)}} {(\UP\SUBJ)}>'\\
    \ASP  &  \PRF\\
    \textsc{a-sb} & [ \PRED  & `pro'\\
                  \PERS & 1\\
                  \NUM  & \SG\\
                  \CASE & \GEN
                ]\\
     {\sc adjuncts} & \{\rnode{b}{[$\;$]}$\;$\}
  ]
}
\CURVE[1]{-2pt}{0}{a}{-2pt}{0}{b}
\z

\noindent
In addition to verbal clauses, there are clauses with adjectival, nominal and prepositional
predicates.  Kroeger argues that these show a different pattern, where some phenomena of
clitic placement are said to show that the main predicate can either appear on its own as first daughter
of S, or as head of a phrase that contains its complements, with the subject final under
S in the former case, as indicated by these (somewhat abbreviated) structures \citep[133]{Kroeger93}:
\ea\label{tagcstrucs}
\begin{xlist}
\item
\begin{forest}
[S
    [XP\\(\PRED)]
    [NP\\(\SUBJ)]
]
\end{forest}
\item
\begin{forest}
[S
      [X$^0$]
      [YP]
      [YP]
]
\end{forest}
\end{xlist}
\z
Unfortunately, \citet[259-260]{Kaufman2010}, working within the Minimalist framework,
finds that the clitic facts cited by Kroeger\footnote
 {Originally from \citet{SchachterOtanes1972} and \citet{Sityar1989}.}
do not appear to be representative, in ways that undermine Kroeger's analysis.  Since
this is of some interest for the history of the subject, I think it is worth
considering the examples, in the hope that it will be further investigated in LFG.

What Kroeger says is that with nominal, adjectival
and propositional phrasal predicates, a personal pronoun clitic can appear either
at the end, or after the predicate word, illustrated here for PP:
\ea\label{galing} Tagalog
%
%  the \DAT etc commands are introducing an unwanted space.
%
\begin{xlist}
\item
\gll Galing sa=Maynila siya.\\
from {\sc dat}=Manila 3\SG.\NOM\\
\glt {`He is from Manila.'}
\item
\gll Galing siya sa=Maynila.\\
from {3\SG.\NOM} {\sc dat}=Manila\\
\glt {`He is from Manila.'}
\end{xlist}
\z
But with a verbal main predicate, the sentence-final position is impossible:

\newpage
\ea Tagalog
\begin{xlist}
\item
??\gll Hinangkan ng=nanay ako.\\
\PRF.kiss.\textsc{dv} {\sc gen}=mother 1\SG.\NOM\\
\glt{`I was kissed by mother.'}
\item
\gll Hinangkan ako ng=nanay.\\
\PRF.kiss.\textsc{dv} {1\SG.\NOM} {\sc gen}=mother\\
\glt{`I was kissed by mother.'}
\end{xlist}
\z
This is to be explained by:
\ea\label{tagclitprinc}
\begin{xlist}
\item A principle to the effect that the clitics are placed after the first
constituent in the domain they apply to, which is the IP.
\item The two constituent structures in (\ref{tagcstrucs}) are available for nonverbal
predicates, but only the flat one of (\ref{tagcstrucs}b) for verbal predicates.
\end{xlist}
\z
However, Kaufman finds that there is no significant difference between the clitic final
position for verbal and nonverbal predicates: both are pretty bad.  He also argues
that Kroeger's generalization about where the clitic goes is insufficient, and proposes
something different, well beyond the scope of this chapter.  This leaves the
flat rule (\ref{tagcstrucs}b) motivated by the evidence, but not (\ref{tagcstrucs}a), for Tagalog.

Nevertheless, there is motivation for structures of the general form of (\ref{tagclitprinc}a) elsewhere
in the Austronesian language family: Dalrymple and Randriamasimanana use it in
their XLE grammar of Malagasy,\footnote
  {\url{http://users.ox.ac.uk/~cpgl0015/pargram/}; argumentation is however not
  provided.}
and \citet{LiuKL2017} presents an LFG analysis of Squliq Atayal arguing on various
grounds for this structure.  Finally, \citet{KaufmanChen2017}
review a rather long tradition of argumentation in Austronesian historical syntax
for the position that structures with a clause-initial predicate phrase and a following
subject are the original form of the `Philippine type' of which Tagalog is the most
often discussed exemplar.

It is perhaps worth emphasizing that the `subject' in Philippine languages
is not the classic subject of western European languages with its strong association
with semantic Agent properties, but rather the `Pragmatic Peak' of
\citet{FoleyVanValin1984}, drawing heavily on earlier work by Paul Schachter and Edward
Keenan \citep{keenan76,Schachter1977}, or the `g-subject' of \citet{Manning1996}.
These have an association with topic-like pragmatic functions, but not with agentivity.
Indeed, the constructions with patient as subject tend to be more common than those
with agent, and are closer in form to the proposed diachronic original, as discussed
by \citet{KaufmanChen2017}.

A final point is that above IP, arguably the domain of clitic positioning, there
are projections treated by Kroeger as CP, and irrelevant to clitic positioning
(resulting in `third position' phenomena), and this general approach is also adopted
by \citet{Kaufman2010}.


\subsection{Warlpiri: non-configurationality in Australian Languages}\label{ausnonconfig}
\citet{AustBres96} update Simpson's (\citeyear{Simpson1983}, \citeyear{Simpson1991}) analysis
of Warlpiri to use the functional projection IP to house the material constituting the auxiliary,
and also give an extended treatment
of Jiwarli.  Two important differences between Warlpiri and Tagalog with
respect to I are:
\ea
\begin{xlist}
\item In Warlpiri, the verb does not appear in I.\footnote
 {\citet{Legate2008warl2p} proposes in her Mimimalist analysis that the verb can be
 attracted to I, but in LFG, there is no advantage to be obtained by allowing it
 to appear there, as we will see below.}
\item I nodes in Warlpiri that meet a certain condition, to be discussed immediately below, cannot
appear initially, at least in a phonologically independent clause.
\item Most items, including the verb, can appear in front of the auxiliary
material (contents of I).  This analysis proposes two mechanisms for how this happens:
NPs appearing in SPEC of IP, and a prosodic inversion operation for verbs and preverbs.
\end{xlist}
\z
The nature of condition (b) calls for some discussion.

As mentioned above, earlier work from \citet{Hale81} to \citet{Simpson1991}  proposed that the AUX had to appear in second position if the augment+base was monosyllabic, but \citet{Laughren2002} shows that the bisyllabicity
condition is not correct, on the basis that the complementizer {\it yi-}
`for, since' followed by a null base can appear initially, as long
as the entire auxiliary, including agreement markers, is bisyllabic (all the other
augments are bisyllabic).   This is also the case for the present imperfective
base {\it -ka}, but apparently not for the past imperfective base {\it -lpa}.\footnote
 {But the existence of possible exceptions is discussed in \citet[125,
     footnote 19]{Laughren2002}.}
So I suggest that the actual condition is a combination of phonology and morphology:
\ea
In order for I to be overtly initial in Warlpiri, its contents must:
\begin{xlist}
\item be at least bisyllabic
\item be phonologically well-formed as a word (initial {\it -lp} clusters are
 not allowed)
\item have an overt augment+base (either augment or base is sufficient, as long as something appears).
\end{xlist}
\z
I will call an auxiliary that meets these conditions `heavy', and one that doesn't, `light'.
So our basic generalization is that only heavy auxiliaries can be initial in the
sentence.

With this issue considered, we examine the basic sentence structure that \citet{AustBres96} propose for Warlpiri, with an initial NP in the SPEC of IP position,
and auxiliary material in the following I:
\ea
%\input{trees/warltree.tex}
\hspace*{-5mm}\begin{forest}
[IP, s sep=0em
    [NP
        [kurdu-jarra-rlu\\child-\DU-\ERG,roof]
    ]
    [I$'$
        [I
               [ka{=}pala\\ \PRS{=}3\DU.\SUBJ]
        ]
        [S
             [NP
                 [maliki\\dog.\ABS,roof]
             ]
             [V
                 [wajilipi-nyi\\chase-\NPST]
             ]
             [NP
                 [wita-jarra-rlu\\small-\DU-\ERG,roof]
             ]
        ]
    ]
]
\end{forest}
\z
This illustrates the first mechanism whereby the auxiliary material can appear in
second position, but is not plausible for cases when the verb is in first position, 
because a lexical category should not be able to occur in SPEC position in Warlpiri \citep[226]{AustBres96}.

To deal with verb-initial sentences,  \citeauthor{AustBres96} propose the rule of prosodic inversion from \citet{Halpern95},
which moves the
contents of I to a position after the verb, or, sometimes, after the initial part
of a complex verb.  This division of labor permits the inversion rule to apply to
a considerably more restricted range of cases than Simpson's Encliticization, removing
the need for it to swap the auxiliary around multiword phrases.

Examples of multiword phrases are modifier+modified nominal constructions with case marking,
on either only the last or both elements, %, object+verbal noun constructions,
% where the bloody hell is it
and also coordinate NPs (auxiliaries in boldface to make the examples easier to follow):
\newpage
\ea Warlpiri
\begin{xlist}
\item
\gll Kurdu(-ngku) wita-ngku={\bf ka} maliki wajilipi-nyi\\
child(-\ERG) small-{\ERG}={\PRS} dog(\ABS) chase-{\NPST}\\
\glt{`The small child is chasing the dog.'}\\
(\citealt[159-160]{Nash1986}, citing \citet{Hale81})
\item
\gll Karnta-ngku manu ngarka-ngku={\bf pala} kurdu nya-ngu\\
woman-{\ERG} and man-{\ERG}=3.\DU.{\SUBJ} child.(\ABS) see-\PST\\
\glt{`The man and the woman saw the child.'}
\citep[177]{Nash1986}
\end{xlist}
\z
Since these multi-word NPs can be generated in a position before I, we do not need to have
any rule putting the auxiliary after them.

However, sentences with something other than an NP appearing before the auxiliary 
pose some tricky problems.  If all and only the things found in this position were
verbs, we could suggest that a V could optionally be adjoined to I,
appearing in front of the auxiliary material.  But this proposal faces two problems.
One is that if the preverb appears in its normal position before the verb, then
the auxiliary can appear between them, as long as it doesn't contain
an augment (but polysyllabic auxiliaries with base {\it ka} are fine.):\footnote
 {The suffix {\it -jinta} glossed {\sc dd} indicates that this is a `failed effect'
  construction discussed in \citet[336]{Hale1973}, in which the object is marked dative, indicating that the action
  indicated by the verb did not succeed.}
\ea\label{warlphrase} Warlpiri  \citep[227]{AustBres96}\\
\begin{xlist}
\item
\gll Rambal-luwa-rnu={\bf rna=rla=jinta} marlu-ku\\
mistake-shoot-\PST=1\SG.\SUBJ-1\SG.\DAT-{\sc dd}  kangaroo-\DAT\\
\item
\gll Rambalpa={\bf rna=rla=jinta} luwa-rnu marlu-ku\\
mistake=1\SG.\SUBJ-3\SG.\DAT-{\sc dd} shoot-{\PST}  kangaroo-\DAT\\
\end{xlist}
`I shot at a kangaroo and failed.'
\z
If the auxiliary intervenes, a preverb whose stem ends in a consonant must end in the stem-extender
{\it -pa}, which it can do anyway.  This indicates that one requirement for
intervention is that the preverb must be construed as in some sense being an independent
word.

Another is that for those `productive' preverbs that can appear after or before the verb, the auxiliary material seems to almost obligatorily
appear after just the verb rather than after the whole verb+preverb combination
when the preverb comes second \citep[117]{Simpson1991}:
\ea Warlpiri \citep[52]{Nash1986}\\
\gll Yani={\bf rli} wurulypa\\
go-\textsc{npst}=12 seclusion\\
\glt{`Let's go and hide.'}
\z
Nash observes that the ordering {\it yani-wurulypa=rli} occurs once in a text,
but seems much less common than the other possibilities, while Simpson characterizes
it as ``hardly ever found, and it is usually rejected by speakers''.  This is a problem
for any analysis which puts V in front of I in c-structure, unless we assume
a category other than V to dominate the V+Preverb order.\footnote
  {On the other hand, \citet[100]{Laughren2002} provides such an example without
  comment, but with a heavy auxiliary which could not be inserted into the verb.}

So the conclusion is that an inversion rule along the lines of Simpson's
encliticization is needed, but applying in a narrower range of cases, more consistent
with being a morphological or phonological operation.  \citet{AustBres96} assume
that it is the `prosodic inversion' of \citet{Halpern95}, which only applies as
a last resort.  This is supposed to explain why the clitics can't be inserted into
phrasal units as in (\ref{warlphrase}), but there is a problem here in that
the structure in which the phrasal NP is sitting in SPEC of IP is different from one
where it is initial in S right after I, so it is not clear that a `last resort'
restriction can apply in a well-defined manner.\footnote
 {And it is furthermore clear that it is in general not impossible for
 second-position clitics to be inserted into otherwise intact NPs; this
 is for example rather common with {\it -que} `and' in Latin.}
A further problem is that it appears to be fine for a heavy auxiliary to appear
after the verb \citep[97]{Laughren2002}, so a last resort restriction won't work.

What I suggest is an inversion rule which can be formulated like this:
\ea
\begin{tabular}[t]{cccc}
I & V/Pvb\\
1 & 2 & $\Rightarrow$ 
  & 2+1
\end{tabular}\\
Subject to restrictions that need further investigation.
\z
The category restriction is sufficient to prevent I from being inserted
into an NP (a restriction documented at considerable length by Laughren),
and another restriction, not formalized here, states that an auxiliary with an augment cannot be inserted
into a verb.  

A final problem discussed by \citet{Laughren2002} and \citet{Legate2008warl2p}
concerns evidence that there is more than one functional projection dominating S, in
spite of no evidence of two distinct head positions being occupied in the same clause.
This is the interaction of topicalization and focus in questions. In
(\ref{warltopint}) below, we see the auxiliary between the topic and a question word:
\ea\label{warltopint}
Warlpiri  \citep[34]{Legate2008warl2p}\\
\gll Kuturu-ju ka=npa=nyanu nyarrpara-wiyi marda-rni?\\
nullanulla-{\TOP} \PRS.\textsc{ipfv}=2\SG.\textsc{nom}=\textsc{anaphobj} where-first have-\NPST\\
\glt{`Where do you have this nullanulla of yours?'}
\z
And in (\ref{legexmpls}a-c)
below, we see that a potentially interrogative word can be interpreted as either 
interrogative or indefinite if it appears right after an auxiliary,
but only interrogative if before, while (\ref{legexmpls}d) shows that if a potential question word
appears further into the clause, after the auxiliary and the verb, it can only be interpreted as
indefinite:
\ea\label{legexmpls} Warlpiri
\begin{xlist}
\item
\gll Kaji=ka=rna nyarrpara-kurra ya-ni.\\
\textsc{nfactc}=\PRS.\IPFV=1{\SG} where-{\ALL} go-\textsc{npst}\\
\glt{`I might go somewhere.'/`Where might I go?' \citep[17]{Legate2008warl2p}}
\item
\gll Nyarrpara-kurra kaji=ka=rna ya-ni.\\
where-{\ALL} \textsc{nfactc}=\PRS.\IPFV=1{\SG}  go-\textsc{npst}\\
\glt{*`I might go somewhere.'/`Where might I go?' \citep[17]{Legate2008warl2p}}
\item
\gll Nyiya=rlangu kaji=ka=rlu nyina wampana-piya-ju.\\
what-for.example {\sc nfactc}=\PRS.\IPFV=3\PL.{\OBJ} be.\textsc{npst} spectacled.hare.wallaby-like-\TOP\\
\glt{`What ones for example might be like this spectacled hare wallaby?'\\
\citep[18]{Legate2008warl2p}}
\item
\gll Kaji=lpa=ngu wanti-yarla nyiya-rlangu milpa-kurra.\\
{\sc nfactc}=\PST.{\IPFV}=2\SG.{\OBJ} fall=\textsc{irr} what-for.example eye-\ALL\\
\glt{`If something fell into your eyes $\ldots$'\\
*`What might have fallen into your eyes?' \citep[18]{Legate2008warl2p}}
\end{xlist}
\z
I tentatively suggest the following analysis.  The auxiliary appears in a
fixed position, which \citeauthor{AustBres96} call I, although C would also work.
The interrogative/indefinite pronouns
are interpreted as interrogative if they appear `external to S', indefinite
if `internal'.  `Internal to S' means that they appear inside the lowest S
in a stack of S's to which things have been adjoined (and are therefore not
themselves adjoined), `external to S' outside of the lowest in such a stack, so
either adjoined to a S or in some higher projection.  In Warlpiri, there are two ways in which this
can happen: they can appear in Spec of CP, giving rise to (\ref{legexmpls}b-c)
above, or adjoined to S, giving rise to the interrogative interpretation of
(\ref{legexmpls}a), where an internal position is also possible, giving rise
to the indefinite interpretation.  But in the case of (\ref{legexmpls}d), the
pronoun can only be internal to S, so only an indefinite interpretation
is  possible. However, for this to be the case we need a bit more, namely a restriction on adjunction to S, that it can only add
a question-focus, which is easy to arrange with appropriate annotations.\footnote
  {I do not know what happens if there is more than interrogative word; typologically,
  there are various possibilities.}
Finally, in the case of
(\ref{warltopint}), the Spec of CP position is occupied by the topic, so adjunction
to S, and consequent position right after the auxiliary, is the only possibility
for an interrogative reading (the only one that makes sense in the context).

The essential difference between the present LFG analysis and Legate's is that in
the LFG analysis, the auxiliary appears in one position, and interrogatives in two, one
on on either side of the auxiliary, while in Legate's, interrogatives appear in one
position, while auxiliaries can appear overtly in two.
I am aware of no clear theory-independent empirical evidence distinguishing
between these possibilities; they are each motivated by what appears to work out
best given the resources of the theory.

\subsection{Russian}\label{russian}
Although Russian has sometimes been presented as non-configurational, \citet{King95}
argues that it is configurational, but with provisions that make the word order
considerably more flexible than in English. She provides first a Government and Binding (GB)
analysis, and then an LFG one, which leans heavily on the GB analysis for data
and associated discussion.

She analyses Russian clauses as having CP, IP and VP layers, with two bar levels in each. The outer level of the VP introduces subjects, and there is one further layer over CP, which is available only in main clauses. This is for ‘external topics’, which have an initial XP, set off by a pause, with possibly an anaphoric pronoun later in the clause  \citep[202]{King95}: 
\ea Russian
\begin{xlist}
\item
\gll Gleb, ja ego ne ljublju.\\
Gleb, I him not like\\
\glt{`Gleb, I don't like him.'}
\item
\gll Opera, net drugogo vida musykal'nogo iskusstva, kotoryj privlekal by k sebe takoe vnimanie.\\
opera not other type musical art which attract would to itself such attention.\\
\glt{`Opera, there is no other kind of musical art which would attract such attention to itself.'}
\end{xlist}
\z
She analyses this with an `expression phrase' rule \citep{Banfield:Unspeakable,Rudin1985}  as follows, outside the
X-bar system (similarly in her GB analysis):
\ea\label{ex:Cstr:33}
\phraserule{EP}{\rulenode{XP\\ {(\UP \textsc{e-top})=\DOWN}}  \rulenode{CP\\ \UP=\DOWN} }
\z
\citet[105]{King95} suggests that these external topics do  not fall under the X-bar
system in Government-Binding theory, and does not attempt to assimilate them
to X-bar theory in LFG either, but I suggest that perhaps the rule in (\ref{ex:Cstr:33}) could be replaced by
`Chomsky-adjunction' to CP,\footnote
 {Meaning that it has one CP node as sister, another as mother,  with identical
 feature-composition.}
with some kind of further restriction, perhaps essentially
semantic, preventing them from occurring in embedded positions \citep[106]{King95}:
\ea
\phraserule{CP}{\rulenode{XP\\ {(\UP \textsc{ e-top})=\DOWN}}  \rulenode{CP\\ \UP=\DOWN} }
\z
Russian is not a full pro-drop language, so that an NP coreferential with
an \textsc{e-top} is normally expressed.

In C we find the complementizer {\it\v sto} and the question-marker {\it li}, most
frequent in embedded questions, while in SPEC of CP we find question words: all question words
in multiple wh-questions \citep[215]{King95}, unlike in English where only one
appears. This is illustrated in the following example (\citealt[216]{King95}, {\UP=\DOWN} annotations
omitted):\footnote
  {In the tree, the annotation over the first constituent should be
  \DOWN $\in$ {(\UP\gloss{adjuncts})}, but in the structure I give the original.} 
\ea Russian
\begin{xlist}
\item
\gll kogda kto udaril Boris-a\\
when who(\NOM) hit Boris-\ACC\\
\glt{`Who hit Boris when?'}
\item
\begin{forest}
[CP
    [{(\UP \ADJ)=\DOWN}\\ \DOWN $\in$ (\UP {\sc qfoc})\\AP [kogda\\when]]
    [{(\UP \SUBJ)=\DOWN}\\ \DOWN $\in$ (\UP {\sc qfoc})\\NP [kto\\who]]
    [C$'$
      [IP
          [I$'$
              [I [udaril\\hit.\PST]]
              [VP  [V$'$   [{(\UP \OBJ)=\DOWN}\\NP   [Borisa\\Boris.\ACC]]]]
          ]
      ]
    ]
]
\end{forest}\\
\end{xlist}
\z
Yes-no questions can be marked either by intonation, or with the marker {\it li},
which appears in second position, after either an XP or the verb.  An XP in front of \textit{li}
is interpreted as a focus, with the body of the question presupposed (\citealt[236--237]{King95}, \citealt{King1994}),
and items in SPEC of IP are Topics, as indicated by the annotation:
\ea\label{rusmainverb}
Russian:
\begin{xlist}
\item
\gll Knigu li ona pro\v citala?\\
book.{\ACC} Q she.{\NOM} read.\PST\\
\glt{`Was it a book that she read?'}
\item
\begin{forest}
[CP
    [{(\UP \OBJ)=\DOWN}\\ \DOWN $\in$ (\UP {\sc qfoc})\\ NP  [N  [{knigu\\book.\ACC}]]]
    [C$'$
        [C  [{li\\Q}]]
        [IP
            [{(\UP \SUBJ)=\DOWN}\\ \DOWN $\in$ (\UP \TOPIC)\\ NP  [{ona\\she.\NOM}]]
            [I$'$  [I  [{pro\v citala\\read.\PST}]]]
        ]
     ]
]
\end{forest}
\end{xlist}
\z

If the yes-no question has no focus, then the verb appears before {\it li}, and
King proposes that the verb is adjoined to C: 
\ea
\begin{forest}
[CP
    [C$'$
        [C
            [I  [pro\v citala\\read.\PST]]
            [C  [li\\Q]]
        ]
        [IP
            [{(\UP \SUBJ)=\DOWN}\\ \DOWN $\in$ (\UP \TOPIC)\\ NP  [ona\\she.\NOM]]
            [I$'$   [{(\UP \OBJ)=\DOWN}\\ NP  [N  [knigu\\book.\ACC]]]]
        ]
    ]
]
\end{forest}
\z
This solution avoids the issue of putting a nonmaximal projection in specifier position,
and is independently motivated by the absence of the focus-presupposi\-tion
articulation that appears when there is an XP in the specifier position.  Formally,
it could also be applied to Warlpiri, but without the additional motivation,
insofar as is now known, and the problem of insertion of the auxiliary into
combinations of verb and preverb would remain unsolved. 

Moving down into IP, what we find here is `internal topics' normally followed by 
contrastive foci, but with the possibility of some topics, especially pronouns,
to come after the foci. Both contrastive foci and post-focus topics bear grammatical functions in their clause, and therefore
do not cooccur with resumptive pronouns.  The ordering phenomena are also connected
to the issue of `emotive' vs. `non-emotive' sentences, an important topic in Russian
syntax that does not get much discussion in King's LFG analysis (although more in
the GB one).  After considering various proposals, including recursive right-branching,
King's final proposal is to use the ID/LP rule format \citep[208]{King95}, previously mentioned in
footnote~\ref{IDLPfn}, where the `$<$' symbol in (b) means that the item in front
of $<$ must appear before the one after, if both occur in the structure: 
\ea
\begin{xlist}
\item
\phraserule{IP}{\rulenode{XP$^*$,\\ {(\UP \GF)=\DOWN} \\  \DOWN $\in$ (\UP\textsc{df})}
                \rulenode{I$'$\\ \UP=\DOWN}}
\item \TOP $<$ {\sc cfoc}, XP $<$ I$'$
\end{xlist}
\z
A constraint putting the I$'$ after the XPs  has been added,  and `{\sc df}' (discourse
function) is assumed to comprise ordinary topics ({\TOP} but not {\sc etop}), and
contrastive foci ({\sc cfoc}).  This formulation allows for considerable flexibility
in word order, even though the language is fundamentally configurational.
The big difference between King's IP rule for Russian and Bresnan's for
English is that in Russian, any \GF\ that is also topic or focus can appear in
Spec of IP, while in English, only subjects can.  \citet[133]{King95} notes that preverbal
subjects tend to appear less markedly topicalized than other preverbal items,
and discusses some possible reasons for this, including the tendency for arguments
to be ordered consistently with the thematic hierarchy, which would put Agentive
subjects, the most frequent kind, at the top, and therefore tending to be first.

In I itself appear finite verbs, either main verbs as in example (\ref{rusmainverb}),
or the future auxiliary {\it budet} `be', used to form imperfective futures, with
the  main verb appearing as head of VP:
\ea\label{rusfutverb}
\begin{forest}
[IP
  [{(\UP\SUBJ)=\DOWN}\\ {\DOWN$\in\,$(\UP\TOPIC)}\\ NP\\ Anna\\ Anna.\NOM]
  [I$'$
     [I\\ budet \\ will]
     [VP
        [V\\ \v citat'\\ read.\INF]
        [{(\UP\OBJ)=\DOWN} \\ NP \\ knigu \\book.\ACC]
     ]
   ]
]
\end{forest}
\z
This leads on to the structure of VP.

A somewhat unusual feature of King's analysis is that it introduces
subjects not only as SPEC of IP, but also as SPEC of VP, along with other governed
GFs that appear in V$'$ as usual, leading to this rule \citep[209]{King95}:\footnote
 {King omits the Kleene star on the complements in (b), presumably as a typographical
 error.}
\ea\label{russvp}
\begin{xlist}
\item
\phraserule{VP}{ \rulenode{(XP) \\ {(\UP\SUBJ)=\DOWN}}
                 \rulenode{V$'$ \\ \UP=\DOWN}
               }
 \item
 \phraserule{V$'$}{ \rulenode{V \\ \UP=\DOWN}
                  \rulenode{XP$^*$ \\ {(\UP\GF)=\DOWN} }
             }
\end{xlist}
\z
The evidence for this comes from various kinds of sentences where the subject
is not also a topic, discussed more in King's GB analysis than in the
LFG version.

One kind of example is `thetic sentences', which assert that something happened,
with no NP or other constituent singled out as the topic.  The order in such examples
is VSX$^*$, as illustrated in the example below, as answer to the question `what
happened yesterday' \citep[101]{King95}:
\ea
\begin{forest}
[IP
     [{(\UP \ADJ)=\DOWN}\\\DOWN $\in$ (\UP \TOPIC)\\AdvP  [v\v cera\\yesterday]]
     [I$'$
         [I   [prislal\\sent ]]
         [VP
             [{(\UP \SUBJ)=\DOWN}\\NP  [mu\v z\\husband]]
             [V$'$   [{(\UP \OBJ)=\DOWN}\\ NP  [den'gi\\money]]]
         ]
      ]
]
\end{forest}\\
`Yesterday (my) husband(\FOC) sent(\FOC) (me) money(\FOC).'
\z
The placement of the subject in SPEC of VP rather than SPEC of IP causes it
to be interpreted as Focus rather than Topic, leading to a thetic interpretation
of the clausal material excluding {\it v\v cera} `yesterday'.
We will see later in \sectref{ltwoconfrev} that this analysis
can be assimilated to that of other languages within the 2001 Synthesis
presented in that section,
by having the complement of I be (configurational) S rather than VP with
internal subject.

\subsection{German}\label{german}
%
% pg 33 discusses how her OT analysis does not yield I over 'auxiliaries' in final position
%
German as analysed by \citet{Choi1999}\footnote
 {A revised version of her 1996 Stanford dissertation of the same title.}
resembles Warlpiri in arguably having only one functional projection,\footnote
 {A possibly confusing factor is that the German problem in \citet[375-379]
  {BresnanEtAl2016} assumes that the auxiliary {\it haben} in final position is an
  occupant of I, but makes no argument for this analysis, which Choi explicitly
  rejects \citep[33]{Choi1999}.  \citet[31-32]{Berman2003}, citing Choi, also discusses the lack of
  evidence for I.}
but differs in a number of respects:
\ea
\begin{xlist}
\item Verbs can appear in the functional projection (and, often, must), but otherwise
appear finally in VP.
\item There is no NP-splitting.
\item There is some evidence for an S node, although this is challenged by the later work
of \citet{Berman2003}.
\end{xlist}
\z
The functional projection in these works is labelled ``C$'$'',\footnote
  {Presumably because it contains some traditional complementizers, such
  as {\it dass} `that', although this choice is essentially arbitrary.}
and houses complementizers in subordinate
clauses, and the topmost (main) verb in matrix clauses.  Otherwise, the verb appears
clause-finally, and the so-called ``auxiliaries'' are treated as main verbs taking
VP complements. Full clauses consist of an S with an NP~VP structure.  A sample
main clause with the auxiliary {\it sollte} is:
\ea\label{germanmain}
German
\begin{xlist}
\item
\gll Nachher sollte der Kurier dem Spion den Brief zustecken.\\
later was.supposed.to the courier.{\NOM} the spy.{\DAT} the note.{\ACC} slip\\
\glt{`Later, the courier was supposed to slip the letter to the spy.'}
\item
\hspace*{-1cm}\begin{forest}
[CP
    [AdvP
        [nachher\\later]
    ]   
    [C$'$
       [C
          [sollte\\was.supposed.to]
       ]
       [S
          [NP
             [der Kurier\\the courier.{\NOM}]
          ]
          [VP [VP?
             [$\overline{\hbox{V}}$
                  [NP
                     [dem Spion\\the spy.{\DAT}]
                  ]
                  [$\overline{\hbox{V}}$
                      [NP
                          [den Brief\\ the note.{\ACC}]
                      ]
                      [V
                          [zustecken\\slip]
                      ]
                  ]
             ]
          ]]
       ]
    ]
]
\end{forest}
\end{xlist}
\z
This structure is extrapolated from \citet[19, ex.~7a]{Choi1999} on the
basis of later
examples such as \citet[27, ex.~20]{Choi1999}.  The nested $\overline{\hbox{V}}$ nodes are postulated to
introduce the complements of V, a feature of the 2001 Synthesis which appears
to be arbitrary, as we will discuss in \sectref{synthsec}.

The question-marked VP solely dominated by another VP is motivated by the fact that
the verbs traditionally called `auxiliaries' in German ({\it sein} `be', {\it haben}
`have', {\it werden} `become', and the modals such as {\it sollen} `should', which also has other
meanings) appear syntactically
to be plausibly taken to be the complement of the auxiliary {\it sollen} in C, whose VP appears immediately
over the one with the question-marks.  On the other hand, LFG for some time has been
strongly oriented towards structure minimization principles, and the upper VP, which
would be annotated `\UP=\DOWN', is not doing anything, and is therefore highly likely to
be omitted, and indeed seems to be omitted by Choi in the somewhat later abbreviated
structure (17) on p.\ 26. 

A typical subordinate clause structure is:
\ea\label{germansub} German \citep[27, ex.~20]{Choi1999}
\begin{xlist}
\item
\gll dass der Kurier nachher dem Spion den Brief zustecken sollte\\
that the courier.{\NOM} later the spy.{\DAT} the note.{\ACC} slip was.supposed.to\\
\glt{`that the courier was supposed to slip the note to the spy later'}
\item
\resizebox{.45\textheight}{!}{
\begin{forest}
[CP
  [C
     [dass\\that]
  ]
  [S
      [NP
          [der Kurier\\the courier.{\NOM}]
      ]
      [VP
          [VP
              [AdvP
                  [nachher\\later]
              ]
              [VP
                [$\overline{\hbox{V}}$
                    [NP
                        [dem Spion\\the spy.{\DAT}]
                    ]
                      [$\overline{\hbox{V}}$
                        [NP
                             [den Brief\\the note.{\ACC}]
                        ]
                        [V
                            [zustecken\\slip]
                        ]
                    ]
                ]
              ]
          ]
          [V
              [sollte\\was.supposed.to]
          ]
      ]
  ]
]
\end{forest}
}
\end{xlist}
\z
Here we see the complementizer occupying C, and the VP complement to the auxiliary
{\it sollte}, at the end.

Although this analysis is in accord with the 2001 Synthesis, to be discussed in
the next section, a later analysis by \citet{Berman2003} rejects certain aspects
of it, especially the arguments for putting the subject under S, as we will consider
in \sectref{germanrev} below.


\subsection{Korean}
The languages we have seen so far have one or two functional projections over S
or equivalent, but \citet{Choi1999} presents Korean as having none: S expands
to NP followed by VP.  She finds no evidence for I or C, since the
functions of these projections
are expressed by formatives on the verb, removing the need for any phrase
structure positions to house them, and no other kinds of evidence for their
existence. 

She provides three arguments for VP \citep[43--47]{Choi1999}, of which I will give two. One
of them is that there is a
contrastive focus-marking particle {\it nun} which can be attached to either an
object NP or the verb to make either the attached-to constituent or the entire VP
the focus, but not the entire clause including the subject.  Illustrating the
two readings for marking on the object, we have:
\ea Korean\\
\gll Mary-ka chayk-un ilk-nun-ta\\
Mary-{\NOM} book-{\TOP} read-\PRS-\textsc{dcl}\\
\glt{`Mary reads nothing but books.'\\
(Mary does nothing but read books)}\\
     (\citealt[45, example 52]{Choi1999}; note that Choi glosses the marker as `\TOP' even though
its function here is described as contrastive focus.)
\z
Another argument is phonological: syllable-initial obstruents become voiced after a vowel
in a phonological phrase, and this happens between an object and the
following adverb {\it caypalli}, but not a subject; here, the segments
voiced for this reason are italicized:\footnote
  {Choi uses underlining to indicate the non-phonemic voicing, rather than
  different segmental phonetic symbols.}
\ea \label{korscram} Korean
\begin{xlist}
\item
\gll  Cwuni-{\it g}a kong-ul {\it j}aypalli {\it j}a{\it b}-a.\\
Cwuni-{\NOM} ball-{\ACC} quickly catch-\PRS\\
\glt{`Cwuni catches balls quickly.'}
\item
\gll Kong-ul Cwuni-{\it g}a caypalli {\it j}a{\it n}-a.\\
ball-{\ACC} Cwuni-{\NOM} quickly catch-\PRS\\
\glt{`Balls Cwuni catches quickly.'}
\end{xlist}
\z
This treatment contrasts with that of \citet{sells1994}, who proposes that Korean
has an `inner VP' (similar to the combinations of verb and preverb in Warlpiri)
which can be plausibly analysed as a V$^0$ with adjoined non-projecting words,
but no S vs.\ VP distinction.
Instead, following \citet{fukui1986}, all arguments are attached by phrase structure rules
expanding VP to XP and VP, recursively. Sells's argument for the VP seems convincing,
but not those for the binary branching structures for the arguments \citep[353, fn.\ 2]{sells1994}.
Later, in \citet{bresnan2001lexical}, a branching VP like that of Sells was
accepted, but with no serious attempt to show that it was superior to the
more traditional flat VP assumed by Choi.

Another important characteristic of Choi's phrase structures is the absence of
verbal functional projections.  She considers an analysis in which tense and
the declarative markers are treated as inhabitants of I and C, respectively, and
morphologically fused with the verb, but rejects such analyses on the
basis of violating the LFG principle that inflected words are inserted under
single terminal nodes in the c-structure.  One could propose that these
projections do exist, but are fused with the verb via lexical sharing, but
then there would be the problem of the nonexistence of any evidence for
the syntactic autonomy of the two components, of the kind that the mechanism
of lexical sharing was devised in order to explain.

As a consequence of the absence of C and I, we cannot use SPEC positions of these
projections to house preposed items to provide a treatment of scrambling
as found in (\ref{korscram}b).  Choi does not in fact present any c-structures
for scrambled sentences in Korean, but states \citep[9]{Choi1999} that this is to be
the structure for scrambling, and illustrates them for German \citep[127, ex.\ 17a]{Choi1999} with
left-adjunction to S.  This illustrates the principle that LFG does not propose
a functional projection if there is no material that can occupy the head of that
projection (in any structure; it is allowed for the head position to be unoccupied in
some structures).

Furthermore, since there is no IP, we can't use SPEC of IP to house
the subject, so Choi proposes S expanding to NP and VP.  Therefore, the structure
of (\ref{korscram}b) is:
\ea
\begin{forest}
[S
    [{(\UP\OBJ)=\DOWN}\\ NP
       [kong-ul \\ ball-{\ACC}]
    ]
    [{\UP=\DOWN} \\ S
       [{(\UP\SUBJ)=\DOWN}\\ NP
          [Cwuni-ka \\ Cwuni-{\NOM}]
       ]
       [{\UP=\DOWN} \\ VP
          [\DOWN$\in${(\UP\ADJ)} \\ AdvP
             [caypalli \\ quickly]
          ]
          [{\UP=\DOWN} \\ $\overline{\hbox{V}}$ [{\UP=\DOWN} \\ V
             [cap-a \\ catch.\PRS]
          ]]
        ]
     ]
]
\end{forest}
\z
{\it Caypalli} is not one of the adverbs listed by Sells as restricted to
immediately preverbal position, so we introduce it here as a daughter of VP
rather than adjoined to V.  The general question of VP versus
$\overline{\hbox{V}}$ is a difficult one, which the new phrase structure theory of
\citet{low:lov:20} might allow us to eliminate and thereby solve, but this is beyond
the scope of this chapter.


\section{The 2001 synthesis}\label{synthsec}
And now we turn to the system presented in \citet{bresnan2001lexical},
foreshadowed at various points in the preceding discussion,
and largely unchanged in its successor \citep{BresnanEtAl2016},
which will be the source of our page reference citations.
We have already introduced many elements of this proposal, so it is time to develop
it more systematically.  The basic ingredients, some of which are present in all
of the intermediate stage analyses, are:
\ea
\begin{xlist}
\item the 3-level X-bar theory of \citet{chomsky1970remarks}, with one lexical
and two phrasal levels, with the option
 for a language to have only two levels (one lexical and one phrasal), as in
 Warlpiri;
\item the modified system of category features (for nouns, verbs, etc.) from
\citet{jackendoff1977} and \citet{bresnan1982control-complementation};
\item functional (extended) projections in the version of
\citet{grimshaw00ep}, in which the extended projections share category features
  with their complements, which facilitates keeping the number of phrasal projections
  to 2. These are normally called I (as in Warlpiri), or C (as in German);
\item the existence of a category S, outside of the X-bar system, which
can either be non-configurational, as proposed for Tagalog
and Warlpiri, or configurational as will be proposed for Welsh and some minor
constructions in English;
\item principles of structure-function mapping that limit what kinds of grammatical
functions can be introduced in what positions;
\item the claim that phrases in the X-bar system are `endocentric', while S is `exocentric'.
\end{xlist}
\z

A new feature of the 2016 version relative to the 2001 version is the `non-projecting words'
of \citet{Toivonen2001}, discussed in \citetv{chapters/CoreConcepts}, which have
a category feature but are adjoined to another phrase without projecting
anything themselves.  Another feature of both versions that is not widespread
in X-bar theory is the treatment of multiple complements.  \citet[127]{BresnanEtAl2016}
observe that there are two possible treatments of complements, (\ref{Cstr:ex:49}a) below
with a nested structure, (\ref{Cstr:ex:49}b) with a flat one:
\ea\label{Cstr:ex:49}
\begin{xlist}
\item
\phraserule{X$'$}{X$'$, YP}
\item
\phraserule{X$'$}{X, YP$^*$}
\end{xlist}
\z
(\ref{Cstr:ex:49}b) is what the older LFG literature assumed, (\ref{Cstr:ex:49}a) what
\citet{bresnan2001lexical} and \citet{BresnanEtAl2016} propose, on the
basis of supporting a `flexible definition of an endocentric complement' (\citealt[118]{bresnan2001lexical}, \citealt[123]{BresnanEtAl2016}).\footnote
 {\citet{BresnanEtAl2016}, fn.~50 refers to footnote 41, but that appears to be
 be irrelevant; the relevant definition appears on the page following fn.~41.}
Option (a) constitutes the choice that Choi makes for German in
example (\ref{germanmain}), and works well when the complements can
be ordered freely, but it is not clear to me how it is
to account for the ordering restrictions in double object constructions,
where the {\OBJ} tends to precede an {\OBJTHETA}; the relevant restrictions
can be easily stated, either with conventional phrase structure rules as in most early LFG,
or with the ID/LP format (briefly mentioned in footnote \ref{IDLPfn}).
I suggest this is an issue best left for future investigation.

A further theme that interacts with the X-bar principles is a tendency to
reduce the complexity of c-structure to a minimum.  Two of the more important
conditions are:
\ea\label{ex:Cstr:50}
\begin{xlist}
\item
A c-structure position is not postulated unless there is a class of items that
can fill it.
\item
In any specific structure, all nodes are optional.
\end{xlist}
\z
(\ref{ex:Cstr:50}a) prevents us from postulating a functional projection such as `T' for topic,
or `E' for `evidential',
unless we can find a class of items that plausibly appear in their head positions,
while (\ref{ex:Cstr:50}b) allows us to leave out items in specific cases, as will
be discussed below.

The optionality of c-structure positions is an aspect of an important more
general principle, Economy of Expression \citep[90]{BresnanEtAl2016}:
\ea
{\bf Economy of Expression}:\\
All syntactic phrase structure nodes are optional and are not used unless
required by independent principles (completeness, coherence, semantic expressivity).
\z
A consequence of this principle is that the traditional principle
of `S as the ``initial symbol'' in a phrase-structure derivation of a sentence'
is abandoned: a `sentence' can phrase-structurally
be an S, an IP, or a CP, depending on the language and details of the particular
sentence.
I suggest that this involves a shift from a traditional `syntactic' notion
of sentence-hood to a more `semantic one', since glue semantics (\citetv{chapters/Glue})
can connect these multiple superficial syntactic structures to a single
semantic type.  Further reductions in the complexity of c-structure are
achieved by the reworking of X-bar theory developed in \citet{lovestrand-lowe2017},
but these have not yet been applied to a substantial typological variety of
clause structures so as to produce results with differently organized overall
structure, and therefore will not be discussed here.

The c-structure principles interact with a set of structure-function
mapping principles, which constrain the relationship between the
c-structures and the f-structures.  The principles
\citep[105-109]{BresnanEtAl2016} assert that:
\ea\label{structurefunction}
\begin{xlist}
\item C-structure heads are f-structure heads.
\item Complements of functional categories are f-structure co-heads.
\item Specifiers of functional categories are the grammaticalized discourse
functions, such as \SUBJ, \FOC, \TOP.
\item Complements of grammatical categories are the non-discourse argument
functions, such as \OBJ, \OBJTHETA (but not \SUBJ).
\item Constituents adjoined to phrasal constituents are optionally nonargument
functions, either adjuncts or nonargument discourse functions.
\end{xlist}
\z
This principle does not apply to S, whose daughters can bear any grammatical function.

Some simple effects of these principles apply in the structure for `Mary swims':\footnote
 {The \citet{bresnan2001lexical} version has S instead of IP, but this is rejected due to the lack of
 independent evidence for S in English main (indeed, finite and most nonfinite) clauses
 in English.}
\ea\label{simpIP}
\begin{forest}
[IP
    [NP [Mary] ]
    [VP
       [V$^0$ [swims]]
    ]
]
\end{forest}\\
\citep[120]{BresnanEtAl2016}
\z
The NP `Mary' can be a subject because it is external to the VP in SPEC of IP position,
while the other nodes will bear the $\UP=\DOWN$ equation and so will correspond to the
single f-structure that the NP's f-structure correspondent is {\SUBJ} of.  Turning
to Economy, unfilled heads and intermediate level nodes that dominate nothing but their head
are eliminated, and the IP, VP, and V nodes will all correspond to the same
f-structure.

The CP level appears in complement clauses, where the complementizer is head of C,
and in certain other structures such as questions, where we get a CP level with an
auxiliary verb filling C, a kind of analysis
that is strengthened by the fact that an auxiliary can replace the complementizer
{\it if} in a somewhat archaic/solemn variant of conditional clauses:
\ea\label{QinC}
Has he called?
\z
\ea\label{ArchCond}
\begin{xlist}
\item If he had called, I would have answered.
\item Had he called, I would have answered.
\end{xlist}
\z
For (\ref{QinC}), given the preceeding, the plausible structure is:
\ea
\begin{forest}
[CP
     [C$'$
          [C [has] ]
          [IP
              [NP [N [he]]]
              [VP [V [called]]]
           ]
      ]
]
\end{forest}
\z
The IP provides the location of the subject, which has no alternative
locations in English finite clauses.\footnote
 {But, as we will see shortly, certain nonfinite `small clauses' arguably
 use S rather than IP.  The exact nature of the connection between  `finiteness'
 and IP deserves further investigation.}
The conditional clause in (\ref{ArchCond}b) has the same structure, with the
preposed auxiliary replacing the overt complementizer that appears in
(\ref{ArchCond}a).
 
In the systematization of \citet[103]{BresnanEtAl2016}, the functional projections are
distinguished from the lexical ones by having values 1 or 2 for a feature F, whose
value is unspecified for lexical projections.  This implies that the choice of C or I in
the 1 level languages such as Warlpiri and German is arbitrary, with a further
consequence that any tendency in two-level languages to express some things in C
and others in I is probably functional in origin.  It is of course also possible
for there to be no verbal functional projections at all, as argued by
Choi for Korean, and is plausibly also the case for Malayalam and Japanese.


\section{Applications to languages}
In this section we consider the application of the 2001 synthesis to various
languages, starting with Welsh, and then reviewing some of the previous ones
which call for comment.  Malayalam and Korean fit without further discussion,
and so are omitted here.

\subsection{Welsh}
The basic structure for a main clause is:
\ea
\begin{forest}
[IP
     [I$'$  [I [Aux/V]]
         [S
              [NP]
              [VP]
         ]
      ]
]
\end{forest}
\z
If there is no auxiliary, the finite verb appears at the front of the
sentence, in the I position, as shown in (\ref{dragons}a) below, very similarly to
King's (\citeyear{King95}) analysis of Russian.  But if there is an
auxiliary, the auxiliary appears in I, the verb initially in V of VP,
again similarly to Russian, as shown in (\ref{dragons}b), but with the VP under S
dominating the subject, rather than a two-level VP
\citep[131-133]{BresnanEtAl2016}:
\ea\label{dragons}
\begin{xlist}
\item
\begin{forest}
[IP
   [I [gwelodd\\see.3.\SG.\PST]]
   [S
      [NP  [Si\^on\\John]]
      [VP
          [NP  [ddraig\\dragon]]
      ]
    ]
]
\end{forest}\\
`John saw a dragon.'
\item
\begin{forest}
[IP
    [I  [gwnaith\\do.\SG.\PST]]
    [S
        [NP  [Si\^on\\John]]
        [VP
              [V  [weld\\see]]
              [NP  [draig\\dragon]]
        ]
     ]
]
\end{forest}\\
`John saw a dragon.'
\end{xlist}
\z
One issue here is how a VP can appear sitting over only an NP in (\ref{dragons}a). This goes against the idea that an endocentric category needs to have a head, while the presence of
a VP is motivated by the range of things that can appear in the position after the subject, where overt VPs sometimes appear over the same material.

This problem is
averted by the `Extended Head Principle' \citep[135-137]{BresnanEtAl2016},
which in effect says that a phrase can have a `displaced head',
as long as that head appears within a higher phrase having the same
f-structure correspondent.  The definition of `extended head' is:
\ea
Given a c-structure containing nodes $\mathcal{N}$ and $\mathcal{C}$ and a c- to f-structure
correspondence $\phi$, $\mathcal{N}$ is an extended head of $\mathcal{C}$  if and only if
$\mathcal{N}$ is the minimal node in $\phi^{-1}(\phi(\mathcal{C}))$ that c-commands $\mathcal{C}$ without
dominating $\mathcal{C}$. \citep[136]{BresnanEtAl2016}
\z
and the principle is:
\ea
Every (phrasal)\footnote
 {This qualification is absent from the original, but seems to me
 to make the principle work properly.}
 lexical category has an extended head
\z

Although we have discussed only I, Welsh also has a functional projection
C, containing complementizers as discussed by for example \citet{Roberts2005}.
Therefore it is a 2-level language like English, but differs from English in
that the complement of I is S rather than VP.  This is required because
in Welsh, the position of the subject is after the auxiliary rather than
before it.  `I+S' languages, such as for example Tagalog, also often have
the property that S can have PP, AP, and NP as well as VP as the predicate
phrase, but this does not appear to be the case for Welsh, since it uses
a copula in sentences where these play the semantic role of predicate.\footnote
 {\citet[130]{BresnanEtAl2016} suggest that Welsh has these as possible predicates under S,
 but no examples are provided, and Welsh appears in fact to use a copula.
 See \citet{Borsley2019} for a recent discussion of Welsh copular clauses
 in the framework of HPSG.}
Welsh also differs from English (and is similar to most other Germanic languages) in
that all verbs can appear in I, rather than only a restricted class of
`auxiliaries'.
 
\subsection{More English}
As observed above, the fact that English puts the subject in front of I
rather than after it indicates that it has VP rather than S as complement
of I, an analysis corroborated by the fact that a verb is obligatory in finite
clauses (recalling that IP shares category features with its complement, so excludes
a non-verbal complement).  But nevertheless, as observed by \citet{ChungMcCloskey1987},
English does arguably have
nonfinite clauses where S expands to NP subject and a predicate phrase, which
can be NP, AP, VP or PP.
These so-called `small clauses' are used
in English to express a combination of incredulity and often
dismay \citep{Akmajian:Mad}:
\ea\label{esc}
\begin{xlist}
\item What?? Him an alcoholic?? 
\item What?? Her sick with the flu??
\item What?? Him run(ning) a company??
\item What?? That guy in the White House??
\end{xlist}
\z
Along with nonfiniteness comes accusative case on the subject and
no agreement with any verbal element.  This is evidence that in English,
nominative case on NPs is a marker of finiteness on the clause, somewhat
in the manner of the proposals of \citet{PesetskyTorrego2007} within the Minimalist
program, although an LFG analysis would have quite a different implementation,
similar to the treatment of `modal case' in Tangkic languages, as presented
in somewhat different forms by \citet{Andrews1996} and \citet{nordlinger1998constructive}.  In these languages, nominative forms would have an equation specifying an appropriate tense
feature value for whatever they were subject of.  

Another plausible case for S would be gerundive nominalizations with accusative
(rather than genitive) subjects, as analysed by \citet{schachter76}:
\newpage
\ea
John giving/*give an invited talk might be a good idea
\z
A potential issue with having NP expanding to S here is that the predicate phrase
of the S is restricted to being a VP whose verb is marked with {\it -ing}.
But this can be accommodated by including an appropriate constraining equation
in the c-structure rule:
\ea
\phraserule{NP}{ \rulenode{S \\ \UP=\DOWN \\ {(\DOWN \VFORM) =$_c$ \textsc{ing}}}
              }
\z
The constraining equation will require the predicate of the S to be a verb, as
well as a verb marked with {\it -ing}. Taking this analysis further would require
entering the realm of `mixed category constructions', beyond the scope of this chapter.
But there is clearly a question of what causes S to have a rather limited distribution
in English, as opposed to other languages such as Welsh or Tagalog.

\subsection{Russian revised}\label{ltwoconfrev}
Moving on to Russian, King's analysis, discussed in \sectref{russian},
diverges from the previous analyses we have discussed in this section in using a
two-level VP, with the top level introducing a subject.  This difference can
be easily eliminated by replacing the upper level VP with S, but is there
any serious motivation for doing that?

The structure-function mapping principles listed in (\ref{structurefunction})
are not entirely clear on this: (\ref{structurefunction}c) says that specifiers of functional categories
are the grammaticalized discourse functions, which suggests no, but grammaticalized
discourse functions can also be adjoined (\ref{structurefunction}e), which suggests possibly yes.
A general point that suggests that the S analysis is correct is that the
question of what should appear in the specifier of VP in many languages, such
as English, has always been rather controversial.  \citet{fukui1986} proposes
that only functional, not lexical, categories, have specifiers, and this appears
to be consistent with what we have seen here.  On the other hand,
specifier is at least a somewhat plausible place to locate quantity and degree
modifiers of nouns and adjectives, but there could be a weaker position that
specifiers of lexical  categories do not supply arguments to those categories,
but serve different functions when they exist at all.

If we accept this reanalysis, a natural concomitant is to place postposed
focussed subjects and certain other NPs under S, following the VP:
\ea
\phraserule{S}{ \rulenode{(XP) \\ {(\UP \SUBJ)=\DOWN}}
                \rulenode{VP \\ \UP=\DOWN}
                \rulenode{XP$^*$\\ {(\UP \GF)=\DOWN}\\ {(\UP \FOC)=\DOWN}}               }
\z
\citet[210]{King95} put these under V$'$ in order to treat examples
such as:
\ea\label{rusfinsub} Russian \citep[209]{King95}\\
\gll Kupila plat'e Inna.\\
bought dress Inna\\
\glt{`Inna.\FOC bought a dress.'}
\z
But this actually does run against the structure-function mapping
principle (\ref{structurefunction}d).  Using S in this way gives us a version of King's analysis
that is clearly within the framework of the 2001 Synthesis.

\subsection{German again}\label{germanrev}
German, however, presents a problem.  Choi's treatment was within the Synthesis,
in spite of originating in 1996, but \citet{Berman2003} eroded an important
aspect of it, that subjects appear under S.  In particular, following
\citet{Haider1990, Haider1995}, she concluded that
various kinds of presumed subjects were included in VP-preposing, including the
rather hard to dismiss unergatives, which cannot be construed as nominative
objects, as is possible for some of the other examples:
\ea German \citep[36]{Berman2003}\\
\gll Kinder gespielt haben hier noch nie.\\
children played have here still never\\
\glt{`Children have never played here.'}
\z
Her solution was to introduce all arguments, including subjects, under VP
in nested VPs, her VP not being clearly distinct from Choi's $\overline{\hbox{V}}$,
with a complex verb at the bottom, as discussed in a considerable amount
of literature in different frameworks.\footnote
 {For example \citet{Wurmbrand2017} in Minimalism, \citet{zaenen-kaplan1995}
 in LFG.}

Unfortunately, she did not provide an actual tree for this example ((28b) in
the text, nor for the similar (28a)), but I suggest:
\newpage
\ea German\\
\begin{forest}
[CP
    [VP
        [VP
            [NP [N [{Kinder\\children}]]]
            [V [{gespielt\\played}]]
        ]
        [V [{haben\\have}]]
     ]
     [C'
         [VP
            [AdvP [{hier\\here}]]
            [VP
                [AdvP [{noch nie\\never}]]
            [(VP) ]
            ]
          ]
      ]
]
\end{forest}
\z
Where the parenthesized (VP) is an informal notation for the fact that the
head of the unparenthesized VP above it is supplied by the Extended Head Principle
from the Spec of CP position.

This would seem to call for revision of Choi's structures for German such as
(\ref{germansub}a), but, on the other hand, the VP-internal subjects appear
to be restricted; for example, they must be indefinite.  So it is not excluded
that there is both an S where most subjects reside, as proposed by Choi, and
subjects appearing in the VP, as proposed by Haider.  Alternate word orders are
also produced by `scrambling', with strong effects on information structure,
as extensively discussed by Choi.

\subsection{Final remarks}\label{final}
We can sum up the discussion in the previous subsections into a set of rules for when
to posit S as opposed to VP as complement of I and in some other environments.
\ea
\begin{xlist}
\item If SPEC of IP has a subject position, and subjects appear there, rather
than in the complement of I, then the complement of I is VP or possibly other
maximal lexical projections.
\item If subjects appear in the complement rather than in the SPEC of IP, then the complement
of I is S.
\item This can happen in two ways: either S is non-configurational, introducing predicates
and arguments in variable order (with the possibility of the predicates being ordered
at the end or beginning of S), or S is configurational, dominating NP and a predicate phrase
(maximal projection).
\end{xlist}
\z
In this context, we interpret `subject' as the core grammatical function
\citep[97]{BresnanEtAl2016} that is also a discourse (topic) grammatical
function \citep[100]{BresnanEtAl2016}.
We also of course assume S when there is no evidence for I, as in Korean.

The organization of functional projections, on the other hand, is to be ascertained
by the arrangements of elements marked by verbal features with respect to other
members of the clause, with a general tendency for there to be more intervening
items when the projections precede the main verb position than when they follow, leading
to a tendency for verb-final languages to appear to lack verbal functional
projections.

It is clear that many of the languages we have considered are due for careful
reanalysis, due to unresolved issues and discrepancies between earlier and later versions
of LFG, and also taking into account the new phrase structure theory of \citet{low:lov:20}.
Of particular interest would be the nature of clitic placement in Tagalog, and
the issue of flat versus nested structure in non-configurational languages such
as Tagalog and German, where the order of arguments is free, but the predicate is
fixed at one end or the other.

A major study of clause structure that we have not tried to work through is
\citet{sells2001} on Swedish, for the reason that this makes heavy use of Optimality
Theory in ways that have not become mainstream in LFG.  However, word order and clause
structure in Scandinavian languages is an extremely complex and interesting area
that deserves further investigation.

I will conclude with a speculation about the nature of nonconfigurationality. This is that nonconfigurational S is what ensues diachronically when a language       makes such extensive use of discourse-conditioned preposing that the learner gives up trying to analyse it, but instead returns as c-structure something       that is essentially a list of fragments, similar to what the XLE system (\citetv{chapters/ImplementationsApplications}) does when it can’t find a parse.  But the syntax does find an f-structure for the c-structure fragment list appearing under S.
The idea of non-configurational S as a kind of partial failure of c-structure
parsing would be consistent with the revision of the theory of \citet{low:lov:20}
proposed in \citet{Lovestrand2022}, whereby non-configurational S has no category
feature value. 

\section*{Acknowledgements}
I would like to acknowledge the help of Jane Simpson and three anonymous reviewers.

\sloppy
\printbibliography[heading=subbibliography,notkeyword=this]
\end{document}
